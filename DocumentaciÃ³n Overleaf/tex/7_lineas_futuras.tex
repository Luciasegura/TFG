\capitulo{7}{Líneas de trabajo futuras}
El trabajo desarrollado ha permitido estudiar distintos modelos deterministas clásicos de evolución de epidemias, tales como los modelos SI, SIS, SIR y SEIR, así como su extensión con estrategias de vacunación. Además, se ha implementado una aplicación interactiva que permite visualizar y simular la evolución de las epidemias bajo diferentes condiciones. A pesar de estos avances, existen diversas direcciones que podrían explorarse en el futuro para ampliar, enriquecer y mejorar este trabajo. A continuación, se detallan algunas de las líneas futuras más relevantes:
\begin{itemize}
    \item Una posible línea de continuación sería el estudio e implementación de \textbf{modelos estocásticos}, que permiten incorporar la variabilidad aleatoria en la transmisión de la enfermedad. Mientras que los modelos deterministas suponen poblaciones grandes y comportamiento promedio, los modelos estocásticos son más realistas en escenarios con poblaciones reducidas o cuando se desea capturar eventos poco frecuentes pero relevantes. Esto permitiría comparar ambas aproximaciones y analizar en qué contextos una es más adecuada que la otra.
    \item Actualmente, los modelos suponen poblaciones homogéneas. Una extensión natural sería introducir estructuras más complejas como modelos por grupos de edad. Estas mejoras aumentarían el realismo del modelo y permitirían realizar simulaciones más específicas y aplicables a escenarios reales.
    \item Una línea interesante sería realizar un análisis de sensibilidad de los parámetros, con el fin de identificar cuáles tienen mayor influencia sobre el comportamiento del sistema. Además, se podrían utilizar técnicas de optimización o inferencia estadística (como algoritmos genéticos, MCMC o métodos bayesianos) para ajustar los parámetros del modelo a datos reales de forma más precisa y rigurosa.
    \item Aunque ya se ha considerado la vacunación en los modelos SIR y SEIR, se podrían analizar escenarios más realistas y complejos, como vacunación por grupos de riesgo o por etapas; vacunación imperfecta, donde no toda la población vacunada desarrolla inmunidad, omedidas no farmacológicas, como cuarentenas, restricciones de movilidad, distanciamiento social o uso de mascarillas. Esto permitiría evaluar el impacto de distintas políticas públicas y comparar su efectividad.
    \item Una posible evolución de la aplicación desarrollada sería conectarla con bases de datos reales o en tiempo real, de forma que los usuarios puedan visualizar y simular la evolución de epidemias actuales con parámetros ajustados a la realidad. Esto haría la herramienta mucho más útil tanto para investigación como para divulgación o docencia.
    \item La aplicación creada podría ampliarse con nuevas funcionalidades, como comparación simultánea de diferentes modelos o escenarios. Interfaz más visual y adaptable, con gráficos avanzados o mapas. Exportación de resultados para análisis externo. Esto permitiría convertirla en una herramienta más versátil, con aplicaciones educativas o incluso profesionales.
    \item Los modelos desarrollados podrían adaptarse para estudiar otras enfermedades infecciosas con diferentes características. Esto permitiría evaluar la aplicabilidad y robustez de los modelos y extender su utilidad a otras áreas de la epidemiología.
    \item Finalmente, una dirección interesante sería comparar los modelos deterministas implementados con modelos basados en agentes, donde cada individuo es modelado de forma independiente con sus propias reglas de comportamiento. Aunque estos modelos son más complejos y costosos computacionalmente, permiten capturar fenómenos emergentes y heterogeneidades que los modelos clásicos no contemplan.
\end{itemize}