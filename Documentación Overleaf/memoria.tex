\documentclass[a4paper,12pt,twoside]{memoir}

% Castellano
\usepackage[spanish,es-tabla]{babel}
\selectlanguage{spanish}
\usepackage[utf8]{inputenc}
\usepackage[T1]{fontenc}
\usepackage{lmodern} % Scalable font
\usepackage{microtype}
\usepackage{placeins}
\usepackage{float}
\usepackage{graphicx}
\usepackage{subcaption}

\RequirePackage{booktabs}
\RequirePackage[table]{xcolor}
\RequirePackage{xtab}
\RequirePackage{multirow}

% Links
\PassOptionsToPackage{hyphens}{url}\usepackage[colorlinks]{hyperref}
\hypersetup{
	allcolors = {red}
}

% Ecuaciones
\usepackage{amsmath}

% Rutas de fichero / paquete
\newcommand{\ruta}[1]{{\sffamily #1}}

% Párrafos
\nonzeroparskip

% Huérfanas y viudas
\widowpenalty100000
\clubpenalty100000

\let\tmp\oddsidemargin
\let\oddsidemargin\evensidemargin
\let\evensidemargin\tmp
\reversemarginpar

% Imágenes

% Comando para insertar una imagen en un lugar concreto.
% Los parámetros son:
% 1 --> Ruta absoluta/relativa de la figura
% 2 --> Texto a pie de figura
% 3 --> Tamaño en tanto por uno relativo al ancho de página
\usepackage{graphicx}

\newcommand{\imagen}[3]{
	\begin{figure}[!h]
		\centering
		\includegraphics[width=#3\textwidth]{#1}
		\caption{#2}\label{fig:#1}
	\end{figure}
	\FloatBarrier
}







\graphicspath{ {./img/} }

% Capítulos
\chapterstyle{bianchi}
\newcommand{\capitulo}[2]{
	\setcounter{chapter}{#1}
	\setcounter{section}{0}
	\setcounter{figure}{0}
	\setcounter{table}{0}
	\chapter*{#2}
	\addcontentsline{toc}{chapter}{#2}
	\markboth{#2}{#2}
}

% Apéndices
\renewcommand{\appendixname}{Apéndice}
\renewcommand*\cftappendixname{\appendixname}

\newcommand{\apendice}[1]{
	%\renewcommand{\thechapter}{A}
	\chapter{#1}
}

\renewcommand*\cftappendixname{\appendixname\ }

% Formato de portada

\makeatletter
\usepackage{xcolor}
\newcommand{\tutor}[1]{\def\@tutor{#1}}
\newcommand{\tutorb}[1]{\def\@tutorb{#1}}

\newcommand{\course}[1]{\def\@course{#1}}
\definecolor{cpardoBox}{HTML}{E6E6FF}
\def\maketitle{
  \null
  \thispagestyle{empty}
  % Cabecera ----------------
\begin{center}
  \noindent\includegraphics[width=\textwidth]{cabeceraSalud}\vspace{1.5cm}%
\end{center}
  
  % Título proyecto y escudo salud ----------------
  \begin{center}
    \begin{minipage}[c][1.5cm][c]{.20\textwidth}
        \includegraphics[width=\textwidth]{escudoSalud.pdf}
    \end{minipage}
  \end{center}
  
  \begin{center}
    \colorbox{cpardoBox}{%
        \begin{minipage}{.8\textwidth}
          \vspace{.5cm}\Large
          \begin{center}
          \textbf{TFG del Grado en Ingeniería de la Salud}\vspace{.6cm}\\
          \textbf{\LARGE\@title{}}
          \end{center}
          \vspace{.2cm}
        \end{minipage}
    }%
  \end{center}
  
    % Datos de alumno, curso y tutores ------------------
  \begin{center}%
  {%
    \noindent\LARGE
    Presentado por \@author{}\\ 
    en Universidad de Burgos\\
    \vspace{0.5cm}
    \noindent\Large
    \@date{}\\
    \vspace{0.5cm}
    %Tutor: \@tutor{}\\ % comenta el que no corresponda
    Tutores: \@tutor{} -- \@tutorb{}\\
  }%
  \end{center}%
  \null
  \cleardoublepage
  }
\makeatother

\newcommand{\nombre}{Lucía Segura Benito}
\newcommand{\nombreTutor}{Daniel Sarabia Ortiz} 
\newcommand{\nombreTutorb}{Alejandro Merino Gómez} 
\newcommand{\dni}{71830574X} 

% Datos de portada
\title{Modelado determinista de epidemias}
\author{\nombre}
\tutor{\nombreTutor}
\tutorb{\nombreTutorb}
\date{\today}


\begin{document}

\maketitle


\newpage\null\thispagestyle{empty}\newpage

%%%%%%%%%%%%%%%%%%%%%%%%%%%%%%%%%%%%%%%%%%%%%%%%%%%%%%%%%%%%%%%%%%%%%%%%%%%%%%%%%%%%%%%%
\thispagestyle{empty}


\noindent\includegraphics[width=\textwidth]{cabeceraSalud}\vspace{1cm}

\noindent D. \nombreTutor, profesor del departamento de Digitalización, área de Ingeniería de Sistemas y Automática.

\noindent D. Alejando Merino Gómez, profesor del departamento de Digitalización, área de Ingeniería de Sistemas y Automática.

\noindent Expone:

\noindent Que la alumna D.ª \nombre, con DNI \dni, ha realizado el Trabajo final de Grado en Ingeniería de la Salud titulado 'Modelado determinista de epidemias'. 

\noindent Y que dicho trabajo ha sido realizado por el alumno bajo la dirección del que suscribe, en virtud de lo cual se autoriza su presentación y defensa.

\begin{center} %\large
En Burgos, {\large \today}
\end{center}

\vfill\vfill\vfill

% Author and supervisor
\begin{minipage}{0.45\textwidth}
\begin{flushleft} %\large
Vº. Bº. del Tutor:\\[2cm]
D. \nombreTutor
\end{flushleft}
\end{minipage}
\hfill
\begin{minipage}{0.45\textwidth}
\begin{flushleft} %\large
Vº. Bº. del Tutor:\\[2cm]
D. \nombreTutorb
\end{flushleft}
\end{minipage}
\hfill

\vfill

% para casos con solo un tutor comentar lo anterior
% y descomentar lo siguiente
%Vº. Bº. del Tutor:\\[2cm]
%D. nombre tutor


\newpage\null\thispagestyle{empty}\newpage




\frontmatter

% Abstract en castellano
\renewcommand*\abstractname{Resumen}
\begin{abstract}
En este proyecto se estudia el modelado determinista de epidemias mediante el análisis e implementación de modelos compartimentales clásicos como SI, SIS, SIR y SEIR, utilizando herramientas de simulación en \texttt{MATLAB} y \texttt{Simulink}. Se incorporan variantes con vacunación y se desarrolla un controlador PID aplicado al modelo SIR para simular medidas de control como cuarentenas.

El trabajo incluye la simulación con datos hipotéticos y reales, permitiendo evaluar la capacidad predictiva de cada modelo. Además, se diseña una aplicación interactiva que permite al usuario introducir parámetros y visualizar la evolución de la epidemia de forma gráfica y accesible, sin necesidad de conocimientos técnicos avanzados.
\end{abstract}

\renewcommand*\abstractname{Descriptores}
\begin{abstract}
Modelos deterministas, SI, SIS, SIR, SEIR, vacunación, simulación, MATLAB, Simulink, controlador PID, enfermedades infecciosas, App Designer, epidemias, SIDA/VIH, gonorrea, sarampión, COVID-19

\end{abstract}

\clearpage

% Abstract en inglés
\renewcommand*\abstractname{Abstract}
\begin{abstract}
This project focuses on the deterministic modeling of epidemics through the analysis and implementation of classical compartmental models such as SI, SIS, SIR, and SEIR, using simulation tools in \texttt{MATLAB} and \texttt{Simulink}. Variants that include vaccination are incorporated, and a PID controller is developed for the SIR model to simulate control measures such as quarantines.

The models are evaluated using both hypothetical and real data, allowing the predictive capability of each model to be assessed. Additionally, an interactive application is designed to allow users to input parameters and visualize the epidemic’s evolution in a graphical and accessible way, without requiring advanced technical knowledge.
\end{abstract}

\renewcommand*\abstractname{Keywords}
\begin{abstract}
Deterministic models, SI, SIS, SIR, SEIR, vaccination, simulation, MATLAB, Simulink, PID controller, infectious diseases, App Designer, epidemics, HIV/AIDS, gonorrhea, measles, COVID-19

\end{abstract}

\clearpage

% Indices
\tableofcontents

\clearpage

\listoffigures

\clearpage

\listoftables
\clearpage


\mainmatter
\capitulo{1}{Introducción}
Las enfermedades infecciosas han sido una de las principales causas de mortalidad y morbilidad a lo largo de la historia de la humanidad, representando un desafío constante para la salud pública mundial. En los últimos años, la aparición de nuevas epidemias y pandemias ha puesto de manifiesto la necesidad de herramientas efectivas para comprender, predecir y controlar la propagación de estos agentes infecciosos. En este contexto, el modelado matemático se ha consolidado como una herramienta fundamental para el estudio y la gestión de epidemias.

Los modelos epidemiológicos deterministas, basados en sistemas de ecuaciones diferenciales, permiten describir la evolución temporal de una enfermedad en una población, dividiendo a los individuos en diferentes categorías según su estado de salud. Entre los modelos más estudiados se encuentran el SI, SIS, SIR y SEIR, cada uno con características y aplicaciones particulares según el tipo de enfermedad y su dinámica.

Este trabajo se centra en el análisis y comparación de estos modelos clásicos, con especial atención a la incorporación de la vacunación en los modelos SIR y SEIR, asumiendo que la inmunidad inducida por la vacunación es similar a la adquirida tras la recuperación. Asimismo, se desarrolla un controlador PID aplicado al modelo SIR, que simula medidas de intervención como cuarentenas, con el objetivo de evaluar estrategias de control para mitigar la propagación del contagio.

Para validar y comparar los modelos, se han utilizado tanto datos simulados como datos reales, facilitando una visión completa de su comportamiento en diferentes escenarios. Además, se ha diseñado una aplicación gráfica que permite visualizar los resultados de manera interactiva.


\capitulo{2}{Objetivos}

Antes de abordar el desarrollo tanto teórico como práctico del trabajo, es fundamental establecer de forma clara los objetivos que guiarán las etapas del proyecto. Dado el carácter técnico y aplicado del estudio, se han dividido los objetivos en tres categorías. La clasificación va a permitir no solo estructurar el alcance del proyecto desde el punto de vista académico y técnico, sino que también va a reflejar el crecimiento personal y profesional que se espera alcanzar con la realización del trabajo.

\section{Objetivo general}

El objetivo principal de este trabajo es el análisis, diseño e implementación de modelos epidemiológicos deterministas mediante herramientas de simulación tanto en \texttt{MATLAB} como \texttt{Simulink}, con el propósito de estudiar la propagación de enfermedades infecciosas, evaluar estrategias de control y mejorar la comprensión del comportamiento dinámico de una epidemia.

Para ello, se utilizarán tanto parámetros hipotéticos como datos reales extraídos de enfermedades conocidas, con el fin de seleccionar el modelo más adecuado para cada caso y simular su evolución. Como parte importante del proyecto, se desarrollará una aplicación interactiva para facilitar la visualización y el análisis de los resultados obtenidos a partir de los modelos simulados.

\section{Objetivos específicos}

\begin{itemize}
  \item Implementar y comparar diferentes modelos compartimentales (SI, SIS, SIR, SEIR, además de variantes con vacunación) en \texttt{Simulink}, incluyendo sus sistemas de ecuaciones diferenciales.
  \item Analizar el significado de los diferentes parámetros de los modelos (beta, gamma, \(R_0\)…) y estudiar cómo afectan a la evolución temporal de la enfermedad en los diferentes modelos.
  \item Simular diferentes escenarios utilizando parámetros hipotéticos y reales, evaluando la capacidad predictiva del modelo.
  \item Identificar enfermedades reales que puedan ajustarse adecuadamente a cada uno de los modelos epidemiológicos estudiados, recopilando sus datos históricos y simulando su comportamiento en \texttt{Simulink} con los parámetros correspondientes.
  \item Estudiar e incorporar al modelo epidemiológico un compartimento asociado a la vacunación, analizando su efecto sobre la evolución de la epidemia y las condiciones necesarias para lograr la inmunidad colectiva.
  \item Estudiar el efecto de modificar parámetros clave para observar posibles estrategias de control y su impacto sobre el comportamiento de la epidemia.
  \item Explorar la aplicación de técnicas de ingeniería de control que permitan influir activamente sobre la evolución de la epidemia.
  \item Diseñar y desarrollar una aplicación interactiva en \texttt{App Designer} que permita al usuario introducir parámetros, visualizar en tiempo real los resultados de la simulación y analizar gráficamente la evolución del sistema.
\end{itemize}

\section{Objetivos personales}

\begin{itemize}
  \item Profundizar en el conocimiento y aplicación práctica de modelos matemáticos en el ámbito de la ingeniería biomédica y la epidemiología, desarrollando la capacidad de traducir fenómenos reales en representaciones matemáticas comprensibles.
  \item Mejorar las habilidades técnicas en \texttt{MATLAB} y \texttt{Simulink}, especialmente en el diseño de modelos dinámicos, simulación de sistemas y creación de aplicaciones interactivas mediante \texttt{App Designer}.
  \item Familiarizarse con el tratamiento, análisis e interpretación de datos reales asociados a enfermedades infecciosas, y aprender a integrarlos en modelos para ajustar sus parámetros y predecir su evolución.
  \item Desarrollar habilidades en visualización de datos y comunicación de resultados científicos a través de herramientas gráficas, mejorando la capacidad de transmitir información técnica de forma accesible.
  \item Aprender a utilizar \texttt{LaTeX} como herramienta profesional para la redacción de documentos técnicos y científicos, adquiriendo autonomía en la generación de textos estructurados, con fórmulas, referencias y formato académico.
\end{itemize}


\capitulo{3}{Conceptos teóricos}
\setcounter{secnumdepth}{3}
\section{Epidemia, pandemia y epidemiología}
En este trabajo, aunque el enfoque principal es el modelado de epidemias, se utilizarán los términos \textit{epidemia} y \textit{pandemia} de manera indistinta, dado que los modelos deterministas que se aplican no distinguen explícitamente entre ambos conceptos. Estos modelos se limitan a simular la dinámica de transmisión de una enfermedad en una población, sin considerar directamente la escala geográfica o el alcance global del brote.
No obstante, es importante señalar que, si bien los modelos empleados pueden aplicarse tanto a epidemias como a pandemias, los términos no deben considerarse sinónimos, ya que su diferenciación radica en el contexto y la magnitud del fenómeno. Además, se analizarán situaciones reales que incluyen casos de ambos tipos, aprovechando la versatilidad de los modelos utilizados.

A continuación se definen ambos términos.

\subsection{Epidemia}
Una epidemia es la aparición de una enfermedad en una comunidad, zona o país durante un periodo de tiempo determinado, que afecta de forma simultánea o sucesiva, pero constante, a un número elevado de personas. Las causas pueden ser diversas e incluyen agentes infecciosos como virus, bacterias, parásitos u hongos, así como factores ambientales y sociales que favorecen la transmisión de la enfermedad. Se considera epidemia cuando cualquier enfermedad infecciosa se descontrola temporalmente y afecta a una proporción significativa de la población.

Cabe destacar que no todas las epidemias son provocadas por enfermedades contagiosas. Aunque muchas se deben a infecciones que se transmiten entre personas, también pueden originarse por factores como el comportamiento humano, el entorno, vectores (como mosquitos) o enfermedades zoonóticas transmitidas por animales. Incluso enfermedades no transmisibles, como la obesidad o la diabetes, pueden alcanzar niveles epidémicos debido a cambios en los estilos de vida.


\subsection{Pandemia}
Por su parte, una pandemia es una epidemia que se ha propagado a nivel mundial, afectando a un gran número de personas en varios países y continentes. Se caracteriza por la transmisión sostenida de persona a persona y genera un impacto significativo en la salud pública global, así como en la economía y la estructura social. El término “pandemia” implica que la enfermedad ha superado fronteras geográficas y ha afectado a diferentes poblaciones.

\subsection{Grupos vulnerables ante las epidemias}
Es importante reconocer que las epidemias afectan de manera desigual a ciertos grupos poblacionales debido a factores sociales, económicos y biológicos. Entre los más vulnerables se encuentran:
\begin{itemize}
  \item Personas mayores de 60 años, debido al mayor riesgo de complicaciones.
  \item Personas con inmunodeficiencia, ya sea adquirida o congénita.
  \item Mujeres embarazadas, por los cambios fisiológicos que afectan al sistema inmunológico.
   \item Personas con obesidad, ya que se ha identificado como factor de riesgo en enfermedades como el COVID-19.
   \item Personas con enfermedades crónicas, como diabetes, enfermedades cardiovasculares, renales, hepáticas o pulmonares.
   \item Grupos socioeconómicos desfavorecidos, que presentan limitaciones en el acceso a servicios de salud y viven en condiciones precarias. 
\end{itemize}

\subsection{Gestión, control y prevención de epidemias}
La gestión de una epidemia requiere de diversas etapas y estrategias. La detección temprana y la vigilancia epidemiológica son fundamentales para identificar brotes y evitar su propagación. Las evaluaciones de riesgo y vulnerabilidad permiten determinar las áreas y poblaciones más afectadas. La preparación incluye la planificación de recursos, la formación del personal y la implementación de sistemas de alerta temprana.

Entre las estrategias preventivas y de control se encuentran la vacunación, el tratamiento de casos, y la mejora de las condiciones de saneamiento e higiene. La respuesta eficaz a una epidemia exige la coordinación entre distintos actores, incluidos gobiernos, organizaciones internacionales y comunidades locales. Para erradicar la enfermedad, pueden llevarse a cabo campañas masivas de vacunación y otras intervenciones a largo plazo. Un enfoque holístico, que integre factores médicos, sociales, económicos, psicológicos y ambientales, resulta clave para mejorar la prevención y respuesta ante epidemias mediante la aplicación de prácticas innovadoras y enfoques interdisciplinarios.

Los tratamientos para enfermedades epidémicas varían según el agente patógeno. Entre las principales medidas de prevención y control se incluyen:
\begin{itemize}
  \item \textbf{Vacunación}, como herramienta fundamental en la prevención de enfermedades transmisibles.
  \item \textbf{Distanciamiento social y cuarentena}, para limitar el contacto entre personas y reducir la transmisión.
  \item \textbf{Higiene y saneamiento}, mediante el uso de mascarillas, el lavado de manos y la mejora del entorno sanitario.
  \item \textbf{Educación y comunicación}, para informar a la población sobre síntomas, prevención y medidas de respuesta.
  \item \textbf{Refuerzo del sistema de salud}, con inversión en recursos humanos, equipamiento e infraestructuras.
\end{itemize}

\subsection{Ejemplos de epidemias y pandemias a lo largo de la historia}
A lo largo de la historia, distintas epidemias han dejado una huella significativa en la sociedad y la salud pública. Algunos ejemplos relevantes son:
\begin{enumerate}
    \item \textbf{La Peste Negra (1347-1351)}: pandemia de peste bubónica causada por la bacteria Yersinia pestis, que provocó la muerte de entre 25 y 30 millones de personas en Europa, aproximadamente un tercio de la población del continente en ese momento. Sus consecuencias económicas y sociales fueron profundas, incluyendo transformaciones en la estructura feudal y escasez de mano de obra.
    \item \textbf{La gripe española (1918-1919)}: provocada por el virus de la influenza A H1N1, infectó a un tercio de la población mundial y causó la muerte de aproximadamente 50 millones de personas. La alta mortalidad se debió a la falta de tratamientos efectivos y a la sobrecarga de los sistemas de salud. Esta pandemia impulsó el desarrollo de los primeros sistemas de vigilancia epidemiológica.
    \item \textbf{VIH/SIDA (desde 1981)}: el virus de la inmunodeficiencia humana ha causado una pandemia global que ha provocado más de 26 millones de muertes. Ha tenido un impacto desproporcionado en regiones como África subsahariana, aunque se han logrado importantes avances en investigación, prevención y tratamiento.
    \item \textbf{Ébola (2013-2016)}: el brote en África Occidental causó alrededor de 11.000 muertes y colapsó los sistemas de salud locales. La respuesta internacional incluyó el desarrollo de tratamientos y vacunas, así como mejoras en infraestructuras sanitarias.
    \item \textbf{COVID-19 (desde 2019)}: causada por el virus SARS-CoV-2, ha provocado millones de muertes y ha tenido un impacto sin precedentes en la salud pública, la economía y la vida cotidiana. Ha subrayado la importancia de la preparación ante emergencias, la cooperación internacional y la rápida implementación de medidas de salud pública.
\end{enumerate}
Estos ejemplos ilustran cómo las epidemias pueden afectar profundamente a las sociedades y refuerzan la necesidad de una vigilancia epidemiológica sólida, preparación ante emergencias y cooperación global para mitigar sus impactos.

\subsection{Epidemiología}
La epidemiología es la ciencia que estudia la distribución y los determinantes de los eventos relacionados con la salud en poblaciones específicas, así como la aplicación de ese conocimiento para la prevención y control de problemas de salud. Esta disciplina se enfoca en identificar factores de riesgo, causas de enfermedades y en desarrollar e implementar intervenciones para su control.

En el contexto de las epidemias, la epidemiología permite investigar brotes infecciosos localizados en tiempo y espacio, facilitando la identificación de la fuente, el riesgo y los mecanismos de transmisión. Por ejemplo, durante la epidemia de Ébola en África Occidental, los estudios epidemiológicos ayudaron a identificar los patrones de transmisión y a establecer medidas de control efectivas.

En el caso de las pandemias, la epidemiología adquiere una dimensión global, evaluando la propagación de enfermedades en múltiples países. La pandemia de COVID-19 ha demostrado el papel esencial de la epidemiología en la comprensión de los mecanismos de transmisión, en el diseño de medidas de prevención y en la coordinación de respuestas internacionales. La vigilancia epidemiológica global y el uso de tecnologías avanzadas para el diagnóstico y rastreo de contactos han sido herramientas fundamentales en su gestión.

En conclusión, la epidemiología resulta crucial para la identificación, prevención y control de epidemias y pandemias. A través de sus métodos de investigación, permite comprender la distribución de las enfermedades, identificar sus causas y diseñar intervenciones eficaces para proteger la salud pública.



\section{Modelos epidemiológicos  deterministas}
\subsection{Modelos epidemiologícos}
Los modelos epidemiológicos son herramientas matemáticas y computacionales que permiten representar de manera simplificada los procesos biológicos, sociales y ambientales que influyen en la propagación de enfermedades dentro de una población. Estos modelos describen cómo las enfermedades infecciosas (y, en algunos casos, no infecciosas) se transmiten entre individuos y cómo varía el estado de salud de la población a lo largo del tiempo.

En términos generales, un modelo epidemiológico intenta captar las dinámicas clave del contagio, recuperación, inmunidad, nacimiento y muerte, mediante el uso de ecuaciones diferenciales, teoría de probabilidades o simulaciones computacionales. Dependiendo de su complejidad, pueden incorporar factores como heterogeneidad en la población, movilidad, redes de contacto, y respuesta a intervenciones sanitarias.

Entre los objetivos principales de los modelos epidemiológicos se encuentran:
\begin{itemize}
    \item Describir la dinámica de transmisión de enfermedades en poblaciones susceptibles, entender cómo y por qué una enfermedad se propaga o desaparece.
    \item Estimar parámetros clave como la duración de la infección, la tasa de transmisión, número básico de reproducción R0 y la proporción de población que debe ser vacunada para llegar a alcanzar la inmunidad colectiva.
    \item Explorar escenarios hipotéticos, permitiendo simular diversas condiciones y prever la evolución en el tiempo de una epidemia bajo diferentes supuestos.
    \item Evaluar estrategias de intervención y control como vacunación, cuarentena, aislamiento de casos, uso de mascarillas, cierre de escuelas. permiten calcular el impacto potencial de estas medidas antes de aplicarlas.
    \item Guiar decisiones en salud pública, ofreciendo soporte cuantitativo para la toma de decisiones en contextos de brotes, pandemias o planificación preventiva.
    \item Comprender el impacto de factores sociales y demográficos, como la densidad poblacional, la movilidad geográfica, la estructura etaria o el comportamiento humano, sobre la propagación de enfermedades.
    \item Contribuir al diseño de políticas sanitarias efectivas, mediante la identificación de puntos críticos donde las intervenciones pueden ser más eficaces o eficientes.
\end{itemize}

Los modelos epidemiológicos pueden clasificarse, entre otros criterios, según la ausencia o presencia de aleatoriedad:
\begin{itemize}
    \item \textbf{Deterministas}, aquellos en los que la evolución del fenómeno depende de manera unívoca del conjunto de condiciones iniciales y de los parámetros establecidos. Es decir, no existe aleatoriedad en el proceso, si se repite el modelo con los mismos valores, el resultado será siempre el mismo. Este tipo de modelos se basa habitualmente en ecuaciones diferenciales y resulta útil para analizar el comportamiento general de una enfermedad en poblaciones grandes, donde se busca una aproximación global y reproducible de un fenómeno.
    \item \textbf{Estocásticos}, incorporan procesos aleatorios, lo que implica que, aun con el mismo conjunto de variables y parámetros, la solución del modelo puede variar en cada ejecución. Esto permite captar mejor la variabilidad inherente a fenómenos reales, especialmente en contextos donde las poblaciones son pequeñas o existe un alto grado de incertidumbre.
\end{itemize}
	
Para el desarrollo del este trabajo se ha elegido el uso de modelos deterministas, dado que permiten describir de manera clara y estructurada la evolución de una enfermedad en función de condiciones iniciales y parámetros conocidos. Este enfoque facilita el análisis matemático y la interpretación de los resultados. Además, es útil cuando se trabaja con poblaciones grandes y se busca comprender el comportamiento general de propagación de una enfermedad.



\section{Vacunación}
\subsection{Importancia}
La vacunación constituye una de las intervenciones más eficaces y trascendentales en la historia de la salud pública. Su implementación ha permitido prevenir la propagación de enfermedades infecciosas, reducir de forma significativa la morbilidad y la mortalidad asociadas, y, en algunos casos, erradicar por completo determinadas patologías.

Según estimaciones de la Organización Mundial de la Salud (OMS), la vacunación previene entre 3,5 y 5 millones de muertes anuales causadas por enfermedades como la difteria, el tétanos, la tos ferina, la influenza y el sarampión. Gracias a los programas de inmunización masiva, millones de vidas se salvan cada año, y muchas enfermedades han sido controladas de forma notable.

En el contexto de las epidemias, la vacunación no solo protege a los individuos vacunados, sino que también contribuye al establecimiento de la inmunidad colectiva o inmunidad de grupo, lo cual reduce la transmisión comunitaria y protege a las personas que no pueden vacunarse, como los inmunocomprometidos o los niños demasiado pequeños.

Además de los beneficios en salud, la vacunación genera impactos económicos y sociales significativos: reduce los costos médicos directos, disminuye la carga sobre los sistemas sanitarios y mejora la productividad al minimizar las pérdidas laborales y escolares. La vacunación universal, tanto rutinaria como de recuperación, es un componente crítico de la atención médica de calidad.

\subsection{Tipos}
Existen diferentes formas de clasificar la vacunación, según su finalidad, estrategia poblacional y tipo de vacuna utilizada:
\begin{enumerate}
    \item \textbf{Según la finalidad}:
    \begin{itemize}
        \item \textbf{Vacunación preventiva}: se administra antes del contacto con el patógeno, con el objetivo de evitar la aparición de la enfermedad. Es el tipo más común y se aplica ampliamente en calendarios infantiles y adultos.
        \item \textbf{Vacunación terapéutica}: se utiliza cuando la persona ya ha sido infectada, buscando reforzar la respuesta inmunitaria o controlar la progresión de la enfermedad. Es menos común y se investiga especialmente en enfermedades como el VIH o ciertos tipos de cáncer.
    \end{itemize}
    \item \textbf{Según la estrategia poblacional}:
    \begin{itemize}
        \item \textbf{Vacunación rutinaria}: incluida en los calendarios de inmunización regulares, generalmente dirigida a la población infantil o a grupos definidos por edad.
        \item \textbf{Vacunación de campaña}: se aplica en situaciones de emergencia, brotes epidémicos o pandemias, para contener la propagación rápida del patógeno.
        \item \textbf{Vacunación selectiva}: dirigida a grupos de riesgo, como personas mayores, embarazadas o personal sanitario, debido a su vulnerabilidad o exposición.
        \item \textbf{Vacunación masiva}: tiene como objetivo alcanzar rápidamente una amplia cobertura poblacional para lograr inmunidad colectiva, como ocurrió durante la pandemia de COVID-19.
    \end{itemize}
    
    \item \textbf{Según el tipo de vacuna}:
    \begin{itemize}
        \item \textbf{Vacunas inactivadas}: contienen el patógeno muerto, incapaz de causar enfermedad pero capaz de generar inmunidad. 
        \item \textbf{Vacunas atenuadas}: contienen microorganismos vivos debilitados que no causan enfermedad en individuos sanos.
        \item \textbf{Vacunas de subunidades o conjugadas}: incluyen fragmentos del patógeno, como proteínas o azúcares.
        \item \textbf{Vacunas recombinantes}: elaboradas mediante ingeniería genética, donde se insertan genes del patógeno en otros organismos para producir antígenos. 
        \item \textbf{Vacunas de ARN mensajero (ARNm)}: utilizan instrucciones genéticas para que las células del cuerpo produzcan proteínas virales y generen una respuesta inmunitaria.
        \item \textbf{Vacunas vectoriales}: usan un virus modificado (vector) para introducir el material genético del patógeno. 
    \end{itemize}

 
\end{enumerate}

\subsection{Evolución e historia}
La historia de la vacunación es un recorrido fundamental en el desarrollo de la medicina y la salud pública. Se remonta a observaciones empíricas sobre la inmunidad natural tras ciertas enfermedades. En 1796, \textbf{Edward Jenner} desarrolló la primera vacuna al utilizar material de pústulas de vacas infectadas con viruela bovina para prevenir la viruela humana, marcando el inicio de la vacunación moderna.

Durante el siglo XIX, \textbf{Louis Pasteur} realizó importantes aportes al crear vacunas utilizando formas atenuadas de microorganismos, desarrollando inmunizaciones contra enfermedades como la rabia. Su trabajo, junto con la teoría microbiana de la enfermedad y las investigaciones de \textbf{Robert Koch}, permitió el desarrollo de vacunas contra la tuberculosis, la difteria y otras enfermedades infecciosas.

En el siglo XX, los avances en biotecnología, como el cultivo celular, posibilitaron la producción de vacunas contra enfermedades como la poliomielitis, el sarampión, las paperas y la rubéola. Posteriormente, la biología molecular permitió el desarrollo de vacunas recombinantes, como la vacuna contra la hepatitis B.

En la actualidad, la tecnología de ARN mensajero ha revolucionado la vacunología, facilitando el desarrollo rápido de vacunas, como se evidenció con las desarrolladas frente al virus SARS-CoV-2 durante la pandemia de COVID-19.

A lo largo de su evolución, la vacunación ha enfrentado retos importantes, como la hesitación vacunal y la necesidad de mejorar la cobertura global, pero su impacto en la salud pública y el desarrollo social y económico es innegable.

\subsection{Impacto y desafíos actuales}
A pesar del impacto positivo y ampliamente documentado de la vacunación, persisten diversos desafíos que afectan su implementación y sostenibilidad a nivel global:
\begin{itemize}
    \item \textbf{Hesitación vacunal y desconfianza}: la desinformación, las creencias en métodos alternativos o "naturales" y la falta de confianza en los sistemas sanitarios generan una baja aceptación de las vacunas en ciertos grupos. La American Academy of Pediatrics subraya la necesidad de estrategias de comunicación eficaces para combatir estas percepciones erróneas.
    \item \textbf{Desigualdades en el acceso}: en muchas regiones de bajos ingresos, existen barreras económicas, logísticas y estructurales que dificultan la distribución equitativa de vacunas. Iniciativas como \textbf{\textit{Gavi, la Alianza para las Vacunas}}, han mejorado la situación, pero aún persisten importantes brechas.
    \item \textbf{Infraestructura limitada}: los sistemas de salud en países de ingresos bajos y medios suelen carecer del personal capacitado, recursos financieros y logística necesarios para mantener programas de vacunación sostenibles, especialmente en lo relativo a la cadena de frío.
    \item \textbf{Impacto de la pandemia de COVID-19}: la crisis sanitaria global interrumpió numerosos programas de vacunación rutinaria, reduciendo las tasas de cobertura y aumentando la exposición a enfermedades prevenibles. Recuperar estos niveles requiere esfuerzos coordinados para restaurar el acceso y la confianza en los servicios de salud.
    \item \textbf{Resurgimiento de enfermedades}: La disminución en la cobertura vacunal ha provocado brotes recientes de enfermedades como el sarampión en comunidades con baja vacunación. La OMS insiste en la necesidad de mantener altas tasas de cobertura para prevenir estas situaciones.
\end{itemize}


La vacunación es una herramienta esencial para la prevención y el control de enfermedades infecciosas. No obstante, para maximizar su impacto, es imprescindible afrontar los desafíos mencionados mediante políticas públicas eficaces, campañas de información, cooperación internacional y fortalecimiento de los sistemas sanitarios.

\section{SIDA/VIH}
El síndrome de inmunodeficiencia adquirida (SIDA) es una enfermedad causada por la infección del virus de la inmunodeficiencia humana (VIH). El VIH es un retrovirus que ataca y destruye las células CD4+ T, esenciales para el sistema inmunológico, lo que provoca una inmunodeficiencia progresiva. Como consecuencia, las personas infectadas se vuelven más susceptibles a infecciones oportunistas y ciertos tipos de cáncer. 
\subsection{Vías de transmisión}
El VIH se transmite a través de varias vías bien definidas:
\begin{itemize}
    \item Contacto sexual: principal forma de transmisión. Ocurre en relaciones sexuales vaginales, anales y orales sin protección, mediante fluidos genitales o rectales infectados.
    \item Uso compartido de agujas: frecuente entre personas usuarias de drogas intravenosas. Es una vía de alto riesgo por la introducción directa del virus en la sangre.
    \item Transfusiones de sangre y productos sanguíneos: aunque posible, su riesgo es mínimo en países donde se realizan pruebas rigurosas en bancos de sangre.
    \item Transmisión madre-hijo: puede ocurrir durante el embarazo, parto o lactancia. El uso de TAR y cesáreas programadas reduce significativamente el riesgo.
    \item Trasplantes de órganos o tejidos: es raro debido a los controles exhaustivos.
    \item Exposición ocupacional: riesgo bajo en profesionales de la salud si se siguen las precauciones adecuadas.
\end{itemize}

\subsection{Diagnóstico}
El diagnóstico se basa en pruebas recomendadas por los Centros para el Control y la Prevención de Enfermedades (CDC):
\begin{enumerate}
    \item Inmunoensayo combinado Ag/Ab: detecta anticuerpos contra VIH-1/VIH-2 y antígeno p24. Puede identificar la infección 2–3 semanas después de la exposición.
    \item Prueba de diferenciación de anticuerpos: permite distinguir entre VIH-1 y VIH-2.
    \item Prueba de amplificación de ácidos nucleicos (NAAT): se emplea si hay resultados indeterminados. Confirma infecciones agudas por VIH-1.
    \item Pruebas rápidas: Ofrecen resultados preliminares en menos de 20 minutos. Se deben confirmar con pruebas de laboratorio.
\end{enumerate}

\subsection{Tratamiento}
El tratamiento se basa en la terapia antirretroviral (TAR), que ha convertido al VIH en una enfermedad crónica controlable.

\textbf{Opciones terapéuticas}. Los inhibidores de la integrasa (INSTI) constituyen la primera línea de tratamiento, generalmente combinados con tenofovir. También existen regímenes de dos medicamentos, como dolutegravir junto con 3TC, que se pueden emplear tras una evaluación inicial adecuada. Además, se utilizan terapias de acción prolongada, por ejemplo, cabotegravir y rilpivirina, que se administran mensualmente o bimensualmente.

\textbf{Consideraciones clínicas}. Entre los efectos secundarios comunes se incluyen náuseas, cefaleas, acidosis láctica y hepatomegalia con esteatosis. Es fundamental ajustar el tratamiento considerando la función renal y hepática del paciente, así como posibles interacciones medicamentosas.

\textbf{Impacto psicosocial}. El estigma asociado al VIH puede afectar negativamente la adherencia al tratamiento y la calidad de vida de los pacientes. Por ello, el apoyo psicológico y social resulta esencial para mejorar su bienestar general.

\subsection{Estrategias de prevención}
La prevención del VIH/SIDA combina medidas farmacológicas, no farmacológicas y de salud pública.

\textbf{Enfoques farmacológicos}.
La profilaxis preexposición (PrEP) consiste en el uso diario de antirretrovirales en personas con alto riesgo de contraer el virus. La profilaxis postexposición (PEP) implica la administración inmediata de antirretrovirales tras una posible exposición, como relaciones sexuales sin protección o accidentes laborales. Además, el tratamiento como prevención (TasP) se basa en que las personas con carga viral indetectable no transmiten el virus, lo que se resume en el principio U=U (indetectable = intransmisible).

\textbf{Enfoques no farmacológicos}.
Entre las medidas no farmacológicas se encuentran el uso de preservativos como barrera física, la circuncisión masculina que reduce el riesgo de infección, la educación y consejería para promover prácticas sexuales seguras, y los programas de intercambio de agujas que disminuyen la transmisión entre usuarios de drogas intravenosas.

\textbf{Medidas de salud pública}.
Las estrategias de salud pública incluyen el diagnóstico temprano del VIH para iniciar tratamiento oportuno, la prevención de la transmisión de madre a hijo mediante intervenciones médicas, y las intervenciones estructurales destinadas a reducir desigualdades sociales y eliminar barreras al acceso a la atención médica.

\subsection{Progresión clínica}
Tras la infección inicial, se observan tres fases: [10][11][12]
\begin{enumerate}
    \item \textbf{Fase aguda}: síntomas similares a los de una gripe.
    \item \textbf{Fase de latencia clínica}: puede durar años; el paciente es asintomático mientras el virus se replica en niveles bajos.
    \item \textbf{Progresión a SIDA}: ocurre cuando el recuento de CD4+ cae por debajo de 200 células/µL o aparecen infecciones oportunistas (ej. Pneumocystis jirovecii, sarcoma de Kaposi, candidiasis esofágica).
\end{enumerate}
La media del tiempo entre la infección y el desarrollo de SIDA en ausencia de tratamiento es de 8 a 10 años. Factores como la carga viral, coinfecciones o la respuesta inmune individual influyen en la progresión.

\subsection{Evolución e impacto}
Desde los años 80, el impacto del SIDA ha cambiado significativamente gracias al desarrollo de la terapia antirretroviral de gran actividad (TARGA). La incidencia de enfermedades oportunistas se redujo drásticamente, pasando de 30.7 a 2.5 casos por cada 100 años-paciente entre 1994 y 1998, según el estudio EuroSIDA. Además, se ha observado un aumento en el recuento de células CD4+ en los diagnósticos recientes, lo que indica un mejor manejo clínico de la enfermedad. Sin embargo, el impacto del SIDA no ha sido uniforme; en Estados Unidos, por ejemplo, ha habido un aumento de infecciones en mujeres, jóvenes y minorías. Los avances en pruebas diagnósticas, como las pruebas rápidas orales, han facilitado la detección temprana del VIH, aunque persisten desafíos importantes, como la falta de acceso a la atención médica en poblaciones vulnerables y el diagnóstico tardío.


\section{Gonorrea}
La gonorrea es una infección de transmisión sexual (ITS) causada por la bacteria Neisseria gonorrhoeae. Puede afectar el tracto urogenital, recto, faringe y, en casos raros, diseminarse sistémicamente. Es una de las ITS más comunes a nivel mundial, con una incidencia anual estimada de 86.9 millones de casos en adultos.

\subsection{Vías de transmisión}
La gonorrea se transmite principalmente a través del contacto sexual sin protección. Durante las relaciones sexuales vaginales, la bacteria puede infectar el tracto genital tanto en hombres como en mujeres. En el caso de las relaciones sexuales anales, la infección puede afectar el recto. Asimismo, durante el sexo oral sin protección, Neisseria gonorrhoeae puede colonizar la faringe.

Otra vía de transmisión es la perinatal, en la cual una madre infectada puede transmitir la bacteria a su bebé durante el parto, lo que puede provocar conjuntivitis neonatal. También se ha documentado la posibilidad de transmisión al utilizar saliva como lubricante, especialmente entre hombres que tienen sexo con hombres (HSH). Además, aunque de forma menos común, existe evidencia limitada que sugiere que la gonorrea orofaríngea podría transmitirse a través del beso profundo.

\subsection{Diagnóstico}
El diagnóstico de la gonorrea debe considerar las posibles infecciones en el tracto urogenital, recto y faringe. Los métodos principales incluyen pruebas de amplificación de ácidos nucleicos (NAAT), cultivo bacteriano y tinción de Gram.

Las pruebas de amplificación de ácidos nucleicos (NAAT) son el estándar de oro debido a su alta sensibilidad y especificidad. Permiten el uso de diferentes tipos de muestras, como endocervicales, vaginales, uretrales, de orina, rectales y faríngeas. Aunque su uso en sitios extragenitales no siempre está aprobado por la FDA, muchos laboratorios han validado estas pruebas y las utilizan con frecuencia. Además, ofrecen la ventaja de facilitar el autodiagnóstico mediante muestras autocolectadas por los pacientes.

El cultivo bacteriano, aunque menos sensible que las NAAT, sigue siendo fundamental para la detección de cepas resistentes a los antimicrobianos. Requiere muestras como las uretrales, endocervicales, rectales, faríngeas o conjuntivales, y es especialmente útil para realizar pruebas de susceptibilidad en casos de fracaso del tratamiento.

La tinción de Gram puede ser diagnóstica en hombres sintomáticos cuando se observan diplococos gramnegativos intracelulares en secreciones uretrales. Sin embargo, su sensibilidad es menor en mujeres y en sitios extragenitales, por lo que no se recomienda como única prueba diagnóstica en estos casos.

La combinación de estos métodos permite un diagnóstico oportuno y efectivo, lo que es clave para el tratamiento adecuado y la prevención de complicaciones y nuevas transmisiones.

\subsection{Prevención}
Estrategias recomendadas incluyen:
\begin{itemize}
    \item Uso de preservativos: Correcto y consistente en relaciones vaginales, anales y orales.
    \item Educación y consejería: Sobre prácticas sexuales seguras y reducción de riesgos.
    \item Detección y tratamiento de parejas sexuales: Para evitar reinfecciones y transmisión. Se recomienda la terapia de pareja expedita (EPT) cuando el acceso a servicios es limitado.
    \item Cribado regular: Anual para mujeres menores de 25 años, mujeres con factores de riesgo, y hombres que tienen sexo con hombres en todos los sitios de exposición.
    \item Profilaxis postexposición (PEP): No estándar para gonorrea, pero estudios recientes sugieren posible beneficio con doxiciclina en ciertos grupos de alto riesgo.
    \item Promoción de salud pública: Acceso a servicios, reducción del estigma y políticas de salud pública efectivas.
\end{itemize}

\subsection{Clínica y complicaciones}
Los síntomas de la gonorrea varían según el sexo. En los hombres, se manifiesta comúnmente como uretritis con disuria y secreción purulenta. En las mujeres, muchas veces es asintomática, aunque puede presentarse como cervicitis con secreción mucopurulenta y dolor pélvico.

Las complicaciones también difieren. En las mujeres, puede derivar en enfermedad inflamatoria pélvica, infertilidad, embarazo ectópico y dolor crónico. En los hombres, la principal complicación es la epididimitis, que también puede conllevar infertilidad. En ambos sexos, existe el riesgo de desarrollar una infección gonocócica diseminada, caracterizada por dermatitis, tenosinovitis y artritis migratoria.

Las infecciones extragenitales incluyen proctitis, faringitis que suele ser asintomática y conjuntivitis neonatal cuando ocurre transmisión perinatal.

Desde el punto de vista de la salud pública, la gonorrea incrementa el riesgo de transmisión del VIH. Además, la creciente resistencia a los antimicrobianos representa un desafío significativo para su control y tratamiento.

\subsection{Historia y evolución}
La gonorrea ha sido una infección de transmisión sexual prevalente durante miles de años. Desde la introducción de los antibióticos, \textit{Neisseria gonorrhoeae} ha desarrollado resistencia progresiva a cada clase utilizada. Primero se observó con las sulfonamidas y penicilinas en las décadas de 1940 y 1950. Posteriormente, surgió resistencia a las tetraciclinas y macrólidos, y más tarde a las fluoroquinolonas durante las décadas de 1980 y 1990.

En la actualidad, la terapia recomendada que anteriormente incluía el uso dual de ceftriaxona y azitromicina también enfrenta un aumento preocupante en la resistencia, con casos reportados de cepas resistentes a ambos antibióticos. Un ejemplo de ello es el clon FC428, resistente a la ceftriaxona, que se ha diseminado a nivel mundial desde 2017.

Esta evolución subraya la necesidad urgente de desarrollar nuevas estrategias terapéuticas y medidas efectivas de prevención para controlar la propagación de esta infección.








\capitulo{4}{Metodología}
Los modelos y simulaciones que desarrollados a continuación tienen como finalidad representar de forma simplificada el comportamiento de ciertas enfermedades infecciosas mediante sistemas dinámicos deterministas. A partir de datos reales, se emplean distintos modelos epidemiológicos clásicos para estudiar su evolución en diferentes contextos.

Además, se incorporan modificaciones a algunos de estos modelos con el objetivo de evaluar el impacto de estrategias de control, siendo la vacunación la medida principal. La metodología empleada no solo busca describir la propagación de la enfermedad, sino también explorar posibles formas de mitigarla.


Se ha optado por la simplificación de los modelos para su implementación en Simulink ya que 
en la formulación original de los modelos epidemiológicos, se trabaja con proporciones respecto a la población total, lo que implica definir las fracciones susceptibles e infectadas como se oberva en \eqref{prop}:

\begin{equation}
    S_f = \frac{S}{N}, \quad I_f = \frac{I}{N}
\label{prop}
\end{equation}

Donde \( N \) representa la población total. La tasa de variación de la fracción susceptible se expresa como \eqref{ene}:

\begin{equation}
    \frac{dS_f}{dt} = -\beta_f \cdot I_f \cdot S_f
\label{ene}
\end{equation}

Sustituyendo las fracciones por sus expresiones en términos absolutos \eqref{absolitos}.

\begin{equation}
    \frac{d}{dt} \left( \frac{S}{N} \right) = -\beta_f \cdot \frac{I}{N} \cdot \frac{S}{N}
\label{absolitos}
\end{equation}

Lo que implica \eqref{impli}:

\begin{equation}
    \frac{1}{N} \cdot \frac{dS}{dt} = -\beta_f \cdot \frac{I \cdot S}{N^2}
\label{impli}
\end{equation}

Multiplicando ambos lados por \( N \) \eqref{multi}:

\begin{equation}
    \frac{dS}{dt} = -\left( \frac{\beta_f}{N} \right) \cdot I \cdot S
\label{multi}
\end{equation}

En este punto, se redefine el parámetro beta $\beta$ de transmisión como se oberva en la ecuación \eqref{rede} 

\begin{equation}
    \beta = \frac{\beta_f}{N}
\label{rede}
\end{equation}

Lo cual permite reescribir la ecuación \eqref{final} de forma más compacta, sin necesidad de hacer referencia explícita a la población total.

\begin{equation}
    \frac{dS}{dt} = -\beta \cdot I \cdot S
\label{final}
\end{equation}

Esta transformación es algébricamente equivalente a la formulación original. Sin embargo, tiene la ventaja de simplificar la implementación en herramientas como \texttt{Simulink}, ya que se trabaja directamente con cantidades absolutas de población, y el efecto del tamaño de la población total queda incorporado dentro del valor del parámetro \( \beta \). Esto permite mantener la coherencia del modelo y facilita su análisis y simulación computacional.





\section{Modelo SI}
El modelo SI es un modelo determinista simple que divide la población en susceptibles (S) e infectados (I). Asume que no hay recuperación, una vez infectado, el individuo permanece infectado de forma permanente. Útil para enfermedades crónicas o sin inmunidad.

\textbf{Supuestos del modelo}. La población total N es contante y homogénea, por lo que no se tienen en cuenta ni nacimientos ni muertes, ya sean por la enfermedad o por otras causas. Por lo tanto, la población susceptible más la infectada es la población total. No existe recuperación, los individuos infectados permanecen en este estado permanentemente. La transmisión de la enfermedad ocurre por contacto entre individuos susceptibles e infectados.

Se muestra una representación esquemática del modelo SI mediante el diagrama de flujo, como se ve en la figura \ref{fig:diagrama SI}.

\begin{figure}[H]
    \centering
    \includegraphics[width=0.7\textwidth]{img/diagrama_SI.png}
    \caption{Diagrama de flujo modelo SI.}
    \label{fig:diagrama SI}
    \vspace{0.5cm} % Ajusta el espacio vertical entre la imagen y el texto
\end{figure}

También se puede describir el modelo a partir del siguiente sistema de ecuaciones diferenciales \eqref{eq:ecuacion 1 Si} \eqref{eq:ecuacion 2 SI}.

\begin{align}
\frac{dS}{dt} &= -\beta SI \label{eq:ecuacion 1 Si} \\
\frac{dI}{dt} &= \beta SI \label{eq:ecuacion 2 SI}
\end{align}

Donde:
\begin{itemize}
    \item 	La ecuación dS⁄dt (\ref{eq:ecuacion 1 Si}), representa la variación del número de personas susceptibles a lo largo del tiempo. Como los individuos susceptibles se van infectando al entrar en contacto con personas contagiadas, este número disminuye progresivamente. Por esta razón, la derivada tiene signo negativo, expresa una pérdida en el grupo de los susceptibles debida al contagio.
    \item 	La ecuación dI⁄dt (\ref{eq:ecuacion 2 SI}), representa la variación del número de personas infectadas en el tiempo. Como los susceptibles que contraen la enfermedad pasan a formar parte del grupo de infectados, este valor aumenta a medida que progresa la transmisión, por lo que la derivada tiene signo positivo.
    \item El parámetro beta ($\beta$) se denomina tasa de transmisión o de contagio. Representa la probabilidad de que un contacto entre un individuo susceptible y uno infectado resulte en un nuevo contagio. Es un valor clave en el modelo, ya que determina la velocidad con la que la enfermedad se propaga por la población. Como se muestra en la ecuación \eqref{eq:beta} tiene unidades inversas al producto de personas y tiempo.
    \begin{equation}
    \beta = \frac{1}{\text{personas} \cdot \text{tiempo}}
    \label{eq:beta}
    \end{equation}
\end{itemize}

 



Este modelo a enfermedad termina por propagarse a toda la población, ya que no tiene cura. Es interesante para enfermedades víricas crónicas, que causan infección de por vida y no tienen cura. La utilidad del modelo es limitada, ya que la mayoría de las enfermedades infecciosas contemplan la recuperación, lo cual no se refleja en este modelo.

El modelo se implementa en Simulink. Para mostrar su funcionamiento y lo que sería un resultado típico de este modelo, se realiza una simulación utilizando datos aleatorios, con el objetivo de observar el comportamiento del modelo SI en un caso práctico, se oberva en la figura \ref{fig:ejemplo SI}. Se considera una población total de 1000 individuos, con 999 personas susceptibles y 1 persona infectada al inicio. La tasa de transmisión es ~$\beta$ = 0,00005. 

\begin{figure}[H]
    \centering
    \includegraphics[width=0.7\textwidth]{img/modelo_SI_resultado_ejemplo.png}
    \caption{Resultado típico de un modelo SI.}
    \label{fig:ejemplo SI}
    \vspace{0.5cm} % Ajusta el espacio vertical entre la imagen y el texto
\end{figure}

La figura \ref{fig:ejemplo SI} muestra una disminución en el número de personas susceptibles y un aumento en el número de infectados conforme pasa el tiempo. Inicialmente, casi toda la población es susceptible, lo que permite una propagación acelerada de la enfermedad, especialmente en las primeras etapas, debido a la alta probabilidad de contacto entre infectados y susceptibles.
La dinámica está determinada por la tasa de transmisión $\beta$.
La enfermedad se propaga hasta que toda la población es infectada. Los infectados aumentan hasta que ya no quedan individuos susceptibles.


\section{Número básico de reproducción}
En los modelos compartimentales, analizados a continuación, resulta fundamental el papel del número básico de reproducción, $R_0$, un parámetro clave en epidemiología que permite anticipar el comportamiento de un brote epidémico.

Este parámetro se define como el número medio de infecciones secundarias que un solo individuo infectado es capaz de generar durante todo su periodo de infecciosidad, en una población compuesta exclusivamente por individuos susceptibles. Mide el potencial de propagación de la enfermedad en sus primeras etapas, antes de que otros factores como la inmunidad o la intervención sanitaria influyan en su dinámica.
El número básico de reproducción puede expresarse mediante la relación entre la tasa de transmisión y la de recuperación como se ve en la ecuación \ref{eq:ecR0}.

\begin{equation}
R_0 = \frac{\beta}{\gamma}
\label{eq:ecR0}
\end{equation}

Este coeficiente refleja el equilibrio entre la capacidad de la enfermedad para transmitirse y la velocidad con la que los individuos infectados se recuperan y regresan al estado de susceptibles. 
El valor de $R_0$ es fundamental para predecir la evolución del brote:
\begin{itemize}
    \item Si $R_0$ > 1, cada personas infectada contagia a más de una persona, lo que implica que la enfermedad se propaga y puede llegar a establecerse de forma endémica en la población.
    \item Si $R_0$ < 1, cada infectado genera, de media, menos de un caso nuevo, por lo que la enfermedad tiende a desaparecer con el tiempo.
    \item Si $R_0$ = 1, cada persona infectada reemplaza a otra, manteniendo el número de casos estable, pero sin expansión, no se produce un brote epidémico.
\end{itemize}
	
Cuando una enfermedad alcanza un valor de $R_0$ >1, pero sin provocar un crecimiento exponencial descontrolado, puede establecerse en un estado endémico. Esto significa que la enfermedad permanece presente de forma continua en una población o región determinada, con un número de casos que se mantiene relativamente constante a lo largo del tiempo.

La \textbf{endemicidad} implica que se ha alcanzado un equilibrio entre la transmisión del patógeno y los mecanismos que limitan su propagación, como, la inmunidad parcial de la población, la adaptación del agente patógeno a sus huéspedes y las intervenciones sanitarias o cambios de comportamiento.

Es importante aclarar que el hecho de que una enfermedad sea endémica no significa que sea benigna o que no represente un riesgo para la salud pública. De hecho, muchas enfermedades endémicas siguen teniendo un impacto considerable en términos de morbilidad y mortalidad, y requieren esfuerzos constantes de vigilancia, prevención y control.

\section{Modelo SIS}
El modelo SIS representa enfermedades donde no se adquiere inmunidad tras la recuperación. Los individuos infectados se recuperan y vuelven a ser susceptibles, permitiendo reinfecciones. Es útil para estudiar enfermedades endémicas con contagio recurrente.

\textbf{Supuestos del modelo}. En este modelo también se asume que no ocurren ni nacimientos ni muertes, ya sean por la propia enfermedad o por otras circunstancias.

Se muestra una representación esquemática del modelo SIS mediante el diagrama de flujo, representado en la figura \ref{fig:diagrama SIS}.
\begin{figure}[H]
    \centering
    \includegraphics[width=0.7\textwidth]{img/diagrama_SIS.png}
    \caption{Diagrama de flujo modelo SIS.}
    \label{fig:diagrama SIS}
    \vspace{0.5cm} % Ajusta el espacio vertical entre la imagen y el texto
\end{figure}

También se puede representar el modelo mediante el siguiente sistema de ecuaciones diferenciales \eqref{eq:ec1SIS} \eqref{eq:ec2SIS}.

\begin{align}
\frac{dS}{dt} &= -\beta SI + \gamma I \label{eq:ec1SIS} \\
\frac{dI}{dt} &= \beta SI - \gamma I \label{eq:ec2SIS}
\end{align}

Donde:
\begin{itemize}
    \item 	La ecuación dS⁄dt (\ref{eq:ec1SIS}), representa la tasa de cambio de la población susceptible. Esta cantidad disminuye cuando los individuos se infectan, se expresa mediante un término negativo asociado a la tasa de contagio. Sin embargo, también aumenta cuando los individuos infectados se recuperan, ya que no se desarrolla inmunidad y vuelven a ser susceptibles. Por eso la tasa de recuperación aparece con signo positivo.
    \item 	La ecuación dI⁄dt (\ref{eq:ec2SIS}), representa la tasa de cambio de la población infectada. Este valor aumenta como consecuencia de los nuevos contagios, se refleja con término positivo asociado a la tasa de transmisión. A su vez, disminuye cuando los infectados se recuperan y regresan al compartimento de susceptibles, por ello gamma aparece con signo negativo.
    \item 	El parámetro beta ($\beta$), tasa de transmisión o de contagio. Probabilidad de que un contacto entre un individuo susceptible y uno infectado resulte en un nuevo contagio. 
    \item 	El parámetro gamma ($\gamma$) se denomina tasa de recuperación. Indica la probabilidad de que un individuo infectado se recupere y vuelva a ser susceptible, es este contexto. Las unidades de la tasa de recuperación gamma son \[[\text{tiempo}]^{-1}\]  Se interpreta como el inverso del tiempo medio de recuperación \eqref{eq:gammacal}.
    \begin{equation}
    \gamma = \frac{1}{\text{tiempo medio de recuperación}}
    \label{eq:gammacal}
    \end{equation}
\end{itemize}


El modelo se implementa en Simulink. Para mostrar su funcionamiento y lo que sería un resultado típico de este modelo, se realiza una simulación utilizando datos aleatorios, con el objetivo de observar el comportamiento del modelo SIS en un caso práctico, se observa en la figura \ref{fig:ejeSIS}. Se tiene una población total de 1000 personas, de las cuales 990 son susceptibles y 10 son infectados. La tasa de transmisión es $\beta$ = 0,0003. 
Y la tasa de recuperación es de 0,1.



\begin{figure}[H]
    \centering
    \includegraphics[width=0.7\textwidth]{img/modelo_SIS_resultado_ejemplo.png}
    \caption{Resultado típico de un modelo SIS.}
    \label{fig:ejeSIS}
    \vspace{0.5cm} % Ajusta el espacio vertical entre la imagen y el texto
\end{figure}


Rápido incremento en el número de individuos infectados, acompañado de una disminución pronunciada en la población susceptible. Este comportamiento inicial se debe a la alta tasa de transmisión.
Las curvas cambian sus pendientes: la velocidad de transmisión se compensa con la velocidad de recuperación, lo que da lugar a un punto de inflexión. El sistema tiende hacia un estado de equilibrio epidemiológico, donde los nuevos contagios son igual a las recuperaciones. Estabilización de ambas curvas.

Característica del modelo, en el cual los individuos se recuperan, pero no adquieren inmunidad, por lo que regresan al estado de susceptibles y pueden volver a infectarse. La enfermedad nunca desaparece completamente, sino que persiste en la población. Se alcanza así un equilibrio endémico.

Se calcula el número básico de reproducción como se ha explicado y para estos datos es $R_0$. Mayor que 1, la enfermedad se propaga y se mantiene en el tiempo. Este valor también explica el comportamiento observado en la gráfica (\ref{fig:ejeSIS}), crecimiento inicial de los casos seguido de una estabilización, lo que confirma que el borte ha alcanzado un equilibrio endémico estable.






\section{Modelo SIR}
El modelo SIR representa enfermedades donde los infectados se recuperan y adquieren inmunidad permanente. La población se divide en susceptibles, infectados y recuperados. Es útil para estudiar epidemias agudas y calcular el número básico de reproducción.

\textbf{Supuestos del modelo}. En este modelo se asume que la población total permanece constante, no hay nacimientos ni muertes. La inmunidad es permanente. Todos los individuos tienen la misma probabilidad de interactuar.
Aunque es un modelo idealizado, proporciona una herramienta para comprender y predecir el comportamiento de muchas enfermedades infecciosas, facilitando la toma de decisiones en salud pública y el diseño de estrategias de vacunación o contención.

Se muestra una representación esquemática del modelo SIS mediante el diagrama de flujo, representado en la figura \ref{fig:diagrama SIR}.
\begin{figure}[H]
    \centering
    \includegraphics[width=0.7\textwidth]{img/diagrama_SIR.png}
    \caption{Diagrama de flujo modelo SIR.}
    \label{fig:diagrama SIR}
    \vspace{0.5cm} % Ajusta el espacio vertical entre la imagen y el texto
\end{figure}

También se puede representar el modelo mediante el siguiente sistema de ecuaciones diferenciales \eqref{eq:ec1SIR}\eqref{eq:ec2SIR}\eqref{eq:ec3SIR}.

\begin{align}
\frac{dS}{dt} &= -\beta SI \label{eq:ec1SIR} \\
\frac{dI}{dt} &= \beta SI - \gamma I \label{eq:ec2SIR} \\
\frac{dR}{dt} &= \gamma I \label{eq:ec3SIR}
\end{align}

Donde:
\begin{itemize}
    \item 	La ecuación dS⁄dt \eqref{eq:ec1SIR}, explica la disminución de individuos susceptibles debido a nuevos contagios. Depende del número de susceptibles, de infectados y de la tasa de contacto efectivo. El signo es negativo ya que los susceptibles disminuyen con el tiempo.
    \item 	La ecuación dI⁄dt \eqref{eq:ec2SIR}, refleja los dos procesos que afectan a los infectados, el aumento de los nuevos contagios y la disminución por las recuperaciones. La diferencia entre estos dos términos determina si el número de infectados crece o disminuye.
    \item 	La ecuación dR⁄dt \eqref{eq:ec3SIR}, describe como aumenta el número de recuperados. 
    \item El parámetro beta ($\beta$), tasa de transmisión o de contagio. Representa la probabilidad de que un contacto entre un individuo susceptible y uno infectado resulte en un nuevo contagio.
    \item 	El parámetro gamma ($\gamma$), tasa de recuperación. Probabilidad de que un individuo infectado se recupere y pase a recuperado.
\end{itemize}





El modelo se implementa en Simulink. Para mostrar su funcionamiento y lo que sería un resultado típico de este modelo, se realiza una simulación utilizando datos aleatorios, con el objetivo de observar el comportamiento del modelo SIR en un caso práctico, se oberva en la figura \ref{fig:ejemplo SIR}. La población total es de 10000, siendo susceptibles iniciales 9900 individuos, 100 individuos iniciales infectados y 0 individuos recuperados.
La tasa de transmisión es $\beta$ = 0,00003 y la tasa de recuperación es $\gamma$ 0.2



\begin{figure}[H]
    \centering
    \includegraphics[width=0.7\textwidth]{img/modelo_SIR_resultado_ejemplo_.png}
    \caption{Resultado típico de un modelo SIR.}
    \label{fig:ejemplo SIR}
    \vspace{0.5cm} % Ajusta el espacio vertical entre la imagen y el texto
\end{figure}


En la gráfica \ref{fig:ejemplo SIR} se observa un comportamiento típico de una epidemia aguda.
Inicialmente, casi toda la población pertenece al compartimento de susceptibles. Desciende en los primeros días, debido a que una gran proporción de personas se infecta en un corto intervalo de tiempo. Esta rápida disminución refleja una enfermedad altamente contagiosa, que logra alcanzar a prácticamente toda la población susceptible.

El número de infectados crece de forma acelerada, alcanzando un máximo en el que se observa un gran número de personas enfermas a la vez. Sin embargo, a medida que estos infectados se recuperan y pasan al compartimento de recuperados, la curva de infectados comienza a descender.

La curva de recuperados comienza en cero y aumenta progresivamente, hasta que toda la población termina en este compartimento. Todos los individuos se recuperan de la enfermedad y adquieren inmunidad, lo que impide que la infección vuelva a propagarse. La inmunidad permanente es la que permite que la epidemia desaparezca de manera definitiva.

El número básico de reproducción con estos parámetros es $R_0$ = 15, indica que cada personas infectada contagia, de media a 15 individuos. La combinación de alta tasa de transmisión y tasa de recuperación relativamente baja favorece una propagación muy rápida, alcanzando una alta proporción de infectados al mismo tiempo. Esto lo se ve en el resultado de la figura \ref{fig:ejemplo SIR}.

Al ser un modelo SIR, los individuos que se recuperan no pueden volver a infectarse, a diferencia del modelo SIS, donde los recuperados vuelven al grupo de susceptibles. Esta diferencia es clave para entender por qué, en este caso, la epidemia se extingue completamente una vez que se alcanza la inmunidad de grupo, dejando a toda la población en el compartimento de recuperados.




\section{Modelo SEIR}
El modelo SEIR añade un estado de exposición al SIR, para representar enfermedades con periodo de incubación. Los individuos pasan de susceptibles a expuestos, luego a infectados y finalmente a recuperados con inmunidad. Relevante cuando se necesita considerar el impacto del periodo de incubación en la propagación de la enfermedad. Simplificación de la realidad, ofrece una aproximación más ajustada que el modelo SIR para muchas infecciones reales.

\textbf{Suposiciones del modelo}. En este modelo se asume también que la población total es constante, sin nacimientos ni muertes, y que los individuos solo atraviesan una vez cada uno de los estados. Asimismo, se supone una mezcla homogénea, es decir, todos los individuos tienen la misma probabilidad de interactuar entre sí.



Se muestra una representación esquemática del modelo SIS mediante el diagrama de flujo, representado en la figura \ref{fig:diagrama SEIR}.
\begin{figure}[H]
    \centering
    \includegraphics[width=0.7\textwidth]{img/diagrama_SEIR.png}
    \caption{Diagrma de flujo modelo SEIR.}
    \label{fig:diagrama SEIR}
    \vspace{0.5cm} % Ajusta el espacio vertical entre la imagen y el texto
\end{figure}

También se puede representar el modelo mediante el siguiente sistema de ecuaciones diferenciales \eqref{eq:dS_SEIR}\eqref{eq:dE_SEIR}\eqref{eq:dI_SEIR}\eqref{eq:dR_SEIR}. 
\begin{align}
\frac{dS}{dt} &= -\beta SI \label{eq:dS_SEIR} \\
\frac{dE}{dt} &= \beta SI - \sigma E \label{eq:dE_SEIR} \\
\frac{dI}{dt} &= \sigma E - \gamma I \label{eq:dI_SEIR} \\
\frac{dR}{dt} &= \gamma I \label{eq:dR_SEIR}
\end{align}

Donde:
\begin{itemize}
    \item 	La ecuación dS⁄dt \eqref{eq:dS_SEIR}, representa la disminución de individuos susceptibles por el contacto con personas infectadas. La disminución depende del número de susceptibles, del número de infectados y de la tasa de transmisión. Es negativa porque los susceptibles disminuyen con el tiempo al contagiarse.
    \item 	La ecuación dE⁄dt \eqref{eq:dE_SEIR}, describe el cambio en el número de individuos expuestos. Aumenta cuando un susceptible se infecta y disminuye cuando uno expuesto pasa a la fase infecciosa.
    \item 	La ecuación dI⁄dt \eqref{eq:dI_SEIR}, muestra la evolución de los infectados. Aumenta cuando los expuestos se vuelven contagiosos y disminuye por las recuperaciones. El balance entre estos dos procesos determina si el número de infectados crece o decrece.
    \item 	La ecuación dR⁄dt \eqref{eq:dR_SEIR}, refleja el aumento de individuos recuperados, que ya no pueden contagiarse ni contagiar. La velocidad de recuperación está determinada por la tasa de recuperación.
    \item Beta ($\beta$), tasa de transmisión: representa la probabilidad de que un individuo susceptible se infecte tras entrar en contacto con un infectado. 
    \item Sigma ($\sigma$), tasa de incubación: la inversa del tiempo medio que tarda un individuo expuesto en volverse contagioso. Se calcula como en la ecuación \eqref{eq:sigma1}.
    \begin{equation}
    \sigma = \frac{1}{\text{tiempo de incubación}}
    \label{eq:sigma1}
    \end{equation}
    \item Gamma ($\gamma$), tasa de recuperación: indica cuántos infectados se recuperan por unidad de tiempo. Es el inverso del tiempo medio de infección.
\end{itemize}

 

Es importante tener en cuenta la distinción entre individuos infectados e infecciosos, especialmente al comparar diferentes modelos epidemiológicos. En los modelos SIR y SIS se asume que un individuo infectado pasa inmediatamente a ser infeccioso, que puede transmitir la enfermedad en el mismo instante en que se contagia. 
Sin embargo, el modelo SEIR introduce una mejora al incorporar un compartimento de individuos \textbf{expuestos}. Cuando una persona susceptible entra en contacto con un individuo infeccioso, pasa primero al estado de \textbf{“expuesto”}, lo que significa que está infectada pero aún no es capaz de contagiar a otros. Este periodo representa el tiempo de incubación del patógeno, durante el cual el individuo no presenta síntomas y no es detectable clínicamente ni contagioso.
Tiene implicaciones relevantes al momento de establecer las condiciones iniciales para una simulación. 
En la práctica, no es posible conocer con exactitud cuántas personas se encuentran en el estado de exposición, ya que no presentan síntomas, no dan positivo en pruebas diagnósticas y tampoco tienen capacidad de contagiar.
Por esta razón, se va a asumir que el número inicial de expuestos es cero en las simulaciones. Aun así, este compartimento resulta fundamental para modelar de forma más realista enfermedades que incluyen un periodo de incubación significativo.


El modelo se implementa en Simulink. Para mostrar su funcionamiento y lo que sería un resultado típico de este modelo, se realiza una simulación utilizando datos aleatorios, con el objetivo de observar el comportamiento del modelo SEIR en un caso práctico \ref{fig:eje SEIR}. La población total es de 1000, siendo susceptibles iniciales 990 individuos, 10 individuos iniciales infectados y tanto expuestos como recuperados 0 individuos.
La tasa de transmisión es $\beta$ = 0,0005, la tasa de recuperación es $\gamma$ = 0,071 y la tasa de incubación es de $\sigma$ = 0,19 .

\begin{figure}[H]
    \centering
    \includegraphics[width=0.7\textwidth]{img/modelo_SEIR_resultado_ejemplo.png}
    \caption{Resultado típico de un modelo SEIR.}
    \label{fig:eje SEIR}
    \vspace{0.5cm} % Ajusta el espacio vertical entre la imagen y el texto
\end{figure}

Como se observa en la figura \ref{fig:eje SEIR} al inicio, casi toda la población es susceptible, existe alto riesgo de contagio. A medida que los susceptibles se infectan pasan al compartimento de expuestos, lo que hace que el número de susceptibles disminuya rápido en las fases iniciales del brote.

El número de expuestos comienza siendo cero, pero crece rápido conforme se producen nuevos contagios. Estas personas están infectadas, pero no contagiosas debido al período de incubación. Por esta razón, la curva de los expuestos alcanza su pico antes que la de los infectados. 
La población infectada aumenta a medida que los expuestos se vuelven infecciosos. El número de infectados alcanza su máximo cuando el ritmo de nuevos contagios supera al de las recuperaciones. A medida que los infectados se recuperan, el número de personas en este compartimento comienza a descender.

Finalmente, el compartimento de recuperados incrementa a medida que los infectados superan la enfermedad y adquieren inmunidad permanente. Esta curva se estabiliza cuando la mayoría de la población ha pasado ya por la enfermedad, indicando el fin de la epidemia.
Importante calcular el número básico de reproducción con estos parámetros, $R_0$ es 7, con lo que es mayor que 1, por lo tanto la enfermedad tiende a crecer, se ve que en la gráfica \ref{fig:eje SEIR} se cumple este comportamiento.






\section{Medidas de control}

Se han incorporado dos medidas principales de control epidemiológico con el objetivo de reducir el impacto de la propagación de la enfermedad
\begin{itemize}
    \item \textbf{Vacunación preventiva}. Se ha añadido un término de vacunación a los modelos SIR y SEIR, permitiendo que la población susceptible adquiera inmunidad sin necesidad de haber pasado por la enfermedad. Esta estrategia representa una intervención sanitaria proactiva, donde una fracción de la población susceptible pasa directamente al compartimento de los inmunizados o recuperados. De este modo, se reduce la cantidad de personas expuestas al virus y, en consecuencia, el número total de infectados a lo largo del tiempo.
    \item \textbf{Control dinámico mediante regulador PID}. En el modelo SIR, se ha implementado un regulador PID que actúa sobre el parámetro $\beta$, correspondiente a la tasa de transmisión de la enfermedad. El objetivo del controlador es mantener el número de individuos infectados cercano a un valor de referencia o setpoint, evitando así un pico epidémico elevado que pueda sobrecargar el sistema sanitario.
   Este control simula políticas como cuarentenas, distanciamiento social, o uso de mascarillas, cuya intensidad se ajusta automáticamente en función de la evolución de los casos activos. Cuanto más se desvían los infectados del valor deseado, mayor es la intervención del regulador para reducir la transmisión, modificando $\beta$ en tiempo real.
\end{itemize}

\subsection{Mejora del modelo SIR}
Mejorar el modelo SIR como medida de control, incorporando explícitamente el efecto de la vacunación. Esta adaptación se conoce como modelo \textbf{SIRV} (Susceptibles – Infectados – Recuperados – Vacunados) y permite representar de forma más precisa la evolución de enfermedades infecciosas en poblaciones donde existen campañas de inmunización.

En el modelo SIRV, las personas susceptibles pueden infectarse al entrar en contacto con individuos contagiados, pero también tienen la opción de vacunarse como medida preventiva. Se asume que la vacunación confiere inmunidad completa y permanente, lo que implica que los individuos vacunados no pueden contraer la enfermedad ni transmitirla, y permanecen en ese estado de forma indefinida. De este modo, el grupo de vacunados actúa como una barrera adicional a la propagación del virus, reduciendo la proporción de personas susceptibles en la población y limitando la posibilidad de nuevos brotes.

Esta mejora del modelo permite analizar el impacto de diferentes tasas de vacunación sobre la evolución de la enfermedad y estudiar posibles estrategias de control. Además, ofrece una visión más ajustada a la situación actual, en la que la vacunación juega un papel fundamental en la protección individual y colectiva frente a enfermedades.

\textbf{Supuestos del modelo}. Es importante señalar que se mantienen los mismos supuestos básicos que en el modelo SIR. La única diferencia es la incorporación de la vacunación como medida de control de la enfermedad.

Se muestra una representación esquemática del modelo SIS mediante el diagrama de flujo, representado en la figura \ref{fig:ejemplo SIRV}.

\begin{figure}[H]
    \centering
    \includegraphics[width=0.7\textwidth]{img/diagrama_SIRV.png}
    \caption{Diagrma de flujo modelo SIRV.}
    \label{fig:ejemplo SIRV}
    \vspace{0.5cm} % Ajusta el espacio vertical entre la imagen y el texto
\end{figure}

También se puede representar el modelo mediante el siguiente sistema de ecuaciones diferenciales \eqref{eq:ec1SIRV}\eqref{eq:ec2SIRV}\eqref{eq:ec3SIRV}.
\begin{align}
\frac{dS}{dt} &= -\beta SI - \nu S \label{eq:ec1SIRV} \\
\frac{dI}{dt} &= \beta SI - \gamma I \label{eq:ec2SIRV} \\
\frac{dR}{dt} &= \gamma I + \nu S \label{eq:ec3SIRV}
\end{align}

Donde:
\begin{itemize}
    \item 	La ecuación ds⁄dt \eqref{eq:ec1SIRV}, representa la disminución del número de personas susceptibles con el tiempo. Se reduce el número por dos mecanismos contagio y vacunación. Cuando los susceptibles entran en contacto con infectados, se contagian y dejan de ser susceptibles. Cuando los susceptibles se vacunan, también abandonan este estado. El signo negativo indica que la población susceptible disminuye con el tiempo.
    \item 	La ecuación dI⁄dt \eqref{eq:ec2SIRV}, describe la evolución del número de infectados. Aumenta por nuevos contagios y disminuye por recuperación. La diferencia entre los dos términos determina si el número de infectados crece generando un brote epidémico o decrece produciéndose un control de la enfermedad.
    \item 	La ecuación dR⁄dt \eqref{eq:ec3SIRV}, muestra como aumenta el número de personas inmunizadas. Se incluyen tanto a las personas recuperadas de la enfermedad como a los vacunados que pasan directamente al estado inmune. El crecimiento del comportamiento recuperado refleja la inmunidad de la población.
    \item 	Beta ($\beta$): tasa de transmisión o contagio, ya explicada anteriormente.
    \item 	Gamma ($\gamma$): tasa de recuperación, explicada anteriormente.
    \item	Nu ($\nu$): tasa de vacunación. Nuevo parámetro incluido que va a representar la fracción de la población susceptible que se vacuna por unidad de tiempo. Permite estudiar distintos escenarios de intervención sanitaria mediante campañas de vacunación. Sus unidades son
\[
[\text{tiempo}]^{-1}
\]

Es importante señalar que, normalmente, lo que se encuentra en los datos disponibles es la \textbf{cobertura de vacunación}, entendida como la proporción de la población objetivo que ha sido vacunada. Sin embargo, para parametrizar modelos como el SIRV, se requiere la \textbf{tasa de vacunación} ($\nu$), es decir, la velocidad a la que se vacunan los individuos susceptibles.

Suponiendo que la vacunación actúa de forma continua y a una tasa constante sobre los susceptibles, se puede modelar la evolución del compartimento $S$ mediante la siguiente ecuación diferencial \eqref{ecudifer}:


\begin{equation}
\frac{dS}{dt} = -\nu S
\label{ecudifer}
\end{equation}

La solución a esta ecuación es una función exponencial \eqref{funcioex} decreciente:

\begin{equation}
S(t) = S(0) e^{-\nu t}
\label{funcioex}
\end{equation}

La \textbf{cobertura de vacunación} en el tiempo se define como la fracción de la población que ha sido vacunada respecto a la población susceptible inicial, es decir \eqref{cobert}:

\begin{equation}
\text{Cobertura}(t) = 1 - \frac{S(t)}{S(0)} = 1 - e^{-\nu t}
\label{cobert}
\end{equation}

Despejando la tasa de vacunación $\nu$, se obtiene:

\begin{equation}
\nu = \frac{-\ln(1 - \text{Cobertura})}{t}
\label{vacu}
\end{equation}

Esta fórmula \eqref{vacu} permite estimar una tasa de vacunación promedio constante, a partir del tiempo transcurrido y la cobertura alcanzada. Es útil para parametrizar modelos SIRV en contextos donde se dispone de datos agregados de vacunación.





\end{itemize}

En este modelo no se ha incluido un compartimento explícito para los vacunados, ya que se asume que la vacunación proporciona una inmunidad equivalente a la adquirida tras pasar la enfermedad. Por lo tanto, las personas vacunadas son incorporadas de manera directa al compartimento de recuperados, simplificando así el modelo sin perder su capacidad explicativa. 

Es importante señalar que la tasa de transmisión ($\beta$) no varía directamente como consecuencia de la vacunación, ya que se trata de un parámetro biológico y social que describe la probabilidad de contagio efectivo entre un individuo susceptible y uno infectado por unidad de tiempo. Esta tasa depende de factores como la contagiosidad del virus, el número medio de contactos diarios entre personas y las medidas de comportamiento social.
Por tanto, $\beta$ se mantiene constante en el modelo si no cambian estos factores. Lo que se ve afectado es el número de susceptibles. Al disminuir el tamaño del grupo susceptible mediante la inmunización, también se reduce la probabilidad de que se produzcan nuevos contagios, incluso si $\beta$ permanece constante. Aunque el virus mantenga su capacidad de contagio, al haber menos personas susceptibles, disminuye el número efectivo de nuevas infecciones.


\subsection{Control PID para modelo SIR}
Con el objetivo de limitar la cantidad de personas infectadas, se incorporó un mecanismo de control basado en un regulador PID. Este controlador actúa modificando el parámetro $\beta$, lo que equivale a simular la implementación de medidas como el distanciamiento social, el confinamiento o el uso de mascarillas. El propósito del controlador es mantener el número de infectados cerca de un valor deseado\footnote{Setpoint.}.

El sistema fue resuelto numéricamente mediante el método de Euler en MATLAB. Tanto las ecuaciones del modelo SIR como el regulador PID se programaron en código, lo que permitió ajustar fácilmente los parámetros del controlador y analizar su impacto en distintos escenarios.

Los parámetros del controlador PID utilizados:
\begin{itemize}
    \item $K_p$ = 0.01

    Término proporcional: reacciona al error actual entre el número de infectados y el setpoint.
    Un valor moderado como este permite responder al brote sin reacciones bruscas.
    \item $K_i$ = 0.001

    Término integral: tiene en cuenta la acumulación del error en el tiempo.
    Un valor bajo como este evita que el controlador se vuelva inestable por acumulaciones prolongadas.
    \item $K_d$ = 0.01

    Término derivativo: responde a la velocidad de cambio del error (si el brote está creciendo rápido).
Con este valor se suaviza la respuesta del sistema ante cambios bruscos.
\end{itemize}
Seleccionados mediante ajuste manual, buscando una respuesta estable del sistema que redujera el pico de infectados sin generar oscilaciones no realistas. Esta elección permite aplicar medidas graduales, representando un equilibrio entre reacción rápida y control suave del brote.

Además, el valor inicial del error fue error\_prev = 0 y la componente integral acumulada comenzó en error\_int = 0. El PID calcula en cada instante de tiempo una corrección sobre $\beta$, en función del error entre el número actual de infectados y el valor objetivo (setpoint), utilizando la expresión \eqref{eq:beta_PID}.
\begin{equation}
\beta(t) = \beta_0 - K_p \cdot e(t) - K_i \cdot \sum_{i=0}^{t} e(i) \cdot \Delta t - K_d \cdot \frac{e(t) - e(t - 1)}{\Delta t}
\label{eq:beta_PID}
\end{equation}
Donde $e(t) = I(t) - I_{\text{setpoint}}$ es el error en cada paso temporal. Esta corrección permite simular una intervención proporcional a la gravedad del brote en cada momento, activando medidas más o menos estrictas según sea necesario.

En la figura \ref{fig:simu3pid} se muestra un ejemplo de la introducción de un regulador en el modelo. Se puede observar que el pico de infectados no alcanza un valor muy elevado, ya que el número de personas infectadas se mantiene aproximadamente en torno al valor de referencia (setpoint). A medida que la epidemia avanza y disminuye la cantidad de personas susceptibles, el número de infectados tiende a cero.
\begin{figure}[H]
    \centering
    \includegraphics[width=0.7\textwidth]{img/modeloSIR_PID3.png}
    \caption{Simulación 1 para modelo SIR con PID.}
    \label{fig:simu3pid}
    \vspace{0.5cm} % Ajusta el espacio vertical entre la imagen y el texto
\end{figure}


\subsection{Mejora modelo SEIR}
Mejorar el modelo SEIR como medida de control, incorporando el explícitamente el efecto de la vacunación. Esta adaptación del modelo clásico SEIR se conoce como modelo \textbf{SEIRV} (Susceptibles – Expuestos – Infectados – Recuperados – Vacunados), se añaden los vacunados que representa a las personas que reciben una vacuna eficaz y desarrollan inmunidad contra el virus.

Se considera que las personas susceptibles pueden seguir el curso natural de exposición e infección o pasar directamente al grupo de inmunizados mediante vacunación, uniéndose al compartimento de recuperados, bajo el supuesto de que la inmunidad inducida por la vacuna es completa y duradera, al igual que en el caso de los individuos que se recuperan de la enfermedad. Aunque en la realidad la duración de la inmunidad puede variar y algunas vacunas no garantizan protección absoluta, esta simplificación permite estudiar de forma clara el efecto global de la inmunización sobre la propagación del virus.

Se muestra una representación esquemática del modelo SIS mediante el diagrama de flujo, representado en la figura \ref{fig:eje SEIRV}.

\begin{figure}[H]
    \centering
    \includegraphics[width=0.7\textwidth]{img/diagrama_SEIRV.png}
    \caption{Diagrama de flujo del modelo SEIRV.}
    \label{fig:eje SEIRV}
    \vspace{0.5cm} % Ajusta el espacio vertical entre la imagen y el texto
\end{figure}
También se puede representar el modelo mediante el siguiente sistema de ecuaciones diferenciales \eqref{eq:dS_vacunacion}\eqref{eq:dE_SEIRV}\eqref{eq:dI_SEIRV}\eqref{eq:dR_vacunacion}.
\begin{align}
\frac{dS}{dt} &= -\beta SI - \nu S \label{eq:dS_vacunacion} \\
\frac{dE}{dt} &= \beta SI - \sigma E \label{eq:dE_SEIRV} \\
\frac{dI}{dt} &= \sigma E - \gamma I \label{eq:dI_SEIRV} \\
\frac{dR}{dt} &= \gamma I + \nu S \label{eq:dR_vacunacion}
\end{align}
Donde:
\begin{itemize}
    \item 	La ecuación dS⁄dt \eqref{eq:dS_vacunacion} representa la disminución de personas susceptibles con el tiempo. Este número disminuye por dos mecanismos el contagio y la vacunación. Por contagio cuando los susceptibles entran en contacto con infectados, se infectan y dejan de ser susceptibles. La vacunación los susceptibles salen del compartimento al ponerse la vacuna, ganando inmunidad directa sin tener que pasar por la enfermedad.
    \item 	La ecuación dE⁄dt \eqref{eq:dE_SEIRV} muestra el aumento de personas expuestas, personas infectadas que todavía no son contagiosas, debido a nuevos contagios, y su disminución a medida que pasan a estado de infectados. Aumentan por el término de contagio y disminuyen por el paso a infectados.
    \item 	La ecuación dI⁄dt \eqref{eq:dI_SEIRV}indica la variación del número de personas infectadas. Aumenta a medida que los expuestos se vuelven infecciosos y disminuye cuando los infectados se recuperan. Este balance determina su el número de infectados aumenta produciéndose una epidemia, o si por el contrario disminuye y se controla.
    \item La ecuación dR⁄dt \eqref{eq:dR_vacunacion}, muestra el aumento del número de personas inmunizadas. Se suman tanto los infectados que se recuperan como los susceptibles que se vacunan.
    \item 	Beta ($\beta$), tasa de transmisión: representa cuantos contagios ocurren por unidad de tiempo cuando una personas susceptible entra en contacto con una infectada.
    \item Sigma ($\sigma$), tasa de incubación. Inverso del tiempo de incubación.
    \item Gamma ($\sigma$), tasa de recuperación: inverso del tiempo de recuperación.
    \item 	Nu ($\nu$), tasa de vacunación: representa la fracción de la población susceptible que se vacuna por unidad de tiempo.
\end{itemize}

En este modelo no se ha incluido un compartimento explícito para los vacunados, ya que se asume que la vacunación proporciona una inmunidad equivalente a la adquirida tras pasar la enfermedad. Por lo tanto, las personas vacunadas son incorporadas de manera directa al compartimento de recuperados, simplificando el modelo sin perder su capacidad explicativa. Con la tasa de transmisión ocurre lo mismo a lo explicado para el modelo SIRV.




\section{Descripción de los datos}
Para cada uno de los modelos estudiados se ha seleccionado una enfermedad cuya dinámica se ajusta adecuadamente a la estructura del modelo. A continuación, se presentan los datos utilizados y se justifica la elección de cada enfermedad en función de las características del modelo correspondiente. Para más información ir a \textbf{Anexos, Apéndice D, apartado Datos utilizados.}

\subsection{Datos modelo SI}
El modelo SI es apropiado para representar enfermedades infecciosas crónicas como el VIH/SIDA, en las que una vez que una persona se infecta, permanece en ese estado de forma permanente. No contempla la recuperación, lo cual se ajusta al comportamiento del VIH, ya que, aunque existen tratamientos que permiten controlar la carga viral, no eliminan el virus del organismo.

En el modelo, la población se divide en dos grupos: S e I. Dado que el VIH no permite volver al estado susceptible ni alcanzar una recuperación definitiva, este enfoque resulta útil para analizar su propagación y estudiar la evolución de la enfermedad en función de parámetros clave como la tasa de transmisión ($\beta$).

\subsection{Datos modelo SIS}
La gonorrea es un ejemplo representativo de enfermedad que se ajusta al modelo SIS. En este modelo, los individuos pueden recuperarse, pero no adquieren inmunidad permanente, por lo que pueden reinfectarse. Esto refleja la dinámica de la gonorrea, ya que, tras el tratamiento, los pacientes pueden volver a contagiarse si se exponen de nuevo a la bacteria \textit{Neisseria gonorrhoeae}.
Este mecanismo de reinfección continua permite que la enfermedad persista en la población y alcance un equilibrio endémico. Por ello, el modelo SIS es una herramienta útil para estudiar su evolución y diseñar estrategias de control. 



\subsection{Datos modelo SIR}
El sarampión es una enfermedad vírica altamente contagiosa que se ajusta bien al modelo SIR. Este modelo describe enfermedades en las que los individuos, tras la recuperación, adquieren inmunidad permanente, como ocurre con el sarampión.
Presenta una transmisión muy eficiente, con un número básico de reproducción elevado, lo que facilita su propagación en poblaciones no inmunizadas. Además, tiene un periodo infeccioso definido, durante el cual puede transmitirse antes de que el individuo se recupere.

Históricamente, el sarampión causaba epidemias recurrentes, especialmente antes de la vacunación. Incluso hoy, cuando bajan las tasas de inmunización, pueden observarse rebrotes cuya dinámica encaja bien con las predicciones del modelo SIR. Por todo ello, el sarampión es un ejemplo claro de enfermedad modelable mediante este enfoque.



\vspace{2em}
\textbf{Modelo SIRV}. 





\subsection{Datos modelo SEIR}
El COVID-19 es una enfermedad que se ajusta bien al modelo SEIR, debido a la presencia de un periodo de incubación durante el cual los individuos están infectados pero aún no son contagiosos. Este periodo se representa mediante el compartimento de expuestos, característico de este modelo.

Tras la incubación, los individuos pasan a ser infectados, con o sin síntomas, y pueden transmitir el virus. Posteriormente, se recuperan y desarrollan cierta inmunidad, al menos temporalmente. Aunque esta inmunidad no siempre es permanente, el modelo SEIR básico permite representar de forma realista la dinámica de transmisión.
Gracias a esta correspondencia con las fases clínicas de la enfermedad, el modelo SEIR ha sido ampliamente utilizado para analizar la evolución del COVID-19, evaluar medidas de control y estimar la carga sanitaria en distintos escenarios.


\vspace{2em}


\textbf{Modelo SEIRV}. 






\section{Técnicas y herramientas}

\subsection{MATLAB}
\texttt{MATLAB} (acrónimo de MATrix LABoratory) \cite{mathworks_matlab} es una plataforma de programación y entorno de cálculo numérico desarrollada por MathWorks, diseñada para facilitar el análisis de datos, la creación de modelos matemáticos y el desarrollo de algoritmos. Gracias a su estructura basada en matrices y a su sintaxis intuitiva, permite realizar cálculos complejos y procesar grandes volúmenes de datos de forma eficiente. Su lenguaje de programación está optimizado para operaciones matemáticas, lo que lo hace más accesible que lenguajes tradicionales para tareas de análisis técnico.

El entorno incluye un completo IDE\footnote{Entorno de desarrollo integrado} que proporciona herramientas para la edición de código, visualización de datos, depuración y creación de interfaces gráficas. Además, cuenta con una amplia colección de toolboxes\footnote{Paquetes de herramientas adicionales} diseñados para áreas como el procesamiento de señales, control automático, aprendizaje automático o simulación de sistemas dinámicos.

\texttt{MATLAB} está disponible para los principales sistemas operativos. Su uso está extendido tanto en industria como en el ámbito académico, gracias a su versatilidad, documentación extensa y soporte técnico especializado. En entornos universitarios, su acceso suele estar facilitado mediante licencias institucionales, lo que permite a estudiantes y docentes utilizar todas sus funcionalidades sin coste adicional.

Aunque existen alternativas de código abierto , \texttt{MATLAB} se mantiene como una herramienta de referencia debido a su estabilidad, potencia en el tratamiento de datos numéricos y facilidad de uso, especialmente en tareas relacionadas con la ingeniería, las matemáticas aplicadas y las ciencias físicas.

\subsection{Simulink}
\texttt{Simulink} \cite{mathworks_simulink} es un entorno de simulación y diseño basado en diagramas de bloques, desarrollado por MathWorks e integrado de en \texttt{MATLAB}. Está orientado al modelado, simulación y análisis de sistemas dinámicos multidominio, que permite representar y estudiar el comportamiento de sistemas complejos mediante bloques funcionales interconectados. Gracias a su enfoque gráfico, \texttt{Simulink} facilita el desarrollo de modelos intuitivos y modulares sin necesidad de escribir código manual, aunque permite combinarlo con scripts de \texttt{MATLAB} para ampliar su funcionalidad.

\texttt{Simulink} se utiliza ampliamente en campos como el control automático, la electrónica, la ingeniería de comunicaciones, la automoción, la robótica, entre otros. Permite modelar sistemas en tiempo continuo, tiempo discreto o híbridos, incluyendo componentes físicos, señales lógicas, redes de control, sistemas mecánicos, eléctricos o térmicos. Su integración con toolboxes específicos amplía sus capacidades para abordar el modelado físico, la lógica de eventos o la optimización de controladores.

Una de las principales ventajas de \texttt{Simulink} es su capacidad de realizar una simulación previa a la implementación, permitiendo validar el comportamiento del sistema antes de llevarlo a hardware real. Además, ofrece herramientas avanzadas para el análisis de rendimiento, depuración de modelos, generación automática de código en C/C++ o HDL y conexión con sistemas de tiempo real.

En el ámbito académico, \texttt{Simulink} es una herramienta clave para enseñar conceptos de sistemas dinámicos, control y simulación, ya que su interfaz visual mejora la comprensión conceptual y reduce la curva de aprendizaje. Al igual que \texttt{MATLAB}, está disponible en muchas universidades mediante licencias académicas, lo que facilita su uso en proyectos de investigación, desarrollo y trabajos de fin de grado.

\subsection{App Desinger}
\texttt{App Designer} \cite{mathworks_matlab} es una herramienta integrada en \texttt{MATLAB} que permite crear aplicaciones interactivas con GUIs\footnote{Interfaces gráficas de usuario.} de forma visual e intuitiva. Está diseñada para facilitar el desarrollo de aplicaciones profesionales, combinando un entorno de diseño gráfico de componentes (botones, gráficos, deslizadores, tablas, etc.) con un editor de código basado en el lenguaje de \texttt{MATLAB}.

A diferencia de herramientas anteriores como \texttt{GUIDE}, \texttt{App Designer} ofrece una experiencia más moderna, con mayor integración, organización del código orientado a objetos, y funcionalidades avanzadas para el desarrollo de interfaces. Las aplicaciones creadas pueden ejecutarse directamente desde \texttt{MATLAB} o compartirse como aplicaciones independientes.

Esta herramienta es especialmente útil para prototipar algoritmos, visualizar datos de forma dinámica o crear herramientas personalizadas para usuarios finales, sin necesidad de conocimientos profundos en lenguajes de programación externos.

\subsection{LaTeX}
\texttt{LaTeX} \cite{latexproject} es un sistema de composición de textos de alta calidad, especialmente diseñado para la creación de documentos científicos y técnicos que requieren la presentación precisa de fórmulas matemáticas, gráficos y referencias bibliográficas. Debido a su potencia y flexibilidad, \texttt{LaTeX} es el estándar en la academia para la elaboración de informes, tesis y artículos científicos.

Para facilitar el trabajo colaborativo y la edición, se utilizó \texttt{Overleaf}, una plataforma en línea que permite editar, compilar y gestionar documentos \texttt{LaTeX} directamente desde el navegador web, sin necesidad de instalar software adicional. \texttt{Overleaf} también ofrece integración con sistemas de control de versiones y plantillas personalizadas, lo que mejora la organización y la eficiencia en la elaboración del documento.

\subsection{GitHub}
\texttt{GitHub} \cite{githubdocs} es una plataforma de desarrollo colaborativo que permite alojar, gestionar y controlar versiones de proyectos mediante el sistema de control de versiones \texttt{Git}. Su uso facilita el seguimiento del progreso del proyecto, el almacenamiento seguro del código y la colaboración entre múltiples desarrolladores.

En este proyecto, \texttt{GitHub} se ha utilizado como herramienta de control de versiones y respaldo del código fuente, incluyendo los modelos en \texttt{Simulink}, el desarrollo de la aplicación en \texttt{App Designer} y los documentos en \texttt{LaTeX}. Gracias a esta plataforma, ha sido posible mantener un historial detallado de los cambios realizados, identificar errores, volver a versiones anteriores del proyecto cuando ha sido necesario y garantizar una mayor organización y trazabilidad durante el desarrollo.

Además, al ser una plataforma en la nube, \texttt{GitHub} ha permitido trabajar desde diferentes dispositivos sin necesidad de configurar repositorios locales complejos, y ha servido como medio para compartir el proyecto en caso de ser necesario.


\capitulo{5}{Resultados}
En esta sección se presentan los resultados obtenidos mediante la simulación de los modelos SI, SIS, SIR y SEIR, tanto en sus versiones básicas como con vacunación, utilizando Simulink. 

Es importante destacar que el parámetro $\beta$ que se emplea en los modelos corresponde a la tasa de transmisión ajustada o efectiva, aque para simplificar a partir de ahora se denominará solo beta, aunque sea la efectiva.

\section{Comportamiento modelo SI}
Para el modelo SI se han realizado tres simulaciones con distintos valores de $\beta$ y condiciones iniciales, tal como se resume en la Tabla \ref{tab:resultadosSI}. Estas simulaciones permiten analizar cómo varía la propagación de una infección.
\begin{table}[H]
\centering
\begin{tabular}{|c|c|c|c|}
\hline
\textbf{Parámetro} & \textbf{Suimulación 1} & \textbf{Simulación 2} & \textbf{Simulación 3} \\
\hline
Susceptibles (S) & 999 & 9990 & 9990 \\
\hline
Infectados (I)  & 1 & 10   & 10   \\
\hline
\(\beta\)   & 0.00005     & 0.000001 & 0.000008 \\
\hline
\end{tabular}
\caption{Datos usados para ver el comportamiento del modelo SI}
\label{tab:resultadosSI}
\end{table}
A continuación, se presentan los resultados de cada simulación en las figuras (\ref{fig:simulación 1 SI})(\ref{fig:simulacion 2 SI})(\ref{fig:simulacion 3 SI}). 
\begin{figure}[H]
    \centering
    \includegraphics[width=0.7\textwidth]{img/modelo_SI_resultado_ejemplo.png}
    \caption{Resultado modelo SI con beta de 0.00005}
    \label{fig:simulación 1 SI}
    \vspace{0.5cm} % Ajusta el espacio vertical entre la imagen y el texto
\end{figure}

\begin{figure}[H]
    \centering
    \includegraphics[width=0.7\textwidth]{img/modelo_SI_0.1.png}
    \caption{Resultado modelo SI con beta de 0.000001}
    \label{fig:simulacion 2 SI}
    \vspace{0.5cm} % Ajusta el espacio vertical entre la imagen y el texto
\end{figure}

\begin{figure}[H]
    \centering
    \includegraphics[width=0.7\textwidth]{img/modelo_SI_08.png}
    \caption{Resultado modelo SI con beta de 0.000008}
    \label{fig:simulacion 3 SI}
    \vspace{0.5cm} % Ajusta el espacio vertical entre la imagen y el texto
\end{figure}

Como se puede observar en las simulaciones, independientemente del valor de $\beta$, todos los individuos susceptibles acaban infectándose con el paso del tiempo. La tasa de transmisión únicamente influye en la velocidad a la que se propaga la infección, pero no en el resultado final. Esto es coherente con el comportamiento esperado del modelo SI, en el que no existe recuperación ni inmunidad, lo que implica que el número de infectados tiende a igualar la población total a largo plazo.


\subsection{Comportamiento epidemia SIDA/VIH}
A continuación, se presenta la simulación del comportamiento de la epidemia del VIH/SIDA, utilizando los parámetros definidos anteriormente en el apartado de descripción de los datos. La figura \ref{fig:vih} muestra la evolución de la enfermedad a lo largo del tiempo, destacando cómo se comporta la dinámica de contagio bajo el modelo utilizado.

\begin{figure}[H]
    \centering
    \includegraphics[width=0.7\textwidth]{img/modelo_SI.jpg}
    \caption{Resultado modelo SI con los datos reales para el VIH/SIDA}
    \label{fig:vih}
    \vspace{0.5cm} % Ajusta el espacio vertical entre la imagen y el texto
\end{figure}

Tras simularlo la grafica \ref{fig:vih} muestra la evolución de la enfermedad en la región MENA a lo largo de un periodo de 365 días, 1 año, utilizando el modelo SI con los parámetros definidos previamente.
La línea amarilla representa el número de personas susceptibles, las que no han contraído la enfermedad, mientras que a línea azul muestra el número de personas infectadas, aquellas que han adquirido la enfermedad y permanecen en este estado, no hay recuperación.
Al principio de la simulación, casi toda la población es susceptible (400 millones de personas) y solo una pequeña fracción se encuentra infectada (240mil personas). Sin embargo, debido a la tasa de transmisión $\beta$ ajustada, se observa un crecimiento exponencial del número de infectados.
Hacia el día 70, la curva de infectados y la de susceptibles se cruzan: a partir de este punto, la mayoría de la población ya está infectada. Finalmente, tras aproximadamente 100 días, la curva de personas infectadas se estabiliza cerca del total de la población, mientras que la de susceptibles tiende a cero.



\section{Comportamiento modelo SIS}
Para el modelo SIS se han realizado simulaciones con distintos
valores para los parámetros como para las condiciones iniciales. Los datos utilizados se representan en la tabla \ref{tab:datos para modelo SIS}
\begin{table}[H]
\centering
\begin{tabular}{|c|c|c|c|}
\hline
\textbf{Parámetro} & \textbf{Simulación 1} & \textbf{Simulación 2}  & \textbf{Simulación 3}\\
\hline
Susceptibles (S) & 990 & 8000 & 9990\\
\hline
Infectados (I)   & 10   & 2000 & 10  \\
\hline
\(\beta\)        & 0.001 & 0.0002  & 0.0007\\
\hline
\(\gamma\)        & 0.08 & 0.3 & 0.15\\
\hline
\end{tabular}
\caption{Datos usados para ver el comportamiento del modelo SIS}
\label{tab:datos para modelo SIS}
\end{table}

Se presentan los resultados de cada simulación en las figuras (\ref{fig:simulacion 1 SIS})(\ref{fig:simulación 2 SIS})(\ref{fig:Simulación 3 SIS})

\begin{figure}[H]
    \centering
    \includegraphics[width=0.7\textwidth]{img/modelo_SIS_1_08.png}
    \caption{Resultado modelo SIS con beta 0.001 y gamma 0.08}
    \label{fig:simulacion 1 SIS}
    \vspace{0.5cm} % Ajusta el espacio vertical entre la imagen y el texto
\end{figure}

\begin{figure}[H]
    \centering
    \includegraphics[width=0.7\textwidth]{img/modelo_SIS_2_3.png}
    \caption{Resultado modelo SIS con beta 0.0002 y gamma 0.3}
    \label{fig:simulación 2 SIS}
    \vspace{0.5cm} % Ajusta el espacio vertical entre la imagen y el texto
\end{figure}

\begin{figure}[H]
    \centering
    \includegraphics[width=0.7\textwidth]{img/modelo_SIS_7_15.png}
    \caption{Resultado modelo SIS con beta 0.0007 y gamma 0.15}
    \label{fig:Simulación 3 SIS}
    \vspace{0.5cm} % Ajusta el espacio vertical entre la imagen y el texto
\end{figure}

Para la figura \ref{fig:simulacion 1 SIS} se ve que los infectados aumentan ligeramente en un principio y se estabilizan en un valor muy bajo, mientras que los susceptibles pasa lo contrario. La enfermedad se mantiene endémica con proporción baja de infectados, alcanzando un equilibrio.
Para la figura \ref{fig:simulación 2 SIS} los infectados disminuyen rápido hasta desaparecer, mientras que los infectados auumentan hasta llegar a ser todos de nuevo susceptibles. Se evoluciona hacia un estado libre de infección. Eliminación de la enfermedad y recuperación de la población susceptible. Por último, en la figura \ref{fig:Simulación 3 SIS} la enfermedad no desaparece, sino que se mantiene permanente en la población, en este caso con más personas infectadas que susceptibles. El modelo acaba en un equilibrio, donde los contagios y las recuperaciones se equilibran. Los infectados no desaparecen y no se contagia toda la población ya que existe la recuperación. Se convierte en una enfermedad endémica, permanece estable en la población

\subsection{Comportamiento epidemia gonorrea}
A continuación, se presenta  la simulación del comportamiento de la epidemia de la gonorrea, utilizando los parámetros definidos anteriormente en el apartado de descripción de los datos. La figura \ref{fig:simugono} muestra la evolución de la enfermedad a lo largodel tiempo, destacando la dinámica de contagio bajo el modelo utilizado.

\begin{figure}[H]
    \centering
    \includegraphics[width=0.7\textwidth]{img/modelo_SIS.jpg}
    \caption{Resultado modelo SIS con los datos reales para la gonorrea}
    \label{fig:simugono}
    \vspace{0.5cm} % Ajusta el espacio vertical entre la imagen y el texto
\end{figure}

La gráfica obtenida \ref{fig:simugono} tras simular el modelo SIS muestra la evolución temporal de la infección a lo largo de un año. Se observa que inicialmente el número de infectados crece de forma acelerada, mientras que la cantidad de individuos susceptibles disminuye de manera brusca. Esta fase inicial refleja el momento en el que la transmisión domina sobre la recuperación, debido a la elevada proporción de individuos susceptibles y la presencia activa de personas infectadas.
Conforme transcurren los días, la velocidad de crecimiento de la infección disminuye y ambas curvas comienzan a estabilizarse. Esto indica que el sistema alcanza un equilibrio epidemiológico estable, característico del modelo SIS, en el que el número de nuevas infecciones diarias se compensa con el número de recuperaciones. Dado que los individuos recuperados no adquieren inmunidad y vuelven al compartimento de susceptibles, la enfermedad no desaparece, sino que persiste con una proporción constante de la población infectada.
Se calcula el número básico de reproducción en la ecuación \eqref{eq:R0_179}.
\begin{equation}
R_0 = \frac{\beta}{\gamma} = \frac{0.25}{0.14} \approx 1{,}79
\label{eq:R0_179}
\end{equation}
Lo que significa que de media cada persona infectada contagiará aproximadamente a 1,79 personas al día mientras este infectada. Como este valor es mayor que uno, se cumple la condición para que la infección se propague en la población y no desaparezca, como se observa en el resultado de la gráfica. Esto implica que los individuos pueden volver a infectarse. El $R_0$ > 1 indica que la tasa de nuevas infecciones supera a la tasa de recuperaciones, y que la enfermedad se propagará hasta que se alcance un equilibrio endémico. En este equilibrio, la proporción de personas infectadas y susceptibles se mantiene constante en el tiempo.





\section{Comportamiento modelo SIR}
Para el modelo SIR se han realizado varias simulaciones con valores distintos para las condiciones iniciales y parámetros. Los datos que se utilizan se encuentran en la tabla \ref{tab:datos para modelo SIR}.
\begin{table}[H]
\centering
\begin{tabular}{|c|c|c|c|}
\hline
\textbf{Parámetro} & \textbf{Simulación 1} & \textbf{Simulación 2}  & \textbf{Simulación 3}\\
\hline
Susceptibles (S) & 9900 & 8000 & 9900\\
\hline
Infectados (I)   & 100   & 2000 & 100  \\
\hline
Recuperados (R)   &  0   & 0 & 0  \\
\hline
\(\beta\)        & 0.00007 & 0.0002  & 0.0003\\
\hline
\(\gamma\)        & 0.01 & 0.4 & 0.2\\
\hline
\end{tabular}
\caption{Datos usados para ver el comportamiento del modelo SIR}
\label{tab:datos para modelo SIR}
\end{table}

Se observan los resultados para cada simulación en las siguientes figuras (\ref{fig:Simulación 1 SIR})(\ref{fig:Simulación 2 SIR})(\ref{fig:Simulación 3 SIR})

\begin{figure}[H]
    \centering
    \includegraphics[width=0.7\textwidth]{img/modelo_SIR_0701.png}
    \caption{Resultado modelo SIR con beta 0.000007 y gamma 0.01}
    \label{fig:Simulación 1 SIR}
    \vspace{0.5cm} % Ajusta el espacio vertical entre la imagen y el texto
\end{figure}

\begin{figure}[H]
    \centering
    \includegraphics[width=0.7\textwidth]{img/modelo_SIR_24.png}
    \caption{Resultado modelo SIR con beta 0.0002 y gamma 0.4}
    \label{fig:Simulación 2 SIR}
    \vspace{0.5cm} % Ajusta el espacio vertical entre la imagen y el texto
\end{figure}

\begin{figure}[H]
    \centering
    \includegraphics[width=0.7\textwidth]{img/modelo_SIR_32.png}
    \caption{Resultado modelo SIR con beta 0.0003 y gamma 0.2}
    \label{fig:Simulación 3 SIR}
    \vspace{0.5cm} % Ajusta el espacio vertical entre la imagen y el texto
\end{figure}

En la figura \ref{fig:Simulación 1 SIR}, la enfermedad se propaga lentamente pero de forma generalizada, todos los susceptibles acaban infectados y posteriormente recuperados. El pico de infecciones se da alrededor del día 100 con casi 6000 casos. El número básico de reproducción es 7, lo que explica la propagación total. Finalmente, la epidemia se extingue al no quedar susceptibles.
Para la figura \ref{fig:Simulación 3 SIR}, la enfermedad apenas se propaga, los infectados iniciales se recuperan rápidamente sin generar nuevos contagios significativos. La mayoría de la población permanece susceptible. El número básico de reproducción es 0,5, lo que indica que cada infectado contagia de media a menos de una persona. Esto provoca que el brote se extinga de forma rápida sin alcanzar una propagación masiva.
EN la última figura \ref{fig:Simulación 3 SIR} los susceptibles disminuyen lentamente y se estabilizan alrededor de 4000, lo que indica que una gran parte de la población no se infecta. El número de infectados crece levemente al inicio pero desciende rápidamente, y los recuperados aumentan hasta estabilizarse. La enfermedad se propaga de forma moderada, aunque el número básico de reproducción es 1,5, suficiente para permitir propagación, la epidemia no alcanza a toda la población y acaba desapareciendo con un brote limitado.

\subsection{Comportamiento epidemia sarampión}
Se simula el comportamiento de la enfermedad del sarampión, utilizando los datos definidos anteriormente en el apartado de descripción de los datos. La figura \ref{fig:simusara} muestra la evolución del sarampión.

\begin{figure}[H]
    \centering
    \includegraphics[width=0.7\textwidth]{img/modelo_SIR.jpg}
    \caption{Resultado modelo SIR con datos reales para el sarampión}
    \label{fig:simusara}
    \vspace{0.5cm} % Ajusta el espacio vertical entre la imagen y el texto
\end{figure}

En la gráfica (\ref{fig:simusara}) obtenida se refleja claramente el comportamiento típico de una enfermedad infecciosa altamente contagiosa, y más si no hay medidas de control como la vacuna.
Al principio, la mayor parte de la población se encuentra en el grupo de susceptibles. Sin embargo, debido al gran valor de $R_0$, la enfermedad se propaga con rapidez. Como consecuencia los susceptibles descienden bruscamente en los primero días, un gran número de personas contrae la enfermedad en un corto periodo de tiempo. Posteriormente, la curva llega a cero, que refleja que toda la población se ha infectado.
En cuanto a los infectados, la curva presenta un crecimiento rápido al principio de la epidemia, alcanzando un pico endémico sobre el día 12. En este punto se registra el número máximo de personas infectadas de forma simultánea. Tras este pico, los infectados empiezan a disminuir progresivamente. Esto es debido a que las personas se recuperan y pasan al compartimento de los recuperados, y además quedan muy pocos susceptibles, por lo que la transmisión se ralentiza hasta que se detiene.
Por otro lado, los recuperados van creciendo sostenidamente en el tiempo, empezando en cero y creciendo de forma acelerada en un principio. A medida que los infectados se recuperan, la curva sube hasta que toda la población ha superado la enfermedad y ha adquirido inmunidad permanente. Cuando la curva de los recuperados se estabiliza indica que la epidemia ha llegado a su fin, porque no quedan personas susceptibles que puedan volver a infectarse.

\subsection{Comportamiento del sarampión tras la vacunación}
Para este modelo, los datos también se han explicado en el apartado de descripción de los datos. Son los mismos que para sin vacunación, lo que pasa que se añade la tasa de vacunación ahora. La figura \ref{fig:simu saramp vacuna} muestra la evolución del sarampión pero contando que ahora se ha introducido como medida de control la vacunación.

\begin{figure}[H]
    \centering
    \includegraphics[width=0.7\textwidth]{img/modelo_SIRV_inicio.jpg}
    \caption{Resultado modelo SIRV con datos reales para el sarampión, con medida de control de vacunación}
    \label{fig:simu saramp vacuna}
    \vspace{0.5cm} % Ajusta el espacio vertical entre la imagen y el texto
\end{figure}

En la gráfica \ref{fig:simu saramp vacuna} las personas susceptibles descienden muy rápido en los primero días, más pronunciado que la simulación sin vacunación. Se debe a que gran parte de los susceptibles son vacunados directamente, lo que reduce la cantidad de personas expuestas al contagio. Prácticamente en los primeros días desaparece el grupo de susceptibles, lo que limita la propagación del virus.
En cuanto a los infectados, aunque se alcanza un pico de infecciones, es más bajo que en la simulación sin vacunación. Se sitúa en torno a 11 millones de infectados, además se produce más temprano, que indica que el brote es más contenido y se disipa más rápido por la inmunidad inducida por la vacunación.
Mientras que el número de recuperados crece rápidamente desde los primeros días. A diferencia del modelo sin vacunación, donde el crecimiento del grupo recuperado dependía únicamente de las personas que superaban la enfermedad, en esta simulación también se incluyen los vacunados, que pasan directamente a este compartimento. Por ello, la curva de recuperados comienza a crecer antes y con mayor pendiente, alcanzando valores más altos en menos tiempo y reflejando una inmunidad colectiva mucho más rápida.

La comparación entre ambas simulaciones muestra claramente que la intorducción de la vacunación masiva es altamente efectiva para contener la propagación del sarampión. Reduce significativamente el número de casos, la duración del brote y la exposición de la población susceptible. Esto evidencia la importancia de las campañas de vacunación como herramienta clave de salud pública, especialmente frente a enfermedades con un número básico de reproducción elevado como el sarampión.


\subsection{Comportamiento modelo SIR con regulador PID}
Con el objetivo de mejorar el control sobre lo representado por el modelo SIR, se propone la implementación de un controlador PID. Este controlador permitirá ajustar dinámicamente una variable de control, la tasa de transmisión, en función del error entre el número actual de individuos infectados y un valor deseado (setpoint), que en este caso será de 1000 infectados. Para el diseño y ajuste del PID, se utilizarán los datos obtenidos previamente a partir de las simulaciones realizadas con el modelo SIR estándar, sin control. Estos datos permitirán identificar el comportamiento del sistema y evaluar la efectividad del controlador en mantener el número de infectados cercano al valor establecido.

En la figuras (\ref{fig:simu1pid})(\ref{fig:simu2pid})(\ref{fig:simu3pid}) se observan las simulaciones para el modelo SIR controlado con un regulador PID.

\begin{figure}[H]
    \centering
    \includegraphics[width=0.7\textwidth]{img/modeloSIR_PID1.png}
    \caption{Simulación 1 para modelo SIR con PID, $\beta$ = 0.07 y $\gamma$ = 0.1}
    \label{fig:simu1pid}
    \vspace{0.5cm} % Ajusta el espacio vertical entre la imagen y el texto
\end{figure}

\begin{figure}[H]
    \centering
    \includegraphics[width=0.7\textwidth]{img/modeloSIR_PID2.png}
    \caption{Simulación 1 para modelo SIR con PID, $\beta$ = 0.2 y $\gamma$ = 0.4}
    \label{fig:simu2pid}
    \vspace{0.5cm} % Ajusta el espacio vertical entre la imagen y el texto
\end{figure}

\begin{figure}[H]
    \centering
    \includegraphics[width=0.7\textwidth]{img/modeloSIR_PID3.png}
    \caption{Simulación 1 para modelo SIR con PID, $\beta$ = 0.3 y $\gamma$ = 0.2}
    \label{fig:simu3pid}
    \vspace{0.5cm} % Ajusta el espacio vertical entre la imagen y el texto
\end{figure}

En la figura \ref{fig:simu1pid}, la enfermedad se propaga lentamente ya que el número de infectados crece de forma moderada y luego disminuye suavemente. En cuanto a los susceptibles bajan poco a poco, mientras que los recuperados aumentan gradualmente. El brote es controlado y no hay un pico muy alto. En la figura \ref{fig:simu2pid}, la infección se propaga más rápidamente, pero también hay una recuperación rápida. Se alcanza un pico alto de infectados en poco tiempo, luego desciende rápidamente.
Gran parte de la población termina recuperada en un tiempo corto. El proceso es más brusco, pero de corta duración. Y por últmo en cuanto a la figura\ref{fig:simu3pid}, se propaga muy rápido y con fuerza. El número de infectados crece rápidamente a un valor alto y tarda más en bajar, mientras que el número de susceptibles cae de manera abrupta. Se observa una epidemia fuerte y más prolongada.


\section{Comportamiento modelo SEIR}
Para el modelo SEIR se han realizado varias simulaciones con valores distintos. Los datos que se utilizan se encuentran en la tabla \ref{tab:datos para modelo SEIR}.

\begin{table}[H]
\centering
\begin{tabular}{|c|c|c|c|}
\hline
\textbf{Parámetro} & \textbf{Simulación 1} & \textbf{Simulación 2}  & \textbf{Simulación 3}\\
\hline
Susceptibles (S) & 9900 & 8000 & 9900\\
\hline
Expuestos (R)   &  0   & 0 & 0  \\
\hline
Infectados (I)   & 100   & 2000 & 100  \\
\hline
Recuperados (R)   &  0   & 0 & 0  \\
\hline
\(\beta\)        & 0.0001 & 0.0001  & 0.0006\\
\hline
\(\sigma\)        & 0.1 & 0.2  & 0.3\\
\hline
\(\gamma\)        & 0.05 & 0.1 & 0.1\\
\hline
\end{tabular}
\caption{Datos usados para ver el comportamiento del modelo SEIR}
\label{tab:datos para modelo SEIR}
\end{table}

Se observan los resultados para cada simulaación en las figuras siguientes (\ref{fig:Simulación 1 SEIR})(\ref{fig:Simulación 2 SEIR})(\ref{fig:Simulación 3 SEIR}).

\begin{figure}[H]
    \centering
    \includegraphics[width=0.7\textwidth]{img/modelo_SEIR_1105.png}
    \caption{Resultado modelo SEIR con beta 0.0001, sigma 0.1 y gamma 0.05}
    \label{fig:Simulación 1 SEIR}
    \vspace{0.5cm} % Ajusta el espacio vertical entre la imagen y el texto
\end{figure}

\begin{figure}[H]
    \centering
    \includegraphics[width=0.7\textwidth]{img/modelo_SEIR_121.png}
    \caption{Resultado modelo SEIR con beta 0.0001, sigma 0,2 y gamma 0.1}
    \label{fig:Simulación 2 SEIR}
    \vspace{0.5cm} % Ajusta el espacio vertical entre la imagen y el texto
\end{figure}

\begin{figure}[H]
    \centering
    \includegraphics[width=0.7\textwidth]{img/modelo_SEIR_631.png}
    \caption{Resultado modelo SEIR con beta 0.0006, sigma 0,3 y gamma 0.01}
    \label{fig:Simulación 3 SEIR}
    \vspace{0.5cm} % Ajusta el espacio vertical entre la imagen y el texto
\end{figure}

En la primera gráfica \ref{fig:Simulación 1 SEIR} el número de personas expuestas crece inicialmente y luego decrece. Los infectados presentan un pico, moderado, debido a la baja tasa de transmisión y la recpueración lenta. Los recuperados aumentan progresivamente. Los susceptibles descienden de manera paulatina. En cuanto a la gráfica \ref{fig:Simulación 2 SEIR}, a pesar de que el paso es más rápido de expuestos a infectados y hay mayor tasa de recuperación, el número de infectados se mantiene muy bajo. El brote está contenido, sin un pico relevante. Gran parte de la población sigue siendo susceptible, con pocos recuperados. El sistema se mantiene estable, no hay propagación significativa. Por último la figura \ref{fig:Simulación 3 SEIR}. El sistema reacciona con una rápida propagación, habiendo un pico marcado de infectados.
La mayoría de la población pasa a la categoría de recuperados.
Los susceptibles caen bruscamente, mostrando que la infección se ha propagado de forma extensa.

\subsection{Comparación modelo SIR y modelo SIR con regulador PID}
Para analizar el impacto del control sobre la evolución de una epidemia, se han comparado los resultados de simulaciones del modelo SIR con y sin la incorporación de un controlador PID. En ambos casos, se utilizaron los mismos valores para los parámetros $\beta$ y $\gamma$, permitiendo evaluar directamente el efecto del regulador sobre la dinámica del sistema.

Para la primera simulación con beta = 0.07 y gamma = 0.01, se observa que la propagación de la enfermedad es lenta debido a un valor muy bajo de $\beta$. En la figura \ref{fig:Simulación 1 SIR} sin PID, se observa que el número de infectados crece paulatinamente hasta alcanzar un pico considerable, y luego desciende lentamente a medida que más personas se recuperan. El sistema evoluciona de forma natural, sin intervención, y se aprecia un brote prolongado en el tiempo. En cambio, en la figura \ref{fig:simu1pid} con PID, el número de infectados se mantiene más contenido durante todo el proceso. El controlador actúa corrigiendo la tasa de contagio de forma dinámica, evitando que el pico de infectados sea muy alto. Además, la epidemia se resuelve en menos tiempo. Este caso demuestra que incluso en escenarios con transmisión lenta, el PID puede mejorar la respuesta del sistema al acelerar el descenso de los infectados y reducir la duración total del brote.

En cuanto a la segunda simulación con beta = 0.2 y gamma = 0.4, se plantea una tasa de recuperación elevada, lo que en teoría debería limitar la duración de la infección. En la simulación sin PID figura \ref{fig:Simulación 2 SIR}, el número de infectados aumenta inicialmente, pero se estabiliza en un valor constante. Este comportamiento sugiere que la enfermedad se vuelve endémica, manteniéndose presente en la población a lo largo del tiempo sin llegar a desaparecer del todo. Con el regulador PID activo figura \ref{fig:simu2pid}, el comportamiento cambia drásticamente. El número de infectados alcanza un valor cercano al setpoint y posteriormente disminuye hasta llegar prácticamente a cero. El controlador, al actuar sobre la tasa de contagio, permite reducir el número de infectados de manera más eficaz, eliminando la posibilidad de una infección sostenida en el tiempo. Esto pone en evidencia la capacidad del PID para forzar al sistema hacia un estado libre de infección, incluso cuando la dinámica natural tiende al equilibrio.



\subsection{Comportamiento pandemia COVID-19}
Se simula el comportamiento de la enfermedad del sarampión, utilizando los datos definidos anteriormente en el apartado de descripción de los datos. La figura \ref{fig:Simucov} muestra la evolución del COVID-19 en España desde su inicio.

\begin{figure}[H]
    \centering
    \includegraphics[width=0.7\textwidth]{img/modelo_SEIR.jpg}
    \caption{Resultado modelo SEIR con datos reales para el COVID-19}
    \label{fig:Simucov}
    \vspace{0.5cm} % Ajusta el espacio vertical entre la imagen y el texto
\end{figure}

En la simulación \ref{fig:Simucov} se pueden obtener las siguientes observaciones. La población susceptible disminuye rápidamente, al inicio prácticamente la mayoría de la población era susceptible. mientras avanza el tiempo, disminuyen de manera pronunciada, debido al contacto entre las personas debido al contacto entre personas infectadas. La curva se estabiliza, gran parte de la población se ha contagiado, pero siguen quedando personas susceptibles.
La curva de los expuestos presenta un crecimiento inicial que se adelanta a la curva de los infectados, lo cual es coherente con este modelo, ya que las personas primero pasan por el periodo de incubación antes de convertirse en transmisores activos. El pico de personas infectadas se alcanza alrededor del día 50 aproximadamente, en este momento se observa mayor número de casos activos simultáneos. La curva desciende ya que los infectados se recuperan y los susceptibles disminuyen.
La curva de los recuperados mantiene un crecimiento constante, que se acelera con el pico de infecciones. Hacia el final de la simulación, la mayoría de la población se encuentra en el compartimento de recuperados, indica que la enfermedad ha dejado de propagarse de forma activa por falta de individuos susceptibles.
Aproximadamente a partir del día 150, el sistema entra en una fase de estabilidad. Los valores de todos los compartimentos se estabilizan, y prácticamente no hay nuevos contagios.  La enfermedad ha agotado su capacidad de transmisión debido a que casi toda la población ha adquirido inmunidad.
El retraso entre las curvas de expuestos e infectados muestra claramente el efecto del periodo de latencia. Este desfase temporal tiene consecuencias importantes desde el punto de vista epidemiológico, ya que dificulta la detección temprana de brotes y puede permitir la transmisión antes de que se manifiesten los síntomas.

\subsection{Comportamiento COVID-19 con vacunación}
Para este modelo, los datos también se han explicado en el apartado de descripción de los datos. Son los mismos que para sin vacunación, lo que pasa que se añade la tasa de vacunación. La figura \ref{fig:Simucov vacunacion} muestra la evolución del sarampión

\begin{figure}[H]
    \centering
    \includegraphics[width=0.7\textwidth]{img/modelo_SEIRV.jpg}
    \caption{Resultado modelo SEIRV con datos reales para el COVID-19, con medida de control de vacunación}
    \label{fig:Simucov vacunacion}
    \vspace{0.5cm} % Ajusta el espacio vertical entre la imagen y el texto
\end{figure}

Tras implementar el modelo SEIRV con la incorporación de la vacunación masiva en la población, se puede observar un cambio sustancial en la evolución de la epidemia de COVID-19 respecto al modelo SEIR sin vacunación \ref{fig:Simucov vacunacion}.
En la gráfica obtenida con el modelo SEIRV, se aprecia cómo el número de infectados alcanza un pico considerablemente menor y de menos duración en comparación con la simulación sin vacunación. Este resultado evidencia que la campaña de vacunación contribuye significativamente a reducir la carga máxima del sistema sanitario, evitando un colapso en momentos críticos.
El número de susceptibles disminuye rápidamente debido a dos factores: el contagio y la vacunación. A diferencia del modelo sin vacunación, donde la disminución se debía únicamente a la transmisión del virus, en este caso parte de la población abandona el estado de susceptible al recibir la vacuna y se incorpora directamente al compartimento de recuperados, sin haber pasado por la enfermedad, pero con la misma inmunidad que las personas que han pasado la enfermedad.
La curva de recuperados muestra un crecimiento más rápido y sostenido gracias al efecto de la vacunación, alcanzando antes un número elevado de personas inmunes. Esto acelera la inmunidad colectiva, lo que a su vez disminuye la probabilidad de nuevos contagios en fases posteriores.
Asimismo, la curva de expuestos, representa a aquellos que han estado en contacto con el virus pero aún no son contagiosos, muestra un comportamiento más controlado, evitando una acumulación excesiva de casos latentes como ocurría en la simulación sin vacunación.
Comparando ambos modelos, se concluye que la vacunación tiene un impacto clave en:
\begin{itemize}
    \item Disminuir el número total de personas infectadas.
    \item Reducir la velocidad de propagación.
    \item Limitar el número de personas susceptibles en el tiempo.
    \item Acelerar el final de la epidemia al aumentar rápidamente el número de recuperados.
\end{itemize}	
Estos resultados reflejan de forma cuantitativa cómo una campaña de vacunación masiva, como la que se implementó en España a partir de diciembre de 2020, puede modificar de manera muy significativa la evolución de una pandemia, confirmando la eficacia de esta estrategia como medida de control sanitario.



\section{Discusión}
\textbf{Modelo SI}. Los resultados muestran que cuando el valor de $\beta$ es elevado, el número de contagios crece rápidamente. Esto se debe a que cada encuentro entre una persona infectada y una susceptible tiene una alta probabilidad de resultar en un nuevo caso. Como consecuencia, la epidemia se desarrolla en un corto periodo de tiempo y afecta a un gran número de personas en fases tempranas del brote. En estos casos, la enfermedad puede propagarse de forma explosiva, reduciendo drásticamente el margen de actuación para implementar medidas de contención.
Por el contrario, cuando $\beta$ es bajo, la probabilidad de que un contacto provoque un contagio disminuye. Esto reduce la expansión del brote, permitiendo que el número de infectados crezca de forma más gradual. En este escenario, aunque la enfermedad puede seguir extendiéndose, lo hace a un ritmo más lento, lo que facilita la intervención temprana y el control del brote.
Estimar con precisión el valor de $\beta$ es fundamental para diseñar medidas de control adecuadas desde el inicio del brote.
Se puede concluir:
\begin{itemize}
    \item A mayor valor de $\beta$, mayor será la velocidad de propagación de la enfermedad.
    \item Una $\beta$ baja ralentiza el brote, permitiendo una intervención más efectiva.
\end{itemize}

\vspace{2em} 

\textbf{Modelo SI, SIDA/VIH}. Cabe destacar que el modelo SI utilizado en esta simulación representa una idealización de la dinámica de transmisión del VIH, y aunque resulta útil para comprender el comportamiento general de la enfermedad, no refleja con total precisión la complejidad de su propagación en la realidad.
El VIH es una enfermedad de transmisión sexual, lo que implica que no toda la población tiene el mismo nivel de riesgo de contagio. En la práctica, la probabilidad de transmisión varía considerablemente en función de factores como el comportamiento sexual, el acceso a educación sexual, el uso de métodos de protección (como preservativos), el número de parejas sexuales, el nivel de conciencia pública sobre la enfermedad y el acceso a servicios sanitarios.
Además, a diferencia del modelo SI que asume una transmisión homogénea en toda la población, en la vida real existen grupos poblacionales con mayor o menor exposición al virus, como pueden ser los trabajadores sexuales, personas que consumen drogas inyectables, o quienes tienen relaciones sexuales sin protección. Esto genera una heterogeneidad en la dinámica del contagio que no está reflejada en un modelo tan simplificado.

Por otro lado, la implementación de medidas de control y prevención (como campañas de concienciación, distribución de preservativos, pruebas de diagnóstico y tratamientos antirretrovirales) puede alterar significativamente la evolución del brote. Si bien el modelo SI predice una propagación continua e irreversible, en contextos reales estas medidas pueden frenar e incluso estabilizar el número de casos, impidiendo que toda la población se infecte.
Por tanto, aunque el modelo SI ofrece una buena aproximación teórica para enfermedades crónicas como el VIH, es fundamental interpretar los resultados con cautela y considerar las particularidades sociales, biológicas y conductuales que influyen en la propagación del virus en la vida real.
\vspace{2em}

\textbf{Modelo SIS}.
Las simulaciones han permitido observar cómo la evolución de la enfermedad está determinada principalmente por dos parámetros fundamentales:
\begin{itemize}
    \item Tasa de transmisión ($\beta$): representa la facilidad con la que se transmite la enfermedad. Un valor elevado implica una rápida propagación, especialmente en las fases iniciales del brote, cuando la mayoría de la población es susceptible. A medida que el número de infectados aumenta, los contagios se aceleran significativamente.
    \item Tasa de recuperación ($\gamma$): indica la velocidad con la que los individuos infectados se recuperan y vuelven a ser susceptibles. Cuanto mayor es, menor es el tiempo durante el cual una persona puede contagiar, lo que contribuye a frenar la propagación de la enfermedad.
\end{itemize}
La interacción entre estos dos parámetros se resume en el número básico de reproducción, $R_0$, que es clave para predecir el comportamiento del brote:
\begin{itemize}
    \item Si $R_0$ < 1, la enfermedad no logra propagarse de manera efectiva. Cada infectado contagia, de media, a menos de una persona, por lo que el número de infectados disminuye con el tiempo y la enfermedad desaparece.
    \item Si $R_0$ > 1, se produce un brote epidémico. La enfermedad persiste en el tiempo y se alcanza un equilibrio endémico, en el que las infecciones se compensan con las recuperaciones. La población mantiene de forma constante una proporción de individuos infectados y susceptibles.
\end{itemize}
Una de las observaciones más relevantes de las simulaciones es que la enfermedad no afecta a toda la población simultáneamente. Debido a la naturaleza del modelo, los individuos infectados se recuperan y vuelven al estado de susceptibles, creando un flujo continuo entre los dos estados. Este ciclo permite que la enfermedad nunca desaparezca por completo y se estabilice en un equilibrio dinámico.
Se ha comprobado que, al modificar los valores de beta y gamma, se altera la evolución del sistema. Por tanto, se puede concluir que:
\begin{itemize}
    \item Si la transmisión supera a la recuperación (beta > gamma), la enfermedad persiste en la población de manera crónica y endémica.
    \item Si la recuperación supera a la transmisión (gamma > beta), la enfermedad tiende a desaparecer con el tiempo.
\end{itemize}	
En definitiva, los resultados obtenidos reflejan que las estrategias para el control o erradicación de enfermedades que se ajustan al modelo SIS deben centrarse en: reducir la tasa de transmisión (mediante prevención, educación o medidas sanitarias) y aumentar la tasa de recuperación (mediante tratamiento médico eficaz y accesible). La correcta gestión de estos factores permitirá contener o eliminar enfermedades sin inmunidad permanente, garantizando un mejor estado de salud pública a largo plazo.

\vspace{2em}

\textbf{Modelo SIS, gonorrea}. Cabe destacar que el modelo SIS utilizado en esta simulación representa una idealización matemática del comportamiento de una enfermedad como la gonorrea, y aunque resulta útil para analizar su propagación general en la población, no refleja con total precisión la complejidad del contagio en contextos reales.
La gonorrea es una enfermedad de transmisión sexual, lo que implica que no toda la población está expuesta con el mismo nivel de riesgo. En la vida real, la probabilidad de infección depende de múltiples factores, como: el comportamiento sexual individual (número de parejas, prácticas sexuales), el uso de métodos de protección como preservativos, el acceso a educación sexual y servicios médicos, el conocimiento de la infección y la frecuencia de pruebas diagnósticas, y factores socioculturales y económicos.

El modelo SIS asume una transmisión homogénea, que todos los individuos tienen la misma probabilidad de infectarse o recuperarse. Sin embargo, en la realidad existe una marcada heterogeneidad poblacional: ciertos grupos presentan un mayor riesgo, como los adolescentes sexualmente activos, trabajadores sexuales o personas con acceso limitado a servicios de salud sexual. Esta heterogeneidad no es captada por el modelo SIS básico.

Además, la existencia de intervenciones sanitarias efectivas, como el diagnóstico precoz, el tratamiento con antibióticos y campañas de prevención, puede alterar significativamente la dinámica de la enfermedad. En el modelo SIS, se supone que las personas infectadas se recuperan y vuelven a ser susceptibles, sin que existan medidas externas que interfieran en esta dinámica. Sin embargo, en contextos reales, la detección temprana y el tratamiento adecuado pueden reducir la transmisión y, en algunos casos, controlar eficazmente la enfermedad en determinadas poblaciones.
Por tanto, aunque el modelo SIS es una herramienta valiosa para entender enfermedades que no confieren inmunidad duradera como la gonorrea, es fundamental interpretar los resultados con cautela. La simulación proporciona una aproximación teórica útil, pero no sustituye el análisis epidemiológico completo, que debe incluir la diversidad de comportamientos, desigualdades en salud, políticas públicas y características propias de cada sociedad.
Este enfoque mixto entre modelos matemáticos y realidad epidemiológica es esencial para diseñar estrategias de prevención más efectivas y políticas de salud pública ajustadas a la dinámica real de la infección.

\vspace{2em}

\textbf{Modelo SIR}. Las simulaciones han permitido observar cómo la evolución de la enfermedad depende principalmente de dos parámetros clave:
\begin{itemize}
    \item Tasa de transmisión ($\beta$): Representa la probabilidad de que un individuo susceptible se contagie al entrar en contacto con una persona infectada. Cuanto mayor es $\beta$, más rápido se propaga la enfermedad, especialmente en las fases iniciales del brote cuando la mayoría de la población es susceptible.
    \item Tasa de recuperación ($\gamma$): Indica la velocidad a la que los individuos infectados se recuperan y adquieren inmunidad permanente. Un valor alto de $\gamma$ implica que los infectados se recuperan rápidamente, reduciendo su tiempo de contagio y, por tanto, disminuyendo la capacidad de expansión del brote.
\end{itemize}
	
La interacción entre estos dos parámetros se resume en el número básico de reproducción, $R_0$, que es fundamental para entender el comportamiento epidémico del sistema:
\begin{itemize}
    \item Si $R_0$ < 1, la enfermedad tiende a desaparecer, ya que cada infectado contagia a menos de una persona, de media.
    \item Si $R_0$ > 1, la enfermedad se propaga inicialmente, pero no se mantiene de forma indefinida como en el modelo SIS. Se observa un pico epidémico, los infectados aumentan rápidamente hasta un punto máximo, tras el cual disminuyen, ya que la cantidad de individuos susceptibles va reduciéndose y la mayoría de la población termina inmunizada.
\end{itemize}
	
Una observación importante es que la enfermedad no permanece en equilibrio permanente en la población. A diferencia del modelo SIS, donde los individuos recuperados regresan al estado de susceptibles, aquí los recuperados se vuelven inmunes. Esto genera una dinámica de agotamiento de susceptibles, lo que impide que la enfermedad siga propagándose de forma indefinida.
Las simulaciones muestran que la propagación se acelera inicialmente, pero después se frena bruscamente cuando el número de susceptibles cae por debajo del umbral necesario para mantener nuevos contagios. Así, la enfermedad desaparece de forma natural sin necesidad de que todos se infecten, gracias a la inmunidad colectiva alcanzada por una fracción significativa de la población.
Por tanto, se puede concluir que:
\begin{itemize}
    \item Si $\beta$ es alta y $\gamma$ es baja, se alcanzará un pico epidémico más rápido y alto.
    \item Si $\gamma$ es alta o $\beta$ es baja, la propagación será más lenta y es posible que el brote se controle sin llegar a afectar a una gran parte de la población.
    \item En todos los casos donde $R_0$ > 1, el brote se extenderá inicialmente, pero terminará decayendo cuando no queden suficientes susceptibles.
\end{itemize}
En definitiva, los resultados del modelo SIR reflejan la importancia de reducir la tasa de transmisión (mediante vacunación, medidas higiénicas o distanciamiento) y aumentar la tasa de recuperación (con tratamientos eficaces), ya que estos factores determinan si la enfermedad provocará una epidemia masiva o se controlará antes de alcanzar un umbral crítico.

\vspace{2em}

\textbf{Modelo SIR, sarampión}. Es importante señalar que el modelo SIR utilizado en esta simulación representa una idealización matemática del comportamiento de una enfermedad como el sarampión, y aunque resulta útil para comprender los mecanismos generales de su propagación, no reproduce con exactitud la complejidad del contagio en escenarios reales.
El modelo asume una población homogénea, donde todos los individuos tienen la misma probabilidad de infectarse y recuperarse, y no contempla factores estructurales, sociales o individuales que puedan alterar significativamente la dinámica real de transmisión. En la simulación se parte del supuesto de que la mayoría de la población es inicialmente susceptible, lo que corresponde al escenario del sarampión antes de la introducción de la vacuna en 1963, cuando efectivamente se registraban cientos de miles de casos anuales en Estados Unidos. Sin embargo, incluso en ese contexto, existían variaciones en la exposición al virus, como la distribución geográfica, densidad de población, edad de los individuos, o el acceso a atención médica, que no son recogidas por el modelo.
En la realidad, la propagación del sarampión está influida por diversos factores adicionales.
No contempla intervenciones externas, como cuarentenas, campañas de concienciación o tratamientos sintomáticos que puedan modificar la duración de la enfermedad o disminuir su propagación. Tampoco considera la posibilidad de inmunidad parcial o pérdida de inmunidad, aunque en el caso del sarampión, la inmunidad adquirida tras la infección es generalmente permanente, lo que sí justifica el uso del modelo SIR para este tipo de enfermedad.
Además, el modelo no tiene en cuenta las desigualdades socioeconómicas, que pueden influir en la velocidad de propagación, el acceso a atención sanitaria o la calidad del diagnóstico, factores todos ellos que alteran de forma relevante la dinámica observada en la práctica.

Por tanto, aunque los resultados obtenidos con la simulación reflejan con claridad el patrón teórico de una epidemia de sarampión en un entorno sin inmunización previa, es necesario interpretar dichos resultados con cautela. El modelo proporciona una aproximación simplificada que resulta muy útil para entender la evolución general de la enfermedad y el impacto del número básico de reproducción, pero no sustituye un análisis epidemiológico completo, que debe considerar la heterogeneidad poblacional, la respuesta del sistema de salud y los factores sociales y demográficos específicos.
En definitiva, la combinación entre modelos matemáticos y datos epidemiológicos reales resulta fundamental para diseñar estrategias de prevención y control más efectivas. Aunque el modelo SIR ofrece una base sólida para la comprensión teórica de la propagación del sarampión, su aplicación requiere ser complementada con información contextual que refleje la diversidad de comportamientos, políticas de salud y condiciones sociales existentes en la población.

\vspace{2em}

\textbf{Modelo SIRV, sarampión} Aunque en el modelo utilizado se ha considerado una cobertura de vacunación del 92,7\%, es importante contextualizar este dato desde un punto de vista histórico y epidemiológico. La vacuna contra el sarampión fue introducida en Estados Unidos en 1963, pero durante los primeros años la tasa de vacunación fue considerablemente más baja. Con el paso del tiempo, y gracias a campañas de inmunización masiva, programas escolares y mejoras en el acceso a servicios de salud pública, la cobertura fue aumentando hasta superar el 90\% en las últimas décadas.
Este esfuerzo permitió que, en el año 2000, Estados Unidos declarara eliminada la transmisión endémica del sarampión, el virus ya no circula de manera continua dentro del país. En la actualidad, los casos que se detectan son importados: se producen por personas infectadas en otros países donde la enfermedad sigue siendo común. Aun así, si existen personas no vacunadas o con inmunización incompleta, pueden aparecer brotes localizados tras la introducción del virus desde el exterior.

Por lo tanto, aunque el modelo SIRV resulta muy útil para comprender la dinámica general del sarampión en presencia de vacunación, hay que tener en cuenta que simplifica la realidad. El modelo asume una alta y constante cobertura vacunal desde el inicio, lo cual no refleja completamente la evolución histórica ni la heterogeneidad del comportamiento poblacional, como diferencias en acceso, cobertura regional o rechazo vacunal.
Aun así, esta simplificación es válida desde el punto de vista teórico, ya que permite visualizar con claridad cómo la vacunación modifica el comportamiento epidémico, reduciendo drásticamente la población susceptible y, por tanto, la propagación de la enfermedad. De este modo, el modelo es una herramienta fundamental para ilustrar el impacto positivo que tiene la vacunación masiva como medida de salud pública.

\vspace{2em}


\vspace{2em}
\textbf{Modelo SIR con regulador PID}
En los casos analizados, el uso del controlador PID mejora la respuesta del sistema frente a la propagación de la epidemia. Mientras que en las simulaciones sin control se observan infecciones prolongadas o estabilizadas en niveles indeseables, el PID permite limitar los picos de contagio, mantener el número de infectados cerca del valor objetivo y, en la mayoría de los casos, conducir el sistema hacia un estado libre de infección.

La principal ventaja del controlador es su capacidad para adaptarse dinámicamente a los cambios del sistema. Al actuar sobre el parámetro beta, el PID simula la implementación de medidas sanitarias o sociales que reducen la transmisión, como confinamientos, restricciones de movilidad o campañas de concienciación. En este sentido, el modelo controlado se convierte en una herramienta útil para estudiar y diseñar estrategias de intervención efectivas durante una epidemia.

\textbf{Modelo SEIR}. Tras realizar diversas simulaciones con el modelo SEIR en Simulink, se ha podido observar cómo la incorporación de un periodo de incubación modifica significativamente la dinámica de propagación de una enfermedad infecciosa respecto a modelos más simples. A diferencia del modelo SIR, donde el contagio y la capacidad de transmisión coinciden temporalmente, el modelo SEIR introduce un retraso natural entre la infección y la aparición de nuevos contagios, lo que genera efectos evidentes en la forma y el desarrollo de las curvas epidémicas.
Uno de los resultados más destacables es el desplazamiento temporal del pico de infecciones. Al existir una fase de incubación, el crecimiento de los casos activos se ralentiza al principio del brote. Esto no implica una menor propagación, sino una distribución más extendida en el tiempo, lo que puede suponer una ventaja desde el punto de vista del sistema sanitario, al evitar colapsos por picos excesivamente altos en corto plazo.
La evolución del brote está condicionada por tres parámetros clave: la tasa de transmisión ($\beta$), la tasa de incubación ($\sigma$) y la tasa de recuperación ($\gamma$). El balance entre estas tasas determina la velocidad de propagación, el momento de los picos y la duración del brote.
Se ha comprobado que:
\begin{itemize}
    \item Un valor alto de $\beta$, combinado con un $\sigma$ elevado, conduce a un brote rápido y agresivo, ya que las personas expuestas se convierten rápidamente en infecciosas, alimentando el ciclo de contagio sin apenas retraso.
    \item Cuando $\sigma$ es bajo, es decir, cuando el periodo de incubación es más largo, el crecimiento inicial de los casos se frena. Esto genera una curva más aplanada, aunque el número total de infectados acumulados puede ser similar. El retraso en los contagios retrasa también el pico de infecciones y lo suaviza, facilitando su gestión.
    \item Un $\gamma$ alto (recuperación rápida) reduce tanto la duración como la magnitud del brote. En cambio, una recuperación lenta implica que los infectados permanecen más tiempo en la población, aumentando las posibilidades de contagio y prolongando la epidemia.
\end{itemize}
En cuanto al número básico de reproducción $R_0$, sigue siendo un parámetro esencial para predecir el comportamiento general del sistema. Cuando $R_0$ > 1, la enfermedad tiende a propagarse inicialmente, pero al haber inmunidad permanente, el brote no se mantiene indefinidamente, alcanza un pico y desciendo conforme la población susceptible disminuye. Mientras que si R0<1, cada persona contagia a menos de una persona susceptible de media. La infección no puede sostenerse en la población y tiende a desaparecer con el tiempo, porque el número de nuevos casos es insuficiente para mantener o aumentar la transición.
En conclusión, las simulaciones del modelo SEIR permiten observar que no solo el valor de $R_0$ condiciona la magnitud del brote, sino también la estructura temporal del contagio, modulada por el periodo de incubación. Esto hace que el modelo SEIR sea más preciso a la hora de representar enfermedades con latencia, como es el caso de muchas infecciones víricas reales. Además, proporciona información clave para anticipar el impacto de medidas de contención tempranas y la importancia de reducir el contacto incluso antes de la aparición de síntomas, ya que el retraso entre infección y capacidad de transmisión introduce un riesgo oculto de propagación.

\vspace{2em}
\textbf{Modelo SEIR, COVID-19} Es importante señalar que el modelo SEIR utilizado en esta simulación representa una idealización matemática del comportamiento de enfermedades infecciosas con periodo de incubación. Este modelo es útil para entender los mecanismos generales de propagación de la enfermedad y permite analizar el impacto de distintos parámetros epidemiológicos, pero no reproduce con total fidelidad la complejidad del contagio en escenarios reales.
Asume una población homogénea y bien mezclada, en la que todos los individuos tienen la misma probabilidad de contacto, infección, incubación y recuperación. En la práctica, sin embargo, existen múltiples factores estructurales, sociales, demográficos y biológicos que influyen significativamente en la dinámica real de transmisión del virus y que el modelo no contempla.

En el caso concreto del COVID-19, la propagación del virus ha estado condicionada por variables que no están incluidas en el modelo, como la heterogeneidad etaria, ya que la gravedad y el riesgo de contagio varían entre grupos de edad. La movilidad geográfica, tanto nacional como internacional, que facilita la expansión del brote. Las condiciones de vida y trabajo, especialmente en espacios cerrados o con alta densidad de población. Las intervenciones sanitarias. La aparición de nuevas variantes. La vacunación, que comenzó a aplicarse en fases avanzadas de la pandemia y modificó de forma importante la evolución epidemiológica.
Además, el modelo tampoco considera la pérdida de inmunidad ni las reinfecciones, fenómenos que han demostrado ser relevantes en la evolución a largo plazo del SARS-CoV-2. Tampoco incluye aspectos como la asimetría en los tiempos de incubación o los casos asintomáticos, que dificultan la detección y control del brote.
Pese a estas limitaciones, el modelo SEIR es adecuado para enfermedades con un periodo de incubación, como es el caso del COVID-19, en el que las personas pasan por una fase latente antes de volverse infecciosas. Esta característica se refleja correctamente mediante el compartimento expuestos, lo que lo diferencia del modelo SIR y le otorga mayor precisión en contextos donde el desfase entre infección y contagio tiene un impacto epidemiológico relevante.

Aunque los resultados obtenidos mediante la simulación permiten visualizar de forma clara la evolución teórica de un brote de COVID-19 en ausencia de inmunidad previa o intervenciones externas, es fundamental interpretar estos resultados con cautela. El modelo proporciona una herramienta potente para el análisis cualitativo de la dinámica de la enfermedad, pero no sustituye el análisis epidemiológico completo, que debe considerar la heterogeneidad poblacional, los factores sociales y sanitarios, así como las políticas públicas implementadas.
La combinación entre modelos matemáticos como el SEIR y los datos reales recopilados durante la pandemia resulta esencial para diseñar estrategias de prevención, mitigación y control efectivas. De este modo, los modelos no solo permiten anticipar escenarios futuros, sino también evaluar el impacto potencial de distintas medidas sanitarias en el curso de la epidemia.

\vspace{2em}

\textbf{Modelo SEIRV, COVID-19}. Aunque en el modelo SEIRV utilizado se ha considerado una cobertura de vacunación elevada y una inmunidad permanente tras la vacunación o la recuperación, es importante destacar que esta suposición representa una simplificación teórica que no refleja con total precisión la complejidad real del comportamiento inmunológico frente al COVID-19.

A diferencia de enfermedades como el sarampión, cuya vacuna proporciona una inmunidad de por vida, en el caso del COVID-19 se ha observado que la inmunidad inducida por las vacunas puede disminuir con el tiempo, especialmente frente a nuevas variantes del virus. Diversos estudios han demostrado que la eficacia vacunal puede reducirse significativamente varios meses después de la administración de la pauta completa, lo que ha hecho necesario recurrir a dosis de refuerzo para mantener una protección adecuada.
Además, no todas las personas desarrollan una respuesta inmunitaria completa tras la vacunación. Factores como la edad avanzada, enfermedades crónicas o inmunodepresión pueden limitar la efectividad de las vacunas, haciendo que ciertos individuos sigan siendo vulnerables al contagio. Incluso con una cobertura de vacunación muy alta, existe un margen de población que sigue siendo susceptible al virus, aunque esté vacunada.
Otra limitación importante del modelo es que no se contempla la aparición de nuevas variantes con mayor transmisibilidad o capacidad de evasión inmunológica. Estas variantes pueden alterar significativamente la dinámica de propagación, provocando nuevos brotes incluso en poblaciones mayoritariamente inmunizadas.
Tampoco se incorpora en el modelo la mortalidad, que representa un factor crítico en la gestión sanitaria y la percepción social de la enfermedad. A lo largo de la pandemia, se ha registrado una elevada tasa de mortalidad, especialmente en personas mayores, que ha tenido un fuerte impacto en los sistemas de salud y en la necesidad de aplicar medidas. En el modelo SEIRV, todos los individuos infectados se asumen como eventualmente recuperados o vacunados, omitiendo el hecho de que una fracción de la población infectada puede fallecer. Esta omisión puede llevar a una subestimación del impacto real de la enfermedad, y limita el alcance del modelo para evaluar con precisión los efectos sanitarios y sociales de la epidemia.

Por tanto, aunque el modelo SEIRV permite visualizar con claridad el impacto teórico de una campaña de vacunación masiva sobre la propagación del COVID-19, sus resultados deben interpretarse con cautela. Asume una inmunidad inmediata, total y permanente en toda la población vacunada, y omite tanto la posibilidad de reinfección como la mortalidad asociada.
A pesar de estas limitaciones, el modelo sigue siendo útil como herramienta didáctica para comprender los principios generales de control epidémico mediante vacunación. Sin embargo, para un análisis más preciso y aplicable a la toma de decisiones sanitarias, es necesario incorporar parámetros más realistas como la duración de la inmunidad, la tasa de fallo vacunal, la aparición de nuevas variantes, y especialmente la mortalidad, lo cual escapa al alcance del modelo básico pero abre la puerta a modelos más avanzados y específicos.

\capitulo{6}{Conclusiones}
Se ha llevado a cabo un estudio completo y detallado sobre el modelado determinista de epidemias, centrándose en los modelos clásicos SI, SIS, SIR y SEIR, que representan diferentes dinámicas de transmisión y recuperación de enfermedades infecciosas en una población. Se ha realizado una ampliación significativa en los modelos SIR y SEIR mediante la incorporación de la vacunación, bajo la hipótesis de que la inmunidad adquirida a través de la vacunación es equivalente a la obtenida tras la recuperación natural. Esta simplificación, aunque no considera variaciones en la eficacia de la vacuna o en la duración de la inmunidad, permite una aproximación adecuada para analizar el impacto de las estrategias de vacunación en la evolución de la epidemia.

Un aporte clave de este trabajo ha sido la implementación de un controlador PID en el modelo SIR, que simula medidas de intervención sanitaria como cuarentenas o restricciones sociales. La integración de este regulador posibilita un análisis más dinámico y realista, mostrando cómo la aplicación de políticas de control puede influir en la reducción de la transmisión y en el manejo de brotes epidémicos.

Para validar la utilidad y aplicabilidad de los modelos, se han empleado tanto datos aleatorios como datos reales. Los datos simulados han permitido examinar la respuesta teórica de los modelos bajo diferentes condiciones y parámetros, mientras que los datos reales han facilitado la comparación con situaciones epidémicas auténticas, evidenciando las fortalezas y limitaciones de cada modelo. Esto proporciona una visión práctica y fundamentada que puede ser de gran utilidad para profesionales en salud pública y modeladores matemáticos.

Además, se ha desarrollado una aplicación interactiva con una interfaz gráfica que hace posible visualizar de manera intuitiva y accesible el comportamiento de cada modelo. Esta herramienta está diseñada para que cualquier usuario, sin necesidad de conocimientos técnicos, pueda interactuar con los modelos y entender cómo distintas variables afectan la evolución de una epidemia. Esta accesibilidad contribuye a la difusión del conocimiento científico y puede apoyar la educación en temas de salud pública y prevención.


\section{Aspectos relevantes}
\begin{itemize}
    \item Estudio exhaustivo de modelos epidemiológicos deterministas, el análisis abarca cuatro modelos fundamentales en la epidemiología matemática: SI (susceptible-infectado), SIS (susceptible-infectado-susceptible), SIR (susceptible-infectado-recuperado) y SEIR (susceptible-expuesto-infectado-recuperado). Cada modelo representa distintas características y escenarios epidemiológicos, lo que permite comprender mejor cómo se propagan diferentes tipos de enfermedades infecciosas y las posibles transiciones entre estados.
    \item Incorporación de la vacunación en modelos SIR y SEIR. La introducción de la vacunación en estos modelos aporta un elemento crucial para el análisis de control epidémico, reflejando el efecto protector de las vacunas al desplazar individuos directamente al estado de inmunidad. Esto facilita la evaluación del impacto potencial de campañas de vacunación masiva, ayudando a predecir cómo puede cambiar la dinámica de contagios y la eventual reducción de la población susceptible.
    \item Diseño y aplicación de un regulador PID en el modelo SIR. El desarrollo de un controlador PID aplicado al modelo SIR representa un avance importante en la simulación de políticas públicas de control. Este controlador permite modelar intervenciones como cuarentenas, restricciones de movilidad y otras medidas no farmacológicas, evaluando su eficacia y optimizando su aplicación para mitigar el avance de la enfermedad.
    \item Uso combinado de datos aleatorios y datos reales para validación. El empleo de datos generados aleatoriamente ha sido esencial para probar la estabilidad y comportamiento de los modelos bajo diferentes escenarios hipotéticos. Por otro lado, la aplicación de datos reales permite una validación práctica, demostrando la capacidad predictiva de los modelos y su utilidad en la toma de decisiones en situaciones reales.
    \item Desarrollo de una aplicación gráfica accesible. La creación de una herramienta visual interactiva, amplía el alcance del trabajo al hacerlo accesible para un público más amplio, incluyendo profesionales, estudiantes y personas interesadas en el tema. La interfaz gráfica facilita la manipulación de parámetros y la observación de resultados en tiempo real, fomentando la comprensión y el aprendizaje de la dinámica epidémica.
    \item Relevancia para la salud pública y la educación. Más allá del aspecto técnico, el trabajo contribuye a la sensibilización y educación sobre la importancia del modelado matemático en la gestión de epidemias. Permite a usuarios no especializados comprender cómo diferentes factores afectan la propagación de enfermedades y la efectividad de intervenciones, apoyando así la difusión de información científica y la toma de decisiones informadas en contextos de salud pública.
    \item Posibilidad de futuras ampliaciones. El enfoque y las herramientas desarrolladas abren la puerta a futuras investigaciones, tales como la incorporación de modelos estocásticos, el análisis de inmunidad temporal, variantes virales, o la inclusión de factores sociales y económicos en la modelización, lo que puede enriquecer y actualizar las predicciones y estrategias de control epidémico.
\end{itemize}










\capitulo{7}{Líneas de trabajo futuras}


\bibliographystyle{apalike}
\bibliography{bibliografia}

\end{document}
