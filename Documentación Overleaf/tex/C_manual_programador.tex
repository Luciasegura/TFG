\apendice{Manual del investigador}

\section{Estructura de directorios}
A continuación se describe la estructura del proyecto, así como el contenido y propósito de cada carpeta y archivo relevante, todo está en el repositorio de GitHub. El proyecto está organizado en una estructura de carpetas que separa claramente los distintos componentes como se describe a continuación.

\begin{itemize}
    \item \textbf{Carpeta \texttt{aplicación}} \\

Contiene los archivos necesarios para la ejecución y distribución de la aplicación desarrollada en \textit{App Designer} de MATLAB. A continuación, se describen los elementos incluidos:

\begin{itemize}
    \item \texttt{aplicacion\_modelos.mlapp}: archivo principal de la aplicación, editable y ejecutable desde MATLAB con App Designer.
    
    \item \texttt{aplicacion\_modelos.exe}: versión compilada de la aplicación para ejecutarse de forma autónoma sin necesidad de abrir MATLAB.
    
    \item \texttt{MyAppInstaller\_web.exe}: instalador web proporcionado por MATLAB que permite instalar las dependencias necesarias para ejecutar la aplicación si el usuario no dispone de una instalación completa de MATLAB.
    
    \item Carpeta \texttt{aplicacion\_modelos}: contiene los archivos auxiliares generados tras la compilación de la aplicación. Son necesarios para su correcto funcionamiento sin entorno MATLAB.
    
    \item \texttt{README.txt}: archivo de texto que actúa como guía de usuario. Describe los componentes anteriores y cómo instalar y ejecutar la aplicación.
\end{itemize}


    \item \textbf{Carpeta bibliografía} \\
    Documentos y artículos científicos en .pdf utilizados para entender y fundamentar los modelos epidemiológicos.

    \item \textbf{Carpeta documentación overleaf} \\
    Carpeta que contiene los archivos fuente del TFG escritos en LaTeX:
    \begin{itemize}
        \item \texttt{Carpeta img}: imágenes utilizadas en la memoria y anexos.
        \item \texttt{Carpeta tex}: secciones y capítulos del TFG, tanto memoria como anexos.
        \item \texttt{anexos.tex, memoria.tex}: documentos principales del trabajo.
        \item \texttt{bibliografia.bib, bibliografiaAnexos.bib}: bases de datos bibliográficas en formato BibTeX.
    \end{itemize}

    \item \textbf{Carpeta informes entrega} \\
    Archivos PDF del trabajo final entregado:
    \begin{itemize}
        \item \texttt{anexos\_Lucía\_Segura.pdf}: anexos en formato .pdf.
        \item \texttt{memoria\_Lucía\_Segura.pdf}: memoria en formato .pdf.
    \end{itemize}

    \item \textbf{Carpeta MATLAB} \\
    Archivos y resultados de las simulaciones en entorno MATLAB:
    \begin{itemize}
        \item \texttt{/resultados/}: gráficas generadas desde scripts.
        \item \texttt{modelo\_SIR\_PID.m}: código de simulación del modelo SIR con control PID.
        \item Imágenes del controlador y parámetros PID.
    \end{itemize}

    \item \textbf{Carpeta Simulink} \\
    Contiene los modelos diseñados en Simulink y sus resultados visuales:
    \begin{itemize}
        \item \texttt{Carpeta modelos Simulink}: archivos \texttt{.slx} de los modelos SI, SIS, SIR, SEIR, SIRV y SEIRV.
        \item \texttt{carpeta Otras simulaciones}: contiene capturas de pantalla y gráficos generados al modificar distintos parámetros en los modelos matemáticos. Estas simulaciones adicionales permiten realizar un análisis comparativo entre diferentes escenarios y evaluar la sensibilidad del modelo ante variaciones en sus condiciones iniciales o parámetros clave.
        \item \texttt{Carpeta resultados}: imágenes de resultados (\texttt{.jpg}).
        \item Imágenes de los diagramas (\texttt{.png}).
    \end{itemize}

    \item README.md \\
    Archivo de introducción al proyecto, útil para usuarios que lo descarguen por primera vez.
\end{itemize}

Esta organización permite acceder rápidamente a cada componente del proyecto según su función: código, documentación, simulaciones o entregas finales.


\section{Compilación, instalación y ejecución del proyecto}

El proceso completo de instalación del entorno \textit{MATLAB} y la ejecución de la aplicación se encuentra detallado en el Apéndice B, concretamente en el apartado B.2. A continuación, se resumen las consideraciones técnicas más relevantes desde el punto de vista del desarrollo y la distribución del software.

El proyecto ha sido desarrollado en \textbf{MATLAB R2019b}, por lo que se recomienda utilizar esta misma versión o una superior compatible con los archivos generados.

\subsection*{Requisitos del entorno de desarrollo}
Para ejecutar los modelos y la aplicación, se requiere disponer de los siguientes componentes instalados:

\begin{itemize}
    \item \textbf{MATLAB} (versión R2019b o superior)
    \item \textbf{Simulink}
    \item \textbf{Control System Toolbox}
    \item \textbf{App Designer}
\end{itemize}

Los modelos matemáticos implementados en Simulink se encuentran en la carpeta \texttt{/Simulink/modelos\_Simulink/} con extensión \texttt{.slx}, y pueden abrirse directamente desde el entorno MATLAB. Por su parte, los scripts, incluyendo el controlador PID desarrollado sobre el modelo SIR, están ubicados en la carpeta \texttt{/MATLAB/} con extensión \texttt{.m}.

La aplicación desarrollada con App Designer se encuentra en la carpeta \texttt{/aplicación/}, dentro del archivo \texttt{aplicacion\_modelos.mlapp}.

\subsection*{Ejecución de la aplicación}

La carpeta \texttt{aplicación} contiene todos los archivos necesarios para ejecutar la \textbf{Aplicación de Modelos Epidemiológicos}. Existen dos modos de ejecución, en función de si el usuario dispone de MATLAB instalado o no:

\begin{enumerate}
    \item \textbf{Con MATLAB instalado}: utilizando directamente el archivo \texttt{.mlapp}.
    \item \textbf{Sin MATLAB instalado}: mediante el ejecutable compilado \texttt{.exe} y el instalador de runtimes de MATLAB.
\end{enumerate}

\subsection*{Archivos incluidos en la carpeta \texttt{aplicación}}

\begin{itemize}
    \item \texttt{aplicacion\_modelos.mlapp}: interfaz gráfica de usuario desarrollada con App Designer. Ejecutable desde MATLAB.
    
    \item \texttt{aplicacion\_modelos.exe}: versión compilada de la aplicación para sistemas Windows (64 bits). Ejecutable independiente que requiere los runtimes de MATLAB.

    \item \texttt{MyAppInstaller\_web.exe}: instalador automático de los \textit{MATLAB Runtime} (MCR), necesarios para ejecutar el archivo \texttt{.exe} si no se dispone de MATLAB instalado.

    \item Carpeta \texttt{aplicacion\_modelos/}: contiene los archivos auxiliares generados durante la compilación de la aplicación.

    \item \texttt{README.txt}: archivo de texto con instrucciones detalladas sobre instalación, ejecución y requisitos del sistema.
\end{itemize}

\subsection*{Requisitos del sistema}

\begin{itemize}
    \item \textbf{Con MATLAB instalado:}
    \begin{itemize}
        \item MATLAB R2019b (o superior)
        \item App Designer
        \item No requiere instalar runtimes adicionales
    \end{itemize}

    \item \textbf{Sin MATLAB instalado:}
    \begin{itemize}
        \item Sistema operativo Windows 64 bits
        \item Espacio libre en disco: mínimo 500 MB
        \item Instalación de runtimes mediante \texttt{MyAppInstaller\_web.exe}
    \end{itemize}
\end{itemize}

\subsection*{Instrucciones de instalación y ejecución}

\begin{itemize}
    \item \textbf{Opción 1: Ejecución con MATLAB}
    \begin{enumerate}
        \raggedright
        \item Abrir MATLAB.
        \item Acceder al directorio donde se encuentra el archivo \texttt{aplicacion\_modelos.mlapp}.
        \item En la línea de comandos de MATLAB, ejecutar: \texttt{\textgreater{}\textgreater{} open('aplicacion\_modelos.mlapp')}.
        \item Pulsar en el botón “Run” dentro de App Designer.
        \item La aplicación se iniciará y permitirá interactuar con los modelos epidemiológicos.
    \end{enumerate}

    \item \textbf{Opción 2: Ejecución sin MATLAB}
    \begin{enumerate}
        \item Instalar los runtimes de MATLAB:
        \begin{itemize}
            \item Ejecutar \texttt{MyAppInstaller\_web.exe}.
            \item Seguir los pasos del asistente de instalación.
        \end{itemize}
        \item Ejecutar la aplicación compilada:
        \begin{itemize}
            \item Hacer doble clic en \texttt{aplicacion\_modelos.exe}.
            \item La aplicación se iniciará automáticamente.
        \end{itemize}
    \end{enumerate}
\end{itemize}

\subsection*{Notas adicionales}

\begin{itemize}
    \item Si ya se dispone de MATLAB instalado, no es necesario ejecutar \texttt{MyAppInstaller\_web.exe} ni utilizar la versión compilada.
    \item La instalación de los runtimes puede tardar varios minutos dependiendo de la velocidad de conexión y del sistema.
    \item La desinstalación de los MATLAB Runtime puede realizarse desde el Panel de control de Windows → “Programas y características” → “MATLAB Runtime”.
\end{itemize}



\section{Pruebas del sistema}
Durante el desarrollo del proyecto se llevaron a cabo diferentes pruebas para asegurar el correcto funcionamiento de los modelos implementados y de la interfaz gráfica desarrollada:

\begin{itemize}
    \item Se validó que los modelos epidemiológicos (SI, SIS, SIR, SEIR y SIR con control PID) producen resultados coherentes para diferentes combinaciones de parámetros.

    \item En los modelos SIR y SEIR se añadió un componente de \textbf{vacunación}. Se realizaron pruebas con tasas de vacunación constantes, observando cómo influye en la evolución de los individuos. Como resultado, se observaron curvas más planas o reducción del pico epidémico, en línea con lo esperado teóricamente.

    \item Se ejecutaron simulaciones con \textbf{datos reales de enfermedades infecciosas} contrastando los resultados generados por el modelo con los datos empíricos, para verificar la validez de la estructura matemática.

    \item Se realizaron pruebas específicas con el modelo \textbf{SIR regulado mediante un controlador PID}, comprobando que el sistema responde adecuadamente a variaciones en el número de infectados y que el regulador permite mitigar oscilaciones o picos no deseados.

    \item En cuanto a la aplicación gráfica (.mlapp) se verificó que cada modelo se carga correctamente, que los valores por defecto permiten una simulación directa y que la visualización gráfica es coherente con la salida del modelo.

    
\end{itemize}

Estas pruebas confirman la robustez funcional del sistema, tanto desde el punto de vista técnico como de su aplicabilidad para visualizar dinámicas epidemiológicas complejas.

\section{Instrucciones para la modificación o mejora del proyecto}

El proyecto ha sido desarrollado de forma modular, por lo que resulta relativamente sencillo introducir mejoras. A continuación, se describen algunas recomendaciones para su modificación o ampliación:

\begin{itemize}
    \item \textbf{Ampliación de modelos:} Se pueden implementar nuevos modelos epidemiológicos (por ejemplo, SEIRS, modelos estocásticos o con movilidad) reutilizando la estructura existente en la aplicación.

    \item \textbf{Incorporación de nuevas funcionalidades:} Sería interesante añadir funciones como el cálculo automático del número básico de reproducción \( R_0 \), análisis de sensibilidad, o exportación de resultados en CSV o PDF.

    \item \textbf{Mejora de la interfaz:} La interfaz puede ampliarse con pestañas o menús desplegables para facilitar la navegación. También se podría implementar la carga de datos desde archivos externos.

    \item \textbf{Uso de datos reales actualizados:} El proyecto puede conectarse con fuentes de datos abiertas (como la OMS o Our World in Data) para actualizar automáticamente los valores utilizados en las simulaciones.


\end{itemize}

Estas sugerencias pueden servir de base para futuras mejoras, tanto a nivel técnico como funcional.




