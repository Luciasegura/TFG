\capitulo{3}{Conceptos teóricos}
\setcounter{secnumdepth}{3}
\section{Epidemia, pandemia y epidemiología}
En este trabajo, aunque el enfoque principal es el modelado de epidemias, se utilizarán los términos \textit{epidemia} y \textit{pandemia} de manera indistinta, dado que los modelos deterministas que se aplican no distinguen explícitamente entre ambos conceptos. Estos modelos se limitan a simular la dinámica de transmisión de una enfermedad en una población, sin considerar directamente la escala geográfica o el alcance global del brote.
No obstante, es importante señalar que, si bien los modelos empleados pueden aplicarse tanto a epidemias como a pandemias, los términos no deben considerarse sinónimos, ya que su diferenciación radica en el contexto y la magnitud del fenómeno. Además, se analizarán situaciones reales que incluyen casos de ambos tipos, aprovechando la versatilidad de los modelos utilizados.

A continuación se definen ambos términos.

\subsection{Epidemia}
Una epidemia es la aparición de una enfermedad en una comunidad, zona o país durante un periodo de tiempo determinado, que afecta de forma simultánea o sucesiva, pero constante, a un número elevado de personas. Las causas pueden ser diversas e incluyen agentes infecciosos como virus, bacterias, parásitos u hongos, así como factores ambientales y sociales que favorecen la transmisión de la enfermedad. Se considera epidemia cuando cualquier enfermedad infecciosa se descontrola temporalmente y afecta a una proporción significativa de la población.

Cabe destacar que no todas las epidemias son provocadas por enfermedades contagiosas. Aunque muchas se deben a infecciones que se transmiten entre personas, también pueden originarse por factores como el comportamiento humano, el entorno, vectores (como mosquitos) o enfermedades zoonóticas transmitidas por animales. Incluso enfermedades no transmisibles, como la obesidad o la diabetes, pueden alcanzar niveles epidémicos debido a cambios en los estilos de vida.


\subsection{Pandemia}
Por su parte, una pandemia es una epidemia que se ha propagado a nivel mundial, afectando a un gran número de personas en varios países y continentes. Se caracteriza por la transmisión sostenida de persona a persona y genera un impacto significativo en la salud pública global, así como en la economía y la estructura social. El término “pandemia” implica que la enfermedad ha superado fronteras geográficas y ha afectado a diferentes poblaciones.

\subsection{Grupos vulnerables ante las epidemias}
Es importante reconocer que las epidemias afectan de manera desigual a ciertos grupos poblacionales debido a factores sociales, económicos y biológicos. Entre los más vulnerables se encuentran:
\begin{itemize}
  \item Personas mayores de 60 años, debido al mayor riesgo de complicaciones.
  \item Personas con inmunodeficiencia, ya sea adquirida o congénita.
  \item Mujeres embarazadas, por los cambios fisiológicos que afectan al sistema inmunológico.
   \item Personas con obesidad, ya que se ha identificado como factor de riesgo en enfermedades como el COVID-19.
   \item Personas con enfermedades crónicas, como diabetes, enfermedades cardiovasculares, renales, hepáticas o pulmonares.
   \item Grupos socioeconómicos desfavorecidos, que presentan limitaciones en el acceso a servicios de salud y viven en condiciones precarias. 
\end{itemize}

\subsection{Gestión, control y prevención de epidemias}
La gestión de una epidemia requiere de diversas etapas y estrategias. La detección temprana y la vigilancia epidemiológica son fundamentales para identificar brotes y evitar su propagación. Las evaluaciones de riesgo y vulnerabilidad permiten determinar las áreas y poblaciones más afectadas. La preparación incluye la planificación de recursos, la formación del personal y la implementación de sistemas de alerta temprana.

Entre las estrategias preventivas y de control se encuentran la vacunación, el tratamiento de casos, y la mejora de las condiciones de saneamiento e higiene. La respuesta eficaz a una epidemia exige la coordinación entre distintos actores, incluidos gobiernos, organizaciones internacionales y comunidades locales. Para erradicar la enfermedad, pueden llevarse a cabo campañas masivas de vacunación y otras intervenciones a largo plazo. Un enfoque holístico, que integre factores médicos, sociales, económicos, psicológicos y ambientales, resulta clave para mejorar la prevención y respuesta ante epidemias mediante la aplicación de prácticas innovadoras y enfoques interdisciplinarios.

Los tratamientos para enfermedades epidémicas varían según el agente patógeno. Entre las principales medidas de prevención y control se incluyen:
\begin{itemize}
  \item \textbf{Vacunación}, como herramienta fundamental en la prevención de enfermedades transmisibles.
  \item \textbf{Distanciamiento social y cuarentena}, para limitar el contacto entre personas y reducir la transmisión.
  \item \textbf{Higiene y saneamiento}, mediante el uso de mascarillas, el lavado de manos y la mejora del entorno sanitario.
  \item \textbf{Educación y comunicación}, para informar a la población sobre síntomas, prevención y medidas de respuesta.
  \item \textbf{Refuerzo del sistema de salud}, con inversión en recursos humanos, equipamiento e infraestructuras.
\end{itemize}

\subsection{Ejemplos de epidemias y pandemias a lo largo de la historia}
A lo largo de la historia, distintas epidemias han dejado una huella significativa en la sociedad y la salud pública. Algunos ejemplos relevantes son:
\begin{enumerate}
    \item \textbf{La Peste Negra (1347-1351)}: pandemia de peste bubónica causada por la bacteria Yersinia pestis, que provocó la muerte de entre 25 y 30 millones de personas en Europa, aproximadamente un tercio de la población del continente en ese momento. Sus consecuencias económicas y sociales fueron profundas, incluyendo transformaciones en la estructura feudal y escasez de mano de obra.
    \item \textbf{La gripe española (1918-1919)}: provocada por el virus de la influenza A H1N1, infectó a un tercio de la población mundial y causó la muerte de aproximadamente 50 millones de personas. La alta mortalidad se debió a la falta de tratamientos efectivos y a la sobrecarga de los sistemas de salud. Esta pandemia impulsó el desarrollo de los primeros sistemas de vigilancia epidemiológica.
    \item \textbf{VIH/SIDA (desde 1981)}: el virus de la inmunodeficiencia humana ha causado una pandemia global que ha provocado más de 26 millones de muertes. Ha tenido un impacto desproporcionado en regiones como África subsahariana, aunque se han logrado importantes avances en investigación, prevención y tratamiento.
    \item \textbf{Ébola (2013-2016)}: el brote en África Occidental causó alrededor de 11.000 muertes y colapsó los sistemas de salud locales. La respuesta internacional incluyó el desarrollo de tratamientos y vacunas, así como mejoras en infraestructuras sanitarias.
    \item \textbf{COVID-19 (desde 2019)}: causada por el virus SARS-CoV-2, ha provocado millones de muertes y ha tenido un impacto sin precedentes en la salud pública, la economía y la vida cotidiana. Ha subrayado la importancia de la preparación ante emergencias, la cooperación internacional y la rápida implementación de medidas de salud pública.
\end{enumerate}
Estos ejemplos ilustran cómo las epidemias pueden afectar profundamente a las sociedades y refuerzan la necesidad de una vigilancia epidemiológica sólida, preparación ante emergencias y cooperación global para mitigar sus impactos.

\subsection{Epidemiología}
La epidemiología es la ciencia que estudia la distribución y los determinantes de los eventos relacionados con la salud en poblaciones específicas, así como la aplicación de ese conocimiento para la prevención y control de problemas de salud. Esta disciplina se enfoca en identificar factores de riesgo, causas de enfermedades y en desarrollar e implementar intervenciones para su control.

En el contexto de las epidemias, la epidemiología permite investigar brotes infecciosos localizados en tiempo y espacio, facilitando la identificación de la fuente, el riesgo y los mecanismos de transmisión. Por ejemplo, durante la epidemia de Ébola en África Occidental, los estudios epidemiológicos ayudaron a identificar los patrones de transmisión y a establecer medidas de control efectivas.

En el caso de las pandemias, la epidemiología adquiere una dimensión global, evaluando la propagación de enfermedades en múltiples países. La pandemia de COVID-19 ha demostrado el papel esencial de la epidemiología en la comprensión de los mecanismos de transmisión, en el diseño de medidas de prevención y en la coordinación de respuestas internacionales. La vigilancia epidemiológica global y el uso de tecnologías avanzadas para el diagnóstico y rastreo de contactos han sido herramientas fundamentales en su gestión.

En conclusión, la epidemiología resulta crucial para la identificación, prevención y control de epidemias y pandemias. A través de sus métodos de investigación, permite comprender la distribución de las enfermedades, identificar sus causas y diseñar intervenciones eficaces para proteger la salud pública.



\section{Modelos epidemiológicos  deterministas}
\subsection{Modelos epidemiologícos}
Los modelos epidemiológicos son herramientas matemáticas y computacionales que permiten representar de manera simplificada los procesos biológicos, sociales y ambientales que influyen en la propagación de enfermedades dentro de una población. Estos modelos describen cómo las enfermedades infecciosas (y, en algunos casos, no infecciosas) se transmiten entre individuos y cómo varía el estado de salud de la población a lo largo del tiempo.

En términos generales, un modelo epidemiológico intenta captar las dinámicas clave del contagio, recuperación, inmunidad, nacimiento y muerte, mediante el uso de ecuaciones diferenciales, teoría de probabilidades o simulaciones computacionales. Dependiendo de su complejidad, pueden incorporar factores como heterogeneidad en la población, movilidad, redes de contacto, y respuesta a intervenciones sanitarias.

Entre los objetivos principales de los modelos epidemiológicos se encuentran:
\begin{itemize}
    \item Describir la dinámica de transmisión de enfermedades en poblaciones susceptibles, entender cómo y por qué una enfermedad se propaga o desaparece.
    \item Estimar parámetros clave como la duración de la infección, la tasa de transmisión, número básico de reproducción R0 y la proporción de población que debe ser vacunada para llegar a alcanzar la inmunidad colectiva.
    \item Explorar escenarios hipotéticos, permitiendo simular diversas condiciones y prever la evolución en el tiempo de una epidemia bajo diferentes supuestos.
    \item Evaluar estrategias de intervención y control como vacunación, cuarentena, aislamiento de casos, uso de mascarillas, cierre de escuelas. permiten calcular el impacto potencial de estas medidas antes de aplicarlas.
    \item Guiar decisiones en salud pública, ofreciendo soporte cuantitativo para la toma de decisiones en contextos de brotes, pandemias o planificación preventiva.
    \item Comprender el impacto de factores sociales y demográficos, como la densidad poblacional, la movilidad geográfica, la estructura etaria o el comportamiento humano, sobre la propagación de enfermedades.
    \item Contribuir al diseño de políticas sanitarias efectivas, mediante la identificación de puntos críticos donde las intervenciones pueden ser más eficaces o eficientes.
\end{itemize}

Los modelos epidemiológicos pueden clasificarse, entre otros criterios, según la ausencia o presencia de aleatoriedad:
\begin{itemize}
    \item \textbf{Deterministas}, aquellos en los que la evolución del fenómeno depende de manera unívoca del conjunto de condiciones iniciales y de los parámetros establecidos. Es decir, no existe aleatoriedad en el proceso, si se repite el modelo con los mismos valores, el resultado será siempre el mismo. Este tipo de modelos se basa habitualmente en ecuaciones diferenciales y resulta útil para analizar el comportamiento general de una enfermedad en poblaciones grandes, donde se busca una aproximación global y reproducible de un fenómeno.
    \item \textbf{Estocásticos}, incorporan procesos aleatorios, lo que implica que, aun con el mismo conjunto de variables y parámetros, la solución del modelo puede variar en cada ejecución. Esto permite captar mejor la variabilidad inherente a fenómenos reales, especialmente en contextos donde las poblaciones son pequeñas o existe un alto grado de incertidumbre.
\end{itemize}
	
Para el desarrollo del este trabajo se ha elegido el uso de modelos deterministas, dado que permiten describir de manera clara y estructurada la evolución de una enfermedad en función de condiciones iniciales y parámetros conocidos. Este enfoque facilita el análisis matemático y la interpretación de los resultados. Además, es útil cuando se trabaja con poblaciones grandes y se busca comprender el comportamiento general de propagación de una enfermedad.



\section{Vacunación}
\subsection{Importancia}
La vacunación constituye una de las intervenciones más eficaces y trascendentales en la historia de la salud pública. Su implementación ha permitido prevenir la propagación de enfermedades infecciosas, reducir de forma significativa la morbilidad y la mortalidad asociadas, y, en algunos casos, erradicar por completo determinadas patologías.

Según estimaciones de la Organización Mundial de la Salud (OMS), la vacunación previene entre 3,5 y 5 millones de muertes anuales causadas por enfermedades como la difteria, el tétanos, la tos ferina, la influenza y el sarampión. Gracias a los programas de inmunización masiva, millones de vidas se salvan cada año, y muchas enfermedades han sido controladas de forma notable.

En el contexto de las epidemias, la vacunación no solo protege a los individuos vacunados, sino que también contribuye al establecimiento de la inmunidad colectiva o inmunidad de grupo, lo cual reduce la transmisión comunitaria y protege a las personas que no pueden vacunarse, como los inmunocomprometidos o los niños demasiado pequeños.

Además de los beneficios en salud, la vacunación genera impactos económicos y sociales significativos: reduce los costos médicos directos, disminuye la carga sobre los sistemas sanitarios y mejora la productividad al minimizar las pérdidas laborales y escolares. La vacunación universal, tanto rutinaria como de recuperación, es un componente crítico de la atención médica de calidad.

\subsection{Tipos}
Existen diferentes formas de clasificar la vacunación, según su finalidad, estrategia poblacional y tipo de vacuna utilizada:
\begin{enumerate}
    \item \textbf{Según la finalidad}:
    \begin{itemize}
        \item \textbf{Vacunación preventiva}: se administra antes del contacto con el patógeno, con el objetivo de evitar la aparición de la enfermedad. Es el tipo más común y se aplica ampliamente en calendarios infantiles y adultos.
        \item \textbf{Vacunación terapéutica}: se utiliza cuando la persona ya ha sido infectada, buscando reforzar la respuesta inmunitaria o controlar la progresión de la enfermedad. Es menos común y se investiga especialmente en enfermedades como el VIH o ciertos tipos de cáncer.
    \end{itemize}
    \item \textbf{Según la estrategia poblacional}:
    \begin{itemize}
        \item \textbf{Vacunación rutinaria}: incluida en los calendarios de inmunización regulares, generalmente dirigida a la población infantil o a grupos definidos por edad.
        \item \textbf{Vacunación de campaña}: se aplica en situaciones de emergencia, brotes epidémicos o pandemias, para contener la propagación rápida del patógeno.
        \item \textbf{Vacunación selectiva}: dirigida a grupos de riesgo, como personas mayores, embarazadas o personal sanitario, debido a su vulnerabilidad o exposición.
        \item \textbf{Vacunación masiva}: tiene como objetivo alcanzar rápidamente una amplia cobertura poblacional para lograr inmunidad colectiva, como ocurrió durante la pandemia de COVID-19.
    \end{itemize}
    
    \item \textbf{Según el tipo de vacuna}:
    \begin{itemize}
        \item \textbf{Vacunas inactivadas}: contienen el patógeno muerto, incapaz de causar enfermedad pero capaz de generar inmunidad. 
        \item \textbf{Vacunas atenuadas}: contienen microorganismos vivos debilitados que no causan enfermedad en individuos sanos.
        \item \textbf{Vacunas de subunidades o conjugadas}: incluyen fragmentos del patógeno, como proteínas o azúcares.
        \item \textbf{Vacunas recombinantes}: elaboradas mediante ingeniería genética, donde se insertan genes del patógeno en otros organismos para producir antígenos. 
        \item \textbf{Vacunas de ARN mensajero (ARNm)}: utilizan instrucciones genéticas para que las células del cuerpo produzcan proteínas virales y generen una respuesta inmunitaria.
        \item \textbf{Vacunas vectoriales}: usan un virus modificado (vector) para introducir el material genético del patógeno. 
    \end{itemize}

 
\end{enumerate}

\subsection{Evolución e historia}
La historia de la vacunación es un recorrido fundamental en el desarrollo de la medicina y la salud pública. Se remonta a observaciones empíricas sobre la inmunidad natural tras ciertas enfermedades. En 1796, \textbf{Edward Jenner} desarrolló la primera vacuna al utilizar material de pústulas de vacas infectadas con viruela bovina para prevenir la viruela humana, marcando el inicio de la vacunación moderna.

Durante el siglo XIX, \textbf{Louis Pasteur} realizó importantes aportes al crear vacunas utilizando formas atenuadas de microorganismos, desarrollando inmunizaciones contra enfermedades como la rabia. Su trabajo, junto con la teoría microbiana de la enfermedad y las investigaciones de \textbf{Robert Koch}, permitió el desarrollo de vacunas contra la tuberculosis, la difteria y otras enfermedades infecciosas.

En el siglo XX, los avances en biotecnología, como el cultivo celular, posibilitaron la producción de vacunas contra enfermedades como la poliomielitis, el sarampión, las paperas y la rubéola. Posteriormente, la biología molecular permitió el desarrollo de vacunas recombinantes, como la vacuna contra la hepatitis B.

En la actualidad, la tecnología de ARN mensajero ha revolucionado la vacunología, facilitando el desarrollo rápido de vacunas, como se evidenció con las desarrolladas frente al virus SARS-CoV-2 durante la pandemia de COVID-19.

A lo largo de su evolución, la vacunación ha enfrentado retos importantes, como la hesitación vacunal y la necesidad de mejorar la cobertura global, pero su impacto en la salud pública y el desarrollo social y económico es innegable.

\subsection{Impacto y desafíos actuales}
A pesar del impacto positivo y ampliamente documentado de la vacunación, persisten diversos desafíos que afectan su implementación y sostenibilidad a nivel global:
\begin{itemize}
    \item \textbf{Hesitación vacunal y desconfianza}: la desinformación, las creencias en métodos alternativos o "naturales" y la falta de confianza en los sistemas sanitarios generan una baja aceptación de las vacunas en ciertos grupos. La American Academy of Pediatrics subraya la necesidad de estrategias de comunicación eficaces para combatir estas percepciones erróneas.
    \item \textbf{Desigualdades en el acceso}: en muchas regiones de bajos ingresos, existen barreras económicas, logísticas y estructurales que dificultan la distribución equitativa de vacunas. Iniciativas como \textbf{\textit{Gavi, la Alianza para las Vacunas}}, han mejorado la situación, pero aún persisten importantes brechas.
    \item \textbf{Infraestructura limitada}: los sistemas de salud en países de ingresos bajos y medios suelen carecer del personal capacitado, recursos financieros y logística necesarios para mantener programas de vacunación sostenibles, especialmente en lo relativo a la cadena de frío.
    \item \textbf{Impacto de la pandemia de COVID-19}: la crisis sanitaria global interrumpió numerosos programas de vacunación rutinaria, reduciendo las tasas de cobertura y aumentando la exposición a enfermedades prevenibles. Recuperar estos niveles requiere esfuerzos coordinados para restaurar el acceso y la confianza en los servicios de salud.
    \item \textbf{Resurgimiento de enfermedades}: La disminución en la cobertura vacunal ha provocado brotes recientes de enfermedades como el sarampión en comunidades con baja vacunación. La OMS insiste en la necesidad de mantener altas tasas de cobertura para prevenir estas situaciones.
\end{itemize}


La vacunación es una herramienta esencial para la prevención y el control de enfermedades infecciosas. No obstante, para maximizar su impacto, es imprescindible afrontar los desafíos mencionados mediante políticas públicas eficaces, campañas de información, cooperación internacional y fortalecimiento de los sistemas sanitarios.

\section{SIDA/VIH}
El síndrome de inmunodeficiencia adquirida (SIDA) es una enfermedad causada por la infección del virus de la inmunodeficiencia humana (VIH). El VIH es un retrovirus que ataca y destruye las células CD4+ T, esenciales para el sistema inmunológico, lo que provoca una inmunodeficiencia progresiva. Como consecuencia, las personas infectadas se vuelven más susceptibles a infecciones oportunistas y ciertos tipos de cáncer. 
\subsection{Vías de transmisión}
El VIH se transmite a través de varias vías bien definidas:
\begin{itemize}
    \item Contacto sexual: principal forma de transmisión. Ocurre en relaciones sexuales vaginales, anales y orales sin protección, mediante fluidos genitales o rectales infectados.
    \item Uso compartido de agujas: frecuente entre personas usuarias de drogas intravenosas. Es una vía de alto riesgo por la introducción directa del virus en la sangre.
    \item Transfusiones de sangre y productos sanguíneos: aunque posible, su riesgo es mínimo en países donde se realizan pruebas rigurosas en bancos de sangre.
    \item Transmisión madre-hijo: puede ocurrir durante el embarazo, parto o lactancia. El uso de TAR y cesáreas programadas reduce significativamente el riesgo.
    \item Trasplantes de órganos o tejidos: es raro debido a los controles exhaustivos.
    \item Exposición ocupacional: riesgo bajo en profesionales de la salud si se siguen las precauciones adecuadas.
\end{itemize}

\subsection{Diagnóstico}
El diagnóstico se basa en pruebas recomendadas por los Centros para el Control y la Prevención de Enfermedades (CDC):
\begin{enumerate}
    \item Inmunoensayo combinado Ag/Ab: detecta anticuerpos contra VIH-1/VIH-2 y antígeno p24. Puede identificar la infección 2–3 semanas después de la exposición.
    \item Prueba de diferenciación de anticuerpos: permite distinguir entre VIH-1 y VIH-2.
    \item Prueba de amplificación de ácidos nucleicos (NAAT): se emplea si hay resultados indeterminados. Confirma infecciones agudas por VIH-1.
    \item Pruebas rápidas: Ofrecen resultados preliminares en menos de 20 minutos. Se deben confirmar con pruebas de laboratorio.
\end{enumerate}

\subsection{Tratamiento}
El tratamiento se basa en la terapia antirretroviral (TAR), que ha convertido al VIH en una enfermedad crónica controlable.

\textbf{Opciones terapéuticas}. Los inhibidores de la integrasa (INSTI) constituyen la primera línea de tratamiento, generalmente combinados con tenofovir. También existen regímenes de dos medicamentos, como dolutegravir junto con 3TC, que se pueden emplear tras una evaluación inicial adecuada. Además, se utilizan terapias de acción prolongada, por ejemplo, cabotegravir y rilpivirina, que se administran mensualmente o bimensualmente.

\textbf{Consideraciones clínicas}. Entre los efectos secundarios comunes se incluyen náuseas, cefaleas, acidosis láctica y hepatomegalia con esteatosis. Es fundamental ajustar el tratamiento considerando la función renal y hepática del paciente, así como posibles interacciones medicamentosas.

\textbf{Impacto psicosocial}. El estigma asociado al VIH puede afectar negativamente la adherencia al tratamiento y la calidad de vida de los pacientes. Por ello, el apoyo psicológico y social resulta esencial para mejorar su bienestar general.

\subsection{Estrategias de prevención}
La prevención del VIH/SIDA combina medidas farmacológicas, no farmacológicas y de salud pública.

\textbf{Enfoques farmacológicos}.
La profilaxis preexposición (PrEP) consiste en el uso diario de antirretrovirales en personas con alto riesgo de contraer el virus. La profilaxis postexposición (PEP) implica la administración inmediata de antirretrovirales tras una posible exposición, como relaciones sexuales sin protección o accidentes laborales. Además, el tratamiento como prevención (TasP) se basa en que las personas con carga viral indetectable no transmiten el virus, lo que se resume en el principio U=U (indetectable = intransmisible).

\textbf{Enfoques no farmacológicos}.
Entre las medidas no farmacológicas se encuentran el uso de preservativos como barrera física, la circuncisión masculina que reduce el riesgo de infección, la educación y consejería para promover prácticas sexuales seguras, y los programas de intercambio de agujas que disminuyen la transmisión entre usuarios de drogas intravenosas.

\textbf{Medidas de salud pública}.
Las estrategias de salud pública incluyen el diagnóstico temprano del VIH para iniciar tratamiento oportuno, la prevención de la transmisión de madre a hijo mediante intervenciones médicas, y las intervenciones estructurales destinadas a reducir desigualdades sociales y eliminar barreras al acceso a la atención médica.

\subsection{Progresión clínica}
Tras la infección inicial, se observan tres fases: [10][11][12]
\begin{enumerate}
    \item \textbf{Fase aguda}: síntomas similares a los de una gripe.
    \item \textbf{Fase de latencia clínica}: puede durar años; el paciente es asintomático mientras el virus se replica en niveles bajos.
    \item \textbf{Progresión a SIDA}: ocurre cuando el recuento de CD4+ cae por debajo de 200 células/µL o aparecen infecciones oportunistas (ej. Pneumocystis jirovecii, sarcoma de Kaposi, candidiasis esofágica).
\end{enumerate}
La media del tiempo entre la infección y el desarrollo de SIDA en ausencia de tratamiento es de 8 a 10 años. Factores como la carga viral, coinfecciones o la respuesta inmune individual influyen en la progresión.

\subsection{Evolución e impacto}
Desde los años 80, el impacto del SIDA ha cambiado significativamente gracias al desarrollo de la terapia antirretroviral de gran actividad (TARGA). La incidencia de enfermedades oportunistas se redujo drásticamente, pasando de 30.7 a 2.5 casos por cada 100 años-paciente entre 1994 y 1998, según el estudio EuroSIDA. Además, se ha observado un aumento en el recuento de células CD4+ en los diagnósticos recientes, lo que indica un mejor manejo clínico de la enfermedad. Sin embargo, el impacto del SIDA no ha sido uniforme; en Estados Unidos, por ejemplo, ha habido un aumento de infecciones en mujeres, jóvenes y minorías. Los avances en pruebas diagnósticas, como las pruebas rápidas orales, han facilitado la detección temprana del VIH, aunque persisten desafíos importantes, como la falta de acceso a la atención médica en poblaciones vulnerables y el diagnóstico tardío.


\section{Gonorrea}
La gonorrea es una infección de transmisión sexual (ITS) causada por la bacteria Neisseria gonorrhoeae. Puede afectar el tracto urogenital, recto, faringe y, en casos raros, diseminarse sistémicamente. Es una de las ITS más comunes a nivel mundial, con una incidencia anual estimada de 86.9 millones de casos en adultos.

\subsection{Vías de transmisión}
La gonorrea se transmite principalmente a través del contacto sexual sin protección. Durante las relaciones sexuales vaginales, la bacteria puede infectar el tracto genital tanto en hombres como en mujeres. En el caso de las relaciones sexuales anales, la infección puede afectar el recto. Asimismo, durante el sexo oral sin protección, Neisseria gonorrhoeae puede colonizar la faringe.

Otra vía de transmisión es la perinatal, en la cual una madre infectada puede transmitir la bacteria a su bebé durante el parto, lo que puede provocar conjuntivitis neonatal. También se ha documentado la posibilidad de transmisión al utilizar saliva como lubricante, especialmente entre hombres que tienen sexo con hombres (HSH). Además, aunque de forma menos común, existe evidencia limitada que sugiere que la gonorrea orofaríngea podría transmitirse a través del beso profundo.

\subsection{Diagnóstico}
El diagnóstico de la gonorrea debe considerar las posibles infecciones en el tracto urogenital, recto y faringe. Los métodos principales incluyen pruebas de amplificación de ácidos nucleicos (NAAT), cultivo bacteriano y tinción de Gram.

Las pruebas de amplificación de ácidos nucleicos (NAAT) son el estándar de oro debido a su alta sensibilidad y especificidad. Permiten el uso de diferentes tipos de muestras, como endocervicales, vaginales, uretrales, de orina, rectales y faríngeas. Aunque su uso en sitios extragenitales no siempre está aprobado por la FDA, muchos laboratorios han validado estas pruebas y las utilizan con frecuencia. Además, ofrecen la ventaja de facilitar el autodiagnóstico mediante muestras autocolectadas por los pacientes.

El cultivo bacteriano, aunque menos sensible que las NAAT, sigue siendo fundamental para la detección de cepas resistentes a los antimicrobianos. Requiere muestras como las uretrales, endocervicales, rectales, faríngeas o conjuntivales, y es especialmente útil para realizar pruebas de susceptibilidad en casos de fracaso del tratamiento.

La tinción de Gram puede ser diagnóstica en hombres sintomáticos cuando se observan diplococos gramnegativos intracelulares en secreciones uretrales. Sin embargo, su sensibilidad es menor en mujeres y en sitios extragenitales, por lo que no se recomienda como única prueba diagnóstica en estos casos.

La combinación de estos métodos permite un diagnóstico oportuno y efectivo, lo que es clave para el tratamiento adecuado y la prevención de complicaciones y nuevas transmisiones.

\subsection{Prevención}
Estrategias recomendadas incluyen:
\begin{itemize}
    \item Uso de preservativos: Correcto y consistente en relaciones vaginales, anales y orales.
    \item Educación y consejería: Sobre prácticas sexuales seguras y reducción de riesgos.
    \item Detección y tratamiento de parejas sexuales: Para evitar reinfecciones y transmisión. Se recomienda la terapia de pareja expedita (EPT) cuando el acceso a servicios es limitado.
    \item Cribado regular: Anual para mujeres menores de 25 años, mujeres con factores de riesgo, y hombres que tienen sexo con hombres en todos los sitios de exposición.
    \item Profilaxis postexposición (PEP): No estándar para gonorrea, pero estudios recientes sugieren posible beneficio con doxiciclina en ciertos grupos de alto riesgo.
    \item Promoción de salud pública: Acceso a servicios, reducción del estigma y políticas de salud pública efectivas.
\end{itemize}

\subsection{Clínica y complicaciones}
Los síntomas de la gonorrea varían según el sexo. En los hombres, se manifiesta comúnmente como uretritis con disuria y secreción purulenta. En las mujeres, muchas veces es asintomática, aunque puede presentarse como cervicitis con secreción mucopurulenta y dolor pélvico.

Las complicaciones también difieren. En las mujeres, puede derivar en enfermedad inflamatoria pélvica, infertilidad, embarazo ectópico y dolor crónico. En los hombres, la principal complicación es la epididimitis, que también puede conllevar infertilidad. En ambos sexos, existe el riesgo de desarrollar una infección gonocócica diseminada, caracterizada por dermatitis, tenosinovitis y artritis migratoria.

Las infecciones extragenitales incluyen proctitis, faringitis que suele ser asintomática y conjuntivitis neonatal cuando ocurre transmisión perinatal.

Desde el punto de vista de la salud pública, la gonorrea incrementa el riesgo de transmisión del VIH. Además, la creciente resistencia a los antimicrobianos representa un desafío significativo para su control y tratamiento.

\subsection{Historia y evolución}
La gonorrea ha sido una infección de transmisión sexual prevalente durante miles de años. Desde la introducción de los antibióticos, \textit{Neisseria gonorrhoeae} ha desarrollado resistencia progresiva a cada clase utilizada. Primero se observó con las sulfonamidas y penicilinas en las décadas de 1940 y 1950. Posteriormente, surgió resistencia a las tetraciclinas y macrólidos, y más tarde a las fluoroquinolonas durante las décadas de 1980 y 1990.

En la actualidad, la terapia recomendada que anteriormente incluía el uso dual de ceftriaxona y azitromicina también enfrenta un aumento preocupante en la resistencia, con casos reportados de cepas resistentes a ambos antibióticos. Un ejemplo de ello es el clon FC428, resistente a la ceftriaxona, que se ha diseminado a nivel mundial desde 2017.

Esta evolución subraya la necesidad urgente de desarrollar nuevas estrategias terapéuticas y medidas efectivas de prevención para controlar la propagación de esta infección.







