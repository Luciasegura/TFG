\apendice{Plan de Proyecto Software}

\section{Introducción}
En este proyecto se ha llevado a cabo el diseño y simulación de modelos epidemiológicos deterministas utilizando Simulink, así como el desarrollo de una aplicación interactiva en App Designer para visualizar los resultados. Para organizar adecuadamente el trabajo, se ha realizado una planificación que contempla los aspectos temporales (distribución de tareas), una estimación básica del coste, y una breve revisión de los aspectos legales relacionados con el uso del software empleado.

\section{Planificación temporal}
Para la planificación temporal del proyecto se ha seguido una metodología ágil inspirada en Scrum, adaptada a un entorno de trabajo individual.Se han utilizado herramientas de GitHub, como los issues y los milestones. Esta metodología ha favorecido una entrega progresiva de resultados, una mejora continua a lo largo de las iteraciones, y una documentación detallada de cada fase del desarrollo.

Todas las tareas y etapas del proyecto han sido organizadas y gestionadas a través del repositorio de GitHub\footnote{Link para entrar en el repositorio https://github.com/Luciasegura/TFG}. En él se encuentran documentados los milestones, issues, versiones del código y los avances del proyecto de forma detallada.

La estructura del proyecto se basa en la definición de milestones, issues y labels, que permiten dividir el trabajo en fases, tareas y categorías específicas, se explica cada una a continuación.
\begin{itemize}
    \item \textbf{Milestones}: son las grandes fases del proyecto. Cada milestone tiene una fecha límite y agrupa tareas relacionadas. Por ejemplo, la Investigación teórica es un milestone que va del 19 de febrero al 4 de marzo.
    \item \textbf{Issues}: son las tareas concretas que hay que hacer dentro de cada milestone. Por ejemplo, “estudiar modelos” es una issue dentro del milestone Investigación teórica.
    \item \textbf{Labels}: las etiquetas ayudan a identificar el tipo de tarea, por ejemplo investigación, implementación, documentación, etc. Así se sabe rápidamente de qué va cada issue.
\end{itemize}



Durante el desarrollo del proyecto, se utilizaron distintas \textit{labels} para clasificar y organizar las \textit{issues}, lo que facilitó la gestión del trabajo y el seguimiento del progreso dentro de cada \textit{milestone}. Estas etiquetas temáticas, descritas en la Tabla \ref{tab:etiquetas}, permitieron categorizar las tareas según su naturaleza técnica, conceptual o funcional.


\begin{table}[H]
    \centering
    
    \begin{tabular}{|p{4cm}|p{10cm}|}
    \hline
    \textbf{Etiqueta} & \textbf{Descripción} \\
    \hline
    \textbf{análisis} & Tareas enfocadas al estudio de resultados obtenidos y su interpretación epidemiológica o matemática. \\
    \hline
    \textbf{aplicación} & Actividades relacionadas con el diseño e implementación de la app interactiva en \texttt{App Designer}. \\
    \hline
    \textbf{conceptos epidemiológicos} & Estudio de conceptos fundamentales como el número reproductivo básico, tasas de transmisión o inmunidad. \\
    \hline
    \textbf{documentation} & Redacción de la memoria, anexos, bibliografía y otros documentos del TFG. \\
    \hline
    \textbf{investigación} & Fase teórica del proyecto: revisión de literatura, estudio de modelos y recopilación de información. \\
    \hline
    \textbf{matlab} & Uso de \texttt{MATLAB} para implementar modelos, generar gráficas o scripts de simulación. \\
    \hline
    \textbf{mejora modelo} & Cambios y ajustes realizados para optimizar el comportamiento de los modelos implementados. \\
    \hline
    \textbf{modelado matemático} & Tareas relacionadas con la formulación y análisis matemático de los modelos compartimentales. \\
    \hline
    \textbf{modelo epidemiológico} & Implementación y análisis de modelos como SI, SIS, SIR, SEIR, SIRV y SEIRV. \\
    \hline
    \textbf{PID} & Aplicación del regulador PID para el control de la propagación epidémica en el modelo. \\
    \hline
    \textbf{regulador} & Estudio y aplicación de técnicas de control para estabilizar o reducir los contagios. \\
    \hline
    \textbf{simulación} & Ejecución de experimentos computacionales en \texttt{MATLAB} o \texttt{Simulink}. \\
    \hline
    \textbf{simulink} & Desarrollo de diagramas de bloques en \texttt{Simulink} para representar los modelos dinámicos. \\
    \hline
    \textbf{vacunación} & Inclusión del efecto de la vacuna en los modelos y análisis del impacto en la propagación. \\
    \hline
    \textbf{variables y parámetros} & Estudio, definición y ajuste de los parámetros clave (como $\beta$, $\gamma$, $\nu$) y variables del sistema. \\
    \hline
    
    \end{tabular}

\caption{Etiquetas temáticas (labels) utilizadas en la planificación del proyecto}
\label{tab:etiquetas}
\end{table}
Estas etiquetas ayudaron a identificar con claridad el objetivo de cada tarea, y permiten entender el enfoque multidisciplinar del trabajo: desde la teoría matemática hasta la aplicación práctica en simuladores y herramientas digitales.



A continuación se explican las milestones con sus issues  correspondientes.
\subsection{Milestone 1: Investigación teórica}
\textbf{Periodo:} 19 de febrero – 14 de mayo

Durante esta fase se llevó a cabo el estudio de los fundamentos del modelado matemático de epidemias. Se analizaron las diferencias entre los modelos compartimentales (SI, SIS, SIR y SEIR), sus variables y parámetros, y se profundizó en los conceptos epidemiológicos necesarios para entender su comportamiento.

\begin{itemize}
    \item Estudio de diferencias entre modelos compartimentales y análisis de variables y parámetros. Se estudian las distintas variantes de modelos compartimentales (SI, SIR, SEIR, etc.) y cómo se definen e interpretan sus variables y parámetros, lo que permite entender mejor su estructura y aplicabilidad. 
    
    \textit{(24 de febrero – 9 de abril)}
    \item Estudio de conceptos básicos de epidemiología. Se investigan los fundamentos teóricos de la epidemiología y su relación con los modelos matemáticos, incluyendo tasas de infección, recuperación, periodo de incubación, y más. 
    
    \textit{(19 de febrero – 4 de marzo)}
    \item Investigación del compartimento de vacunación y su efecto en la dinámica del modelo. Se analiza cómo introducir el efecto de la vacunación dentro de los modelos, especialmente en variantes como SIRV y SEIRV, y se investiga su impacto epidemiológico. 
    
    \textit{(19 de febrero – 14 de mayo)}
\end{itemize}

\subsection*{Milestone 2: Implementación en Simulink}
\textbf{Periodo:} 6 de marzo – 16 de mayo

Se implementaron los modelos epidemiológicos en \texttt{Simulink}, analizando el efecto de los parámetros y añadiendo el compartimento de vacunación. Esta fase permitió simular distintos escenarios de propagación.

\begin{itemize}
    \item Implementación de modelos epidemiológicos. Se lleva a cabo la implementación de los modelos básicos en el entorno gráfico de Simulink, permitiendo visualizar la evolución de las epidemias a través de diagramas de bloques. 

    
    \textit{(6 de marzo – 1 de abril)}
    \item Análisis del efecto de los parámetros y mejora del modelo. Se experimenta cómo pequeñas variaciones en parámetros clave (como beta, gamma, sigma) afectan la dinámica del modelo, lo que permite afinar su comportamiento y ajustarlo a escenarios realistas.  
    
    \textit{(6 de marzo – 18 de abril)}
    \item Incorporación del compartimento de vacunación en los modelos. Se integra la vacunación en los modelos ya implementados, ajustando ecuaciones y bloques para simular campañas vacunales constantes.  
    
    \textit{(6 de marzo – 16 de mayo)}
\end{itemize}

\subsection*{Milestone 3: Análisis de datos reales}
\textbf{Periodo:} 23 de marzo – 16 de mayo

En esta etapa se buscaron enfermedades reales susceptibles de ser modeladas mediante los modelos implementados, y se realizaron simulaciones comparativas.

\begin{itemize}
    \item Buscar enfermedades e investigar modelos aplicables.
    Se buscan datos de enfermedades reales para identificar cuáles se ajustan a los modelos creados y permiten una validación básica de su comportamiento. 
    
    \textit{(10 de abril – 16 de mayo)}
    \item Simulación de modelos relacionados con enfermedades reales mediante \texttt{Simulink}. Se utilizan los modelos implementados para simular el comportamiento de ciertas enfermedades y comparar sus curvas teóricas con datos reales.

    
    \textit{(23 de marzo - 16 de mayo)}
\end{itemize}

\subsection*{Milestone 4: Control}
\textbf{Periodo:} 16 de mayo – 6 de junio

Se estudió el impacto de medidas de control como la vacunación y el distanciamiento social, y se implementó un controlador PID en \texttt{MATLAB} para la gestión dinámica de los infectados.

\begin{itemize}
    \item Análisis de la vacunación como medida de control. Se evalúa el impacto de la vacunación dentro del modelo como una forma de mitigar o evitar la propagación epidémica. 
    
    \textit{(16 – 19 de mayo)}
    \item Estudio de medidas de control no farmacológicas. Se investigan otros mecanismos de control (cuarentenas, reducción de contactos, etc.) y cómo se pueden representar matemáticamente en el modelo. 
    
    \textit{(19 de mayo – 3 de junio)}
    \item Implementación y simulación del controlador PID. Se aplica un controlador PID en MATLAB para actuar sobre la tasa de infección o el número de infectados, simulando medidas de control automáticas sobre el sistema.
    
    \textit{(5 – 6 de junio)}
\end{itemize}

\subsection*{Milestone 5: Aplicación}
\textbf{Periodo:} 16 de mayo – 3 de junio

Se desarrolló una interfaz interactiva en \texttt{App Designer} para facilitar la visualización y manipulación de los modelos, promoviendo la comprensión intuitiva del comportamiento epidémico.

\begin{itemize}
    \item Diseño inicial de la aplicación en \texttt{MATLAB}. Se define cómo debe funcionar la interfaz gráfica de usuario para que permita modificar parámetros y visualizar las simulaciones de forma interactiva.
    
    \textit{(16 – 19 de mayo)}
    \item Implementación funcional de la aplicación. Se desarrolla la app que integra los modelos epidemiológicos con una interfaz usable, permitiendo simular, visualizar y analizar distintos escenarios epidémicos.

    \textit{(19 de mayo – 3 de junio)}
\end{itemize}

\subsection*{Milestone 6: Redacción en LaTeX}
\textbf{Periodo:} 7 de mayo – 8 de junio

Durante esta etapa se elaboró la memoria del proyecto y sus respectivos anexos, documentando todos los aspectos teóricos, prácticos y experimentales desarrollados.

\begin{itemize}
    \item Redacción de la memoria principal del TFG. Se escribe la documentación del proyecto, incluyendo introducción, objetivos, metodología, resultados y conclusiones, utilizando LaTeX.
    
    \textit{(7 de mayo – 6 de junio)}
    \item Preparación y revisión de los anexos. Se desarrollan los anexos donde se detallan aspectos más técnicos del proyecto como diseño en Simulink, scripts de MATLAB, estructuras de la aplicación y sostenibilidad. 
    
    \textit{(16 de mayo – 8 de junio)}
\end{itemize}




\section{Planificación económica}
La planificación económica de este trabajo tiene como finalidad estimar y justificar los recursos utilizados durante su desarrollo, tanto materiales como humanos. Se han clasificado los costes en tres grandes categorías: hardware, software y mano de obra. Además, se incluye un resumen económico total que refleja el coste imputado al TFG, es decir, el valor proporcional de cada recurso directamente relacionado con el proyecto. Esta estimación proporciona una visión global del esfuerzo económico asociado y permite valorar la inversión requerida para llevar a cabo un proyecto de estas características.

\subsection{Costes de hardware}

Se recoge en la Tabla \ref{hardware} el coste del equipo informático utilizado durante el desarrollo del TFG. Aunque el equipo ya estaba disponible, se ha imputado un coste proporcional en función del uso específico para el proyecto
\begin{table}[H]

\centering
\small
\begin{tabular}{|l|l|c|c|}
\hline
\textbf{Concepto} & \textbf{Descripción} & \textbf{Coste total (€)} & \textbf{Coste imputado (€)} \\
\hline
Portátil & Asus Vivobook X409JB & 650 & 350 \\
\hline
\end{tabular}
\caption{Costes de hardware imputados al TFG}
\label{hardware}
\end{table}

\subsection{Costes de Software}
Se incluyen en la Tabla \ref{software} las licencias de software necesarias para el desarrollo del TFG, como MATLAB, Simulink, Office y el sistema operativo Windows. En cada caso, se ha estimado el grado de uso del software en relación al TFG para calcular el coste imputado.
\begin{table}[H]
\centering
\small
\begin{tabular}{|l|l|c|c|c|}
\hline
\textbf{Concepto} & \textbf{Licencia} & \textbf{Coste (€)} & \textbf{Uso TFG} & \textbf{Imputado (€)} \\
\hline
Windows & Windows 10 Pro\footnote{Obtenido de \cite{microsoft365personal}} & 150 & 50\% & 75 \\
Office & Office 365 (1 año)\footnote{Obtienido de \cite{microsoft_windows11_get}} & 69 & 100\% & 69 \\
MATLAB & Educativa estudiante\footnote{Obtenido de \cite{mathworks_matlab_student}} & 150 & 100\% & 150 \\
Simulink & Incluida en MATLAB & 0 & 100\% & 0 \\
\hline
\end{tabular}
\caption{Costes de software imputados al TFG}
\label{software}
\end{table}

\subsection{Costes de Mano de Obra}
Este apartado refleja en la Tabla \ref{mano obra} el valor estimado del trabajo realizado por el estudiante desde una perspectiva profesional. Se ha considerado un total de 360 horas de trabajo, valoradas a razón de 20\€/hora, correspondiente a la tarifa media de un ingeniero informático junior \cite{glassdoor_junior_ing_inf}. Aunque no representa un coste real, sirve para valorar el esfuerzo implicado.
\begin{table}[H]
\centering
\begin{tabular}{|l|p{5cm}|c|c|c|}
\hline
\textbf{Concepto} & \textbf{Detalle} & \textbf{Horas} & \textbf{Precio/hora (€)} & \textbf{Coste total (€)} \\
\hline
Desarrollo app & Ingeniero & 360 & 20 & 7.200 \\
\hline
\end{tabular}
\caption{Costes de mano de obra imputados al TFG}
\label{mano obra}
\end{table}

\subsection{Resumen económico total}
En el resumen económico representado en la tabla \ref{total} se agrupan todos los costes imputados al proyecto, reflejando el coste total estimado para el desarrollo del TFG. Esta cifra proporciona una valoración económica global del proyecto teniendo en cuenta todos los recursos utilizados.
\begin{table}[h]
\centering
\begin{tabular}{|l|c|}
\hline
\textbf{Concepto} & \textbf{Coste imputado (€)} \\
\hline
Hardware & 450 \\
Software & 294 \\
Mano de obra & 7.200 \\
\hline
\textbf{Total general} & \textbf{7.944} \\
\hline
\end{tabular}
\caption{Resumen económico total del TFG}
\label{total}
\end{table}

El análisis económico realizado pone de manifiesto la implicación de recursos tanto técnicos como humanos en su desarrollo. Gran parte del material y software ya estaba disponible previamente, el coste imputado ofrece una estimación realista del valor del proyecto en un contexto profesional. Permite también reforzar la visión del trabajo como un proceso completo que, más allá del contenido académico, implica una gestión eficiente de recursos.

\section{Viabilidad legal}

Esta sección analiza los aspectos legales relacionados con el desarrollo del proyecto, con el fin de garantizar que se ha actuado conforme a la normativa vigente en materia de propiedad intelectual, uso de software, protección de datos y cumplimiento del reglamento académico.

\subsection{Propiedad intelectual y licencias}

Durante el desarrollo del trabajo se ha hecho uso de diversas herramientas y recursos, tanto propios como de terceros. Para asegurar la legalidad del uso de estos elementos, se ha revisado la licencia de cada uno de ellos:

\begin{itemize}
    \item \textbf{MATLAB y Simulink:} se han utilizado bajo licencia educativa proporcionada por la universidad, lo cual permite el uso no comercial y con fines académicos del software. Esta licencia limita su uso a entornos de formación e investigación.
    \item \textbf{Microsoft Windows 10 y Office 365:} se han empleado versiones con licencia personal, válidas para su uso en el ámbito educativo y no profesional, sin vulnerar las condiciones de uso del proveedor.
    \item \textbf{Repositorios y herramientas de desarrollo:} el código fuente del proyecto se encuentra alojado en GitHub. Se han seguido las recomendaciones de uso de la plataforma y se ha seleccionado una licencia de software libre (\texttt{MIT License}) que permite la reutilización, modificación y distribución del código, siempre que se otorgue reconocimiento al autor original.
    \item \textbf{Librerías y recursos externos:} todas las librerías de software utilizadas en el desarrollo de la aplicación han sido seleccionadas por estar bajo licencias de código abierto (como MIT o GPL), lo que permite su uso legal en proyectos académicos. En los casos donde se han empleado fragmentos de código o documentación externa, se ha incluido la correspondiente cita y fuente.
\end{itemize}

\subsection{Protección de datos}

Durante el desarrollo del proyecto se han utilizado datos relacionados con enfermedades reales, obtenidos a partir de fuentes públicas y abiertas disponibles en internet, como artículos científicos, bases de datos institucionales y documentos divulgativos. 

En todo momento se ha respetado la legalidad vigente en materia de protección de datos. Los datos empleados no contienen información personal identificable, por lo que no se consideran datos personales según el Reglamento (UE) 2016/679 (RGPD) \cite{ue2016rgpd} y la Ley Orgánica 3/2018 (LOPDGDD) \cite{lopdgdd2018}.

Además, se ha prestado especial atención a que las fuentes de donde se han extraído dichos datos sean de carácter público, accesibles libremente por cualquier usuario y con fines divulgativos, científicos o educativos. No se ha requerido consentimiento alguno al no tratarse de información sensible vinculada a individuos concretos, sino a descripciones generales de patologías, síntomas, prevalencias u otras variables de carácter epidemiológico.
Por tanto, se concluye que el uso de estos datos es legal y éticamente aceptable en el marco de un trabajo académico con fines formativos y no comerciales.

\subsection{Normativa académica y originalidad}

El trabajo ha sido realizado de forma íntegra y original por la autora del TFG, sin incurrir en plagio, copia de código sin atribución o uso indebido de materiales de terceros.
Todo el desarrollo se ha realizado conforme a las normas establecidas por la UBU\footnote{Universidad de Burgos} en relación con la elaboración de Trabajos Fin de Grado, asegurando la autoría propia, la honestidad académica y el cumplimiento del código ético de la institución.






