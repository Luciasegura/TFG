\apendice{Plan de Proyecto Software}

\section{Introdicción}
En este proyecto se ha llevado a cabo el diseño y simulación de modelos epidemiológicos deterministas utilizando Simulink, así como el desarrollo de una aplicación interactiva en App Designer para visualizar los resultados. Para organizar adecuadamente el trabajo, se ha realizado una planificación que contempla los aspectos temporales (distribución de tareas), una estimación básica del coste, y una breve revisión de los aspectos legales relacionados con el uso del software empleado.

\section{Planificación temporal}
Para la planificación temporal del proyecto se ha seguido una metodología ágil inspirada en Scrum, adaptada a un entorno de trabajo individual.Se han utilizado herramientas de GitHub, como los issues y los milestones. Esta metodología ha favorecido una entrega progresiva de resultados, una mejora continua a lo largo de las iteraciones, y una documentación detallada de cada fase del desarrollo.

Todas las tareas y etapas del proyecto han sido organizadas y gestionadas a través del repositorio de GitHub\footnote{Link para entrar en el repositorio https://github.com/Luciasegura/TFG}. En él se encuentran documentados los milestones, issues, versiones del código y los avances del proyecto de forma detallada.

La estructura del proyecto se basa en la definición de milestones, issues y labels, que permiten dividir el trabajo en fases, tareas y categorías específicas, se explica cada una a continuación.
\begin{itemize}
    \item \textbf{Milestones}: son las grandes fases del proyecto. Cada milestone tiene una fecha límite y agrupa tareas relacionadas. Por ejemplo, la Investigación teórica es un milestone que va del 19 de febrero al 4 de marzo.
    \item \textbf{Issues}: son las tareas concretas que hay que hacer dentro de cada milestone. Por ejemplo, “estudiar modelos” es una issue dentro del milestone Investigación teórica.
    \item \textbf{Labels}: las etiquetas ayudan a identificar el tipo de tarea, por ejemplo investigación, implementación, documentación, etc. Así se sabe rápidamente de qué va cada issue.
\end{itemize}

Milestone: Investigación teórica. Issues
\begin{itemize}
    \item Diferencias entre modelos investigación modelado matemático variables y parámetros (Febrero, 24 - Abril, 9)
    \item Estudiar modelos conceptos epidemiológicos investigación (Febrero, 19 - Marzo, 4)
    \item Compartimento vacuna investigación vacunación (Febrero, 19 - Mayo, 14)
\end{itemize}

Milestone: Implementación en Simulink
\begin{itemize}
    \item Efecto parámetros análisis investigación mejora modelo simulación simulink (Marzo, 6 - Abril, 18)
    \item Implementar modelos modelo epidemiológico simulación simulink (Marzo, 6 - Abril, 1)
    \item Añadir vacunación modelo epidemiológico simulación simulink vacunación (Marzo, 6 - Mayo, 16)
\end{itemize}

Milestone: investigación teórica
\begin{itemize}
    \item Diferencias entre modelos investigación modelado matemático variables y parámetros (Febrero, 24 - Abril, 9)
    \item Estudiar modelos conceptos epidemiológicos investigación (Febrero, 19 - Marzo, 4)
    \item Compartimento vacuna investigación vacunación (Febrero, 19 - Mayo, 14)
\end{itemize}

Milestone: Control
\begin{itemize}
    \item Vacuna análisis mejora modelo simulink vacunación (Mayo, 16 - Mayo, 19)
    \item Medidas de control conceptos epidemiológicos documentation investigación mejora modelo (Mayo, 19 - Junio, 3)
    \item Regulador PID matlab mejora modelo PID simulación (Junio, 5 - Junio, 6)
\end{itemize}

Milestone: Aplicación
\begin{itemize}
    \item Desing App aplicación documentation investigación matlab (Mayo, 16 - Mayo, 19)
    \item Implementar aplicación aplicación matlab modelado matemático modelo epidemiológico simulación (Mayo, 19 - Junio, 3)
\end{itemize}

Milestone: LATEX
\begin{itemize}
    \item Memoria documentation investigación modelado matemático simulación (Mayo, 7 - Junio, 6)
    \item Anexos documentation investigación matlab simulink (Mayo, 16 -  Junio, )
\end{itemize}

\section{Planificación temporal}

Para la planificación temporal del proyecto se ha seguido una metodología ágil inspirada en \textit{Scrum}, adaptada a un entorno de trabajo individual. Se han utilizado herramientas de GitHub como los \textit{milestones}, \textit{issues} y \textit{labels}. Esta metodología ha favorecido una entrega progresiva de resultados, una mejora continua durante las iteraciones y una documentación detallada en cada fase del desarrollo.

Todas las tareas y etapas del proyecto han sido organizadas y gestionadas a través del repositorio de GitHub\footnote{\url{https://github.com/Luciasegura/TFG}}, donde se encuentran documentados los \textit{milestones}, tareas, versiones del código y los avances alcanzados.

\subsection*{Estructura de trabajo}

\begin{itemize}
    \item \textbf{Milestones:} fases principales del proyecto que agrupan varias tareas con una fecha límite común.
    \item \textbf{Issues:} tareas concretas asociadas a cada milestone.
    \item \textbf{Labels:} etiquetas que clasifican las issues por tipo de trabajo, como investigación, documentación, implementación, etc.
\end{itemize}

\subsection*{Resumen de planificación}

\subsubsection*{1. Investigación teórica \hfill \textnormal{\textit{(19 feb. – 14 may.)}}}
\begin{itemize}
    \item Estudiar modelos: conceptos epidemiológicos \hfill \textit{(19 feb. – 4 mar.)}
    \item Diferencias entre modelos: modelado matemático, variables y parámetros \hfill \textit{(24 feb. – 9 abr.)}
    \item Compartimento vacuna: estudio sobre vacunación \hfill \textit{(19 feb. – 14 may.)}
\end{itemize}

\subsubsection*{2. Implementación en Simulink \hfill \textnormal{\textit{(6 mar. – 16 may.)}}}
\begin{itemize}
    \item Efecto de parámetros: análisis y mejora del modelo \hfill \textit{(6 mar. – 18 abr.)}
    \item Implementar modelos epidemiológicos \hfill \textit{(6 mar. – 1 abr.)}
    \item Añadir vacunación al modelo en Simulink \hfill \textit{(6 mar. – 16 may.)}
\end{itemize}

\subsubsection*{3. Control del modelo \hfill \textnormal{\textit{(16 may. – 6 jun.)}}}
\begin{itemize}
    \item Análisis del efecto de la vacuna en el modelo \hfill \textit{(16 may. – 19 may.)}
    \item Medidas de control: investigación y documentación \hfill \textit{(19 may. – 3 jun.)}
    \item Implementación del regulador PID en MATLAB \hfill \textit{(5 jun. – 6 jun.)}
\end{itemize}

\subsubsection*{4. Desarrollo de la aplicación \hfill \textnormal{\textit{(16 may. – 3 jun.)}}}
\begin{itemize}
    \item Diseño inicial de la app \hfill \textit{(16 may. – 19 may.)}
    \item Implementación funcional de la app en MATLAB \hfill \textit{(19 may. – 3 jun.)}
\end{itemize}

\subsubsection*{5. Documentación en \LaTeX \hfill \textnormal{\textit{(7 may. – 6 jun.)}}}
\begin{itemize}
    \item Redacción de la memoria principal \hfill \textit{(7 may. – 6 jun.)}
    \item Elaboración de anexos \hfill \textit{(16 may. – 6 jun.)}
\end{itemize}



\section{Planificación económica}

\subsection{Costes de hardware}

\begin{table}[H]

\centering
\small
\begin{tabular}{|l|l|c|c|}
\hline
\textbf{Concepto} & \textbf{Descripción} & \textbf{Coste total (€)} & \textbf{Coste imputado (€)} \\
\hline
Portátil & Asus Vivobook X409JB & 650 & 350 \\
\hline
\end{tabular}
\caption{Costes de hardware imputados al TFG}
\end{table}

\subsection{Costes de Software}

\begin{table}[H]
\centering
\small
\begin{tabular}{|l|l|c|c|c|}
\hline
\textbf{Concepto} & \textbf{Licencia} & \textbf{Coste (€)} & \textbf{Uso TFG} & \textbf{Imputado (€)} \\
\hline
Windows & Windows 10 Pro & 150 & 50\% & 75 \\
Office & Office 365 (1 año) & 69 & 100\% & 69 \\
MATLAB & Educativa estudiante & 150 & 100\% & 150 \\
Simulink & Incluida en MATLAB & 0 & 100\% & 0 \\
\hline
\end{tabular}
\caption{Costes de software imputados al TFG}
\end{table}

\subsection{Costes de Mano de Obra}

\begin{table}[H]
\centering
\begin{tabular}{|l|p{5cm}|c|c|c|}
\hline
\textbf{Concepto} & \textbf{Detalle} & \textbf{Horas} & \textbf{Precio/hora (€)} & \textbf{Coste total (€)} \\
\hline
Desarrollo app & Ingeniero informático & 360 & 20 & 7.200 \\
\hline
\end{tabular}
\caption{Costes de mano de obra imputados al TFG}
\end{table}

\subsection{Resumen económico total}

\begin{table}[h]
\centering
\begin{tabular}{|l|c|}
\hline
\textbf{Concepto} & \textbf{Coste imputado (€)} \\
\hline
Hardware & 450 \\
Software & 294 \\
Mano de obra & 7.200 \\
\hline
\textbf{Total general} & \textbf{7.944} \\
\hline
\end{tabular}
\caption{Resumen económico total del TFG}
\end{table}

\section{Viabilidad legal}

Esta sección analiza los aspectos legales relacionados con el desarrollo del proyecto, con el fin de garantizar que se ha actuado conforme a la normativa vigente en materia de propiedad intelectual, uso de software, protección de datos y cumplimiento del reglamento académico.

\subsection{Propiedad intelectual y licencias}

Durante el desarrollo del trabajo se ha hecho uso de diversas herramientas y recursos, tanto propios como de terceros. Para asegurar la legalidad del uso de estos elementos, se ha revisado la licencia de cada uno de ellos:

\begin{itemize}
    \item \textbf{MATLAB y Simulink:} se han utilizado bajo licencia educativa proporcionada por la universidad, lo cual permite el uso no comercial y con fines académicos del software. Esta licencia limita su uso a entornos de formación e investigación.
    \item \textbf{Microsoft Windows 10 y Office 365:} se han empleado versiones con licencia personal, válidas para su uso en el ámbito educativo y no profesional, sin vulnerar las condiciones de uso del proveedor.
    \item \textbf{Repositorios y herramientas de desarrollo:} el código fuente del proyecto se encuentra alojado en GitHub. Se han seguido las recomendaciones de uso de la plataforma y se ha seleccionado una licencia de software libre (\texttt{MIT License}) que permite la reutilización, modificación y distribución del código, siempre que se otorgue reconocimiento al autor original.
    \item \textbf{Librerías y recursos externos:} todas las librerías de software utilizadas en el desarrollo de la aplicación han sido seleccionadas por estar bajo licencias de código abierto (como MIT o GPL), lo que permite su uso legal en proyectos académicos. En los casos donde se han empleado fragmentos de código o documentación externa, se ha incluido la correspondiente cita y fuente.
\end{itemize}

\subsection{Protección de datos}

Durante el desarrollo del proyecto se han utilizado datos relacionados con enfermedades reales, obtenidos a partir de fuentes públicas y abiertas disponibles en internet, como artículos científicos, bases de datos institucionales y documentos divulgativos. 

En todo momento se ha respetado la legalidad vigente en materia de protección de datos. Los datos empleados no contienen información personal identificable, por lo que no se consideran datos personales según el Reglamento (UE) 2016/679 (RGPD) \cite{ue2016rgpd} y la Ley Orgánica 3/2018 (LOPDGDD) \cite{lopdgdd2018}.

Además, se ha prestado especial atención a que las fuentes de donde se han extraído dichos datos sean de carácter público, accesibles libremente por cualquier usuario y con fines divulgativos, científicos o educativos. No se ha requerido consentimiento alguno al no tratarse de información sensible vinculada a individuos concretos, sino a descripciones generales de patologías, síntomas, prevalencias u otras variables de carácter epidemiológico.
Por tanto, se concluye que el uso de estos datos es legal y éticamente aceptable en el marco de un trabajo académico con fines formativos y no comerciales.

\subsection{Normativa académica y originalidad}

El trabajo ha sido realizado de forma íntegra y original por la autora del TFG, sin incurrir en plagio, copia de código sin atribución o uso indebido de materiales de terceros.
Todo el desarrollo se ha realizado conforme a las normas establecidas por la UBU\footnote{Universidad de Burgos} en relación con la elaboración de Trabajos Fin de Grado, asegurando la autoría propia, la honestidad académica y el cumplimiento del código ético de la institución.

\subsection{Licencia del proyecto}

El proyecto se ha publicado bajo la \textbf{licencia MIT}, una licencia permisiva ampliamente utilizada en proyectos de software libre. Esta licencia permite:

\begin{itemize}
    \item Uso, copia, modificación y distribución del software de forma gratuita.
    \item Inclusión del software en otros proyectos, incluidos proyectos comerciales.
    \item Obligación de incluir el aviso de copyright original y la licencia en todas las copias o partes sustanciales del software.
\end{itemize}




