\apendice{Plan de Proyecto Software}

\section{Introdicción}
En este proyecto se ha llevado a cabo el diseño y simulación de modelos epidemiológicos deterministas utilizando Simulink, así como el desarrollo de una aplicación interactiva en App Designer para visualizar los resultados. Para organizar adecuadamente el trabajo, se ha realizado una planificación que contempla los aspectos temporales (distribución de tareas), una estimación básica del coste, y una breve revisión de los aspectos legales relacionados con el uso del software empleado.


\section{Planificación temporal}
Para la planificación temporal del proyecto se ha seguido una metodología ágil inspirada en Scrum, adaptada a un entorno de trabajo individual. En lugar de equipos y reuniones formales, se han utilizado herramientas de GitHub, como los issues (para desglosar tareas concretas) y los milestones (para agrupar tareas relacionadas y marcar etapas del desarrollo).

Cada milestone representa una fase clave del proyecto, compuesta por tareas específicas (issues) que han sido organizadas y etiquetadas según su naturaleza. Esta estructura ha permitido una gestión flexible del tiempo, facilitando un seguimiento claro del avance del proyecto, el estado de cada tarea y su porcentaje de finalización.

Esta metodología ha favorecido una entrega progresiva de resultados, una mejora continua a lo largo de las iteraciones, y una documentación detallada de cada fase del desarrollo.

Todas las tareas y etapas del proyecto han sido organizadas y gestionadas a través del repositorio de GitHub\footnote{Link para entrar en el repositorio https://github.com/Luciasegura/TFG}. En él se encuentran documentados los milestones, issues, versiones del código y los avances del proyecto de forma detallada.





\section{Planificación económica}

\section{Viabilidad legal}

