\capitulo{1}{Introducción}
Las enfermedades infecciosas han sido una de las principales causas de mortalidad y morbilidad a lo largo de la historia de la humanidad, representando un desafío constante para la salud pública mundial. En los últimos años, la aparición de nuevas epidemias y pandemias ha puesto de manifiesto la necesidad de herramientas efectivas para comprender, predecir y controlar la propagación de estos agentes infecciosos. En este contexto, el modelado matemático se ha consolidado como una herramienta fundamental para el estudio y la gestión de epidemias.

Los modelos epidemiológicos deterministas, basados en sistemas de ecuaciones diferenciales, permiten describir la evolución temporal de una enfermedad en una población, dividiendo a los individuos en diferentes categorías según su estado de salud. Entre los modelos más estudiados se encuentran el SI, SIS, SIR y SEIR, cada uno con características y aplicaciones particulares según el tipo de enfermedad y su dinámica.

Este trabajo se centra en el análisis y comparación de estos modelos clásicos, con especial atención a la incorporación de la vacunación en los modelos SIR y SEIR, asumiendo que la inmunidad inducida por la vacunación es similar a la adquirida tras la recuperación. Asimismo, se desarrolla un controlador PID aplicado al modelo SIR, que simula medidas de intervención como cuarentenas, con el objetivo de evaluar estrategias de control para mitigar la propagación del contagio.

Para validar y comparar los modelos, se han utilizado tanto datos simulados como datos reales, facilitando una visión completa de su comportamiento en diferentes escenarios. Además, se ha diseñado una aplicación gráfica que permite visualizar los resultados de manera interactiva.

