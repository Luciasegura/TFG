\capitulo{2}{Objetivos}

Antes de abordar el desarrollo tanto teórico como práctico del trabajo, es fundamental establecer de forma clara los objetivos que guiarán las etapas del proyecto. Dado el carácter técnico y aplicado del estudio, se han dividido los objetivos en tres categorías. La clasificación va a permitir no solo estructurar el alcance del proyecto desde el punto de vista académico y técnico, sino que también va a reflejar el crecimiento personal y profesional que se espera alcanzar con la realización del trabajo.

\section{Objetivo general}

El objetivo principal de este trabajo es el análisis, diseño e implementación de modelos epidemiológicos deterministas mediante herramientas de simulación tanto en \texttt{MATLAB} como \texttt{Simulink}, con el propósito de estudiar la propagación de enfermedades infecciosas, evaluar estrategias de control y mejorar la comprensión del comportamiento dinámico de una epidemia.

Para ello, se utilizarán tanto parámetros hipotéticos como datos reales extraídos de enfermedades conocidas, con el fin de seleccionar el modelo más adecuado para cada caso y simular su evolución. Como parte importante del proyecto, se desarrollará una aplicación interactiva para facilitar la visualización y el análisis de los resultados obtenidos a partir de los modelos simulados.

\section{Objetivos específicos}

\begin{itemize}
  \item Implementar y comparar diferentes modelos compartimentales (SI, SIS, SIR, SEIR, además de variantes con vacunación) en \texttt{Simulink}, incluyendo sus sistemas de ecuaciones diferenciales.
  \item Analizar el significado de los diferentes parámetros de los modelos (beta, gamma, \(R_0\)…) y estudiar cómo afectan a la evolución temporal de la enfermedad en los diferentes modelos.
  \item Simular diferentes escenarios utilizando parámetros hipotéticos y reales, evaluando la capacidad predictiva del modelo.
  \item Identificar enfermedades reales que puedan ajustarse adecuadamente a cada uno de los modelos epidemiológicos estudiados, recopilando sus datos históricos y simulando su comportamiento en \texttt{Simulink} con los parámetros correspondientes.
  \item Estudiar e incorporar al modelo epidemiológico un compartimento asociado a la vacunación, analizando su efecto sobre la evolución de la epidemia y las condiciones necesarias para lograr la inmunidad colectiva.
  \item Estudiar el efecto de modificar parámetros clave para observar posibles estrategias de control y su impacto sobre el comportamiento de la epidemia.
  \item Explorar la aplicación de técnicas de ingeniería de control que permitan influir activamente sobre la evolución de la epidemia.
  \item Diseñar y desarrollar una aplicación interactiva en \texttt{Design App} que permita al usuario introducir parámetros, visualizar en tiempo real los resultados de la simulación y analizar gráficamente la evolución del sistema.
\end{itemize}

\section{Objetivos personales}

\begin{itemize}
  \item Profundizar en el conocimiento y aplicación práctica de modelos matemáticos en el ámbito de la ingeniería biomédica y la epidemiología, desarrollando la capacidad de traducir fenómenos reales en representaciones matemáticas comprensibles.
  \item Mejorar las habilidades técnicas en \texttt{MATLAB} y \texttt{Simulink}, especialmente en el diseño de modelos dinámicos, simulación de sistemas y creación de aplicaciones interactivas mediante \texttt{Design App}.
  \item Familiarizarse con el tratamiento, análisis e interpretación de datos reales asociados a enfermedades infecciosas, y aprender a integrarlos en modelos para ajustar sus parámetros y predecir su evolución.
  \item Desarrollar habilidades en visualización de datos y comunicación de resultados científicos a través de herramientas gráficas, mejorando la capacidad de transmitir información técnica de forma accesible.
  \item Aprender a utilizar \texttt{LaTeX} como herramienta profesional para la redacción de documentos técnicos y científicos, adquiriendo autonomía en la generación de textos estructurados, con fórmulas, referencias y formato académico.
\end{itemize}

