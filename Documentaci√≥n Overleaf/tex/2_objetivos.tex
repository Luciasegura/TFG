\capitulo{2}{Objetivos}

Antes de abordar el desarrollo tanto teórico como práctico del trabajo, es fundamental establecer de forma clara los objetivos que guiarán las etapas del proyecto. Dado el carácter técnico y aplicado del estudio, se han dividido los objetivos en tres categorías. La clasificación va a permitir no solo estructurar el alcance del proyecto desde el punto de vista académico y técnico, sino que también va a reflejar el crecimiento personal y profesional que se espera alcanzar con la realización del trabajo.

\section{Objetivo general}
El objetivo principal de este trabajo es analizar, diseñar e implementar modelos epidemiológicos deterministas utilizando herramientas de simulación como \texttt{MATLAB} y \texttt{Simulink}, con el fin de estudiar la propagación de enfermedades infecciosas, evaluar distintas estrategias de control, y favorecer una mejor comprensión del comportamiento dinámico de las epidemias.

Para lograrlo, se emplearán tanto parámetros hipotéticos como datos reales provenientes de enfermedades conocidas, con el propósito de seleccionar el modelo más adecuado en cada caso y simular su evolución. Además, como parte esencial del proyecto, se desarrollará una aplicación interactiva que facilite la visualización y el análisis gráfico de los resultados obtenidos, haciendo accesible la comprensión del modelo a usuarios sin conocimientos técnicos avanzados.

\section{Objetivos específicos}
\begin{itemize}
\item Implementar y comparar distintos modelos compartimentales (SI, SIS, SIR, SEIR y sus variantes con vacunación) en \texttt{Simulink}, definiendo sus correspondientes sistemas de ecuaciones diferenciales.
\item Analizar el significado y la influencia de los parámetros característicos de los modelos (como la tasa de transmisión, la tasa de recuperación y el número básico de reproducción) en la evolución temporal de la enfermedad.
\item Simular diferentes escenarios utilizando tanto parámetros hipotéticos como datos reales, con el objetivo de evaluar la capacidad predictiva de cada modelo y su adecuación a diferentes tipos de enfermedades.
\item Identificar enfermedades reales que se ajusten adecuadamente a cada uno de los modelos analizados, recopilando datos históricos fiables y simulando su comportamiento mediante parámetros representativos en \texttt{Simulink}.
\item Incorporar al modelo epidemiológico un compartimento asociado a la vacunación en los modelos SIR y SEIR, estudiando su impacto sobre la evolución de la epidemia y las condiciones necesarias para alcanzar la inmunidad colectiva.
\item Analizar el efecto de modificar parámetros clave del modelo para observar el impacto de distintas estrategias de control sanitario y políticas públicas.
\item Explorar la aplicación de técnicas de control, como el diseño de un regulador PID, que permitan intervenir de forma activa en la evolución de la epidemia mediante medidas como confinamientos o restricciones de contacto.
\item Diseñar y desarrollar una aplicación interactiva utilizando \texttt{App Designer}, que permita al usuario introducir parámetros, ejecutar simulaciones en tiempo real y visualizar gráficamente la evolución del sistema epidemiológico.
\end{itemize}

\section{Objetivos personales}
\begin{itemize}
\item Profundizar en el conocimiento y aplicación práctica de modelos matemáticos dentro del ámbito de la ingeniería biomédica y la epidemiología, adquiriendo habilidades para representar fenómenos reales mediante formulaciones matemáticas comprensibles.
\item Perfeccionar el manejo de herramientas como \texttt{MATLAB} y \texttt{Simulink}, especialmente en lo relativo al diseño de modelos dinámicos, simulación de sistemas y desarrollo de interfaces interactivas mediante \texttt{App Designer}.
\item Familiarización con el tratamiento, análisis e interpretación de datos reales asociados a enfermedades infecciosas, aprendiendo a integrarlos eficazmente en modelos para ajustar parámetros y realizar predicciones fiables.
\item Desarrollar competencias en visualización de datos y comunicación de resultados científicos mediante herramientas gráficas, mejorando así la capacidad para transmitir información técnica de manera clara y accesible.
\item Adquirir soltura en el uso de \texttt{LaTeX} como herramienta profesional para la redacción de documentos científicos y técnicos, logrando autonomía en la elaboración de textos estructurados, con contenido matemático y estilo académico.
\end{itemize}

