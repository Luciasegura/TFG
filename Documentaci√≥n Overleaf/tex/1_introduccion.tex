\capitulo{1}{Introducción}
Las enfermedades infecciosas han acompañado al ser humano desde sus orígenes, generando crisis sanitarias de gran impacto a lo largo de la historia. Epidemias como la peste negrea, la gripe española o la pandemia de COVID-19, han puesto a prueba a los sistemas de salud pública y han motivado al desarrollo de modelos matemáticos capaces de predecir su evolución y apoyar a la toma de decisiones sanitarias. La modelización epidemiológica es una herramienta clave para comprender la dinámica de transmisión de enfermedades infecciosas, estimar la magnitud de los brotes y evaluar la eficacia de intervenciones como la vacunación, el confinamiento o el distanciamiento social.


Entre los modelos existentes, los modelos deterministas han demostrado ser especialmente útiles por su simplicidad y capacidad para captar la dinámica general de una epidemia. En este tipo de modelos, la población se divide en compartimentos, y se describe el paso de individuos entre estos grupos mediante ecuaciones diferenciales ordinarias. Según el número de compartimentos y de relaciones entre ellos, surgen diferentes modelos: SI, SIS, SIR, SEIR, entre otros. Permiten obtener indicadores clave como el número reproductivo básico, la duración estimada de la epidemia o el pico de contagios.


En este proyecto se propone el estudio, simulación y análisis de varios de estos modelos epidemiológicos deterministas utilizando las herramientas proporcionadas por \texttt{MATLAB} y \texttt{Simulink}. Estas plataformas permiten implementar visualmente los sistemas dinámicos y modificar sus parámetros, lo que es ideal para simular diferentes escenarios epidemiológicos y observar su evolución bajo distintas condiciones iniciales. Además, \texttt{Simulink} ofrece herramientas para el desarrollo de interfaces gráficas \texttt{(App Designer)}, lo que va a facilitar la visualización de los resultados y la comprensión del modelo por parte de usuarios no especializados.


El trabajo no solo tiene como finalidad construir e implementar modelos epidemiológicos con \texttt{Simulink}, sino que también analiza la influencia de los parámetros clave en el comportamiento del sistema, explorar enfermedades con datos reales y estudiar la posibilidad de aplicar técnicas de ingeniería de control para modificar la evolución del brote. El objetivo es tratar a la epidemia como un sistema dinámico susceptible de ser controlado, en el que intervenciones externas pueden actuar como entradas reguladoras del sistema.


Por tanto, este proyecto se sitúa en la intersección de la ingeniería, la matemática aplicada y la epidemiología, con un enfoque práctico orientado a la simulación, análisis y control de epidemias mediante herramientas modernas de modelado. El proyecto pretende no solo aportar una visión técnica del problema, sino también desarrollar habilidades transversales y proponer una aproximación ingenieril a fenómenos con un fuerte impacto en la sociedad.
