\apendice{Anexo de sostenibilidad curricular}

Este proyecto está centrado en el modelado matemático de epidemias mediante modelos compartimentales y su implementación en \texttt{MATLAB} y \texttt{Simulink}, se vincula de forma directa con diversos ODS\footnote{Objetivos de Desarrollo Sostenible.} establecidos en la Agenda 2030 de las Naciones Unidas \cite{onu2015agenda2030}.

\section*{ODS relacionados}

\begin{itemize}
    \item \textbf{ODS 3: Salud y Bienestar.} El proyecto contribuye al entendimiento de la propagación de enfermedades infecciosas y la evaluación de estrategias de control como la vacunación y las cuarentenas, aspectos clave para mejorar la resiliencia de los sistemas sanitarios ante futuras pandemias \cite{who2020disorder}.
    
    \item \textbf{ODS 4: Educación de Calidad.} La creación de una aplicación interactiva para la visualización y manipulación de los modelos permite democratizar el acceso al conocimiento técnico, facilitando la comprensión de dinámicas epidemiológicas incluso a personas sin formación especializada. Esto potencia el aprendizaje autónomo y el desarrollo de competencias digitales \cite{unesco2020education}.
    
    \item \textbf{ODS 10: Reducción de las Desigualdades.} Al permitir la simulación y análisis de medidas sanitarias en distintos escenarios, el trabajo también reflexiona sobre cómo las epidemias afectan de manera desigual a distintas poblaciones. El desarrollo de herramientas accesibles y adaptables puede contribuir a reducir brechas tecnológicas y de conocimiento \cite{who2021inequality}.
\end{itemize}

\section*{Contribuciones a la sostenibilidad curricular}

El trabajo fomenta diversas competencias clave para el desarrollo sostenible, alineadas con los principios de la sostenibilidad curricular en educación superior \cite{lozano2017teaching}. Entre ellas destacan:

\begin{itemize}
    \item \textbf{Pensamiento crítico y sistémico:} al modelar fenómenos complejos como la dinámica epidémica desde un enfoque cuantitativo e interdisciplinar.
    \item \textbf{Innovación responsable:} mediante el diseño de herramientas digitales orientadas a la educación y la toma de decisiones en salud pública.
    \item \textbf{Conciencia ética y social:} al abordar la dimensión humana y desigual de las pandemias, así como la necesidad de respuestas colaborativas y equitativas.
\end{itemize}

En conjunto, este trabajo de fin de grado no solo aporta un enfoque técnico y analítico al estudio de epidemias, sino que también contribuye a una visión integral de la sostenibilidad, reconociendo la interdependencia entre ciencia, tecnología, salud pública y justicia social.
De este modo, el TFG se alinea con los principios de una educación superior comprometida con la Agenda 2030 y con la formación de profesionales capaces de abordar los desafíos globales desde una perspectiva ética y sostenible.


